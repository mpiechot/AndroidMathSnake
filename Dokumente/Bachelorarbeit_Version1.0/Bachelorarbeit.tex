
%%%%%%%%%%%%%%%%%%%%%%%%%%%%%%%%%%%%%%%%%%%%%%%%%%%%%%%%%%%%%%%%%%%%%%%
%% Haupteil des Dokuments -Main-                                     %%
%%%%%%%%%%%%%%%%%%%%%%%%%%%%%%%%%%%%%%%%%%%%%%%%%%%%%%%%%%%%%%%%%%%%%%%


%% Basierend auf einer TeXnicCenter-Vorlage von Mark M�ller
%%%%%%%%%%%%%%%%%%%%%%%%%%%%%%%%%%%%%%%%%%%%%%%%%%%%%%%%%%%%%%%%%%%%%%%

% W�hlen Sie die Optionen aus, indem Sie % vor der Option entfernen  
% Dokumentation des KOMA-Script-Packets: scrguide

%%%%%%%%%%%%%%%%%%%%%%%%%%%%%%%%%%%%%%%%%%%%%%%%%%%%%%%%%%%%%%%%%%%%%%%
%% Optionen zum Layout des Buchs                                     %%
%%%%%%%%%%%%%%%%%%%%%%%%%%%%%%%%%%%%%%%%%%%%%%%%%%%%%%%%%%%%%%%%%%%%%%%
\documentclass[
%a5paper,							% alle weiteren Papierformat einstellbar
%landscape,						% Querformat
12pt,								% Schriftgr��e (12pt, 11pt (Standard))
BCOR 1cm,							% Bindekorrektur, bspw. 1 cm
%DIVcalc,							% f�hrt die Satzspiegelberechnung neu aus
twoside ,							% s. scrguide 2.4
%oneside,							% einseitiges Layout
%twocolumn,						% zweispaltiger Satz
%openany,							% Kapitel k�nnen auch auf linken Seiten beginnen
openright,							% Kapitel k�nnen nur auf rechten Seiten beginnen
%openleft,
%halfparskip*,				% Absatzformatierung s. scrguide 3.1
%headsepline,					% Trennline zum Seitenkopf	
%footsepline,					% Trennline zum Seitenfu�
%notitlepage,					% in-page-Titel, keine eigene Titelseite
%chapterprefix,				% vor Kapitel�berschrift wird "Kapitel Nummer" gesetzt
%appendixprefix,				% Anhang wird "Anhang" vor die �berschrift gesetzt 
%normalheadings,			% �berschriften etwas kleiner (smallheadings)
%idxtotoc,						% Index im Inhaltsverzeichnis
%liststotoc,					% Abb.- und Tab.verzeichnis im Inhalt
%bibtotoc,						% Literaturverzeichnis im Inhalt
%leqno,								% Nummerierung von Gleichungen links
%fleqn,								% Ausgabe von Gleichungen linksb�ndig
%draft								% �berlangen Zeilen in Ausgabe gekennzeichnet
]
{scrbook}

\usepackage{chngcntr}
\counterwithout{equation}{chapter}
%\pagestyle{empty}		% keine Kopf und Fu�zeile (k. Seitenzahl)
%\pagestyle{headings}	% lebender Kolumnentitel  

%% Deutsche Anpassungen %%%%%%%%%%%%%%%%%%%%%%%%%%%%%%%%%%%%%
%\usepackage[english]{babel}
\usepackage[ngerman]{babel}
\usepackage[utf8]{inputenc}
%\usepackage[ansinew]{inputenc}
%\usepackage[latin1]{inputenc}
%\usepackage[T1]{fontenc}
%\usepackage[scaled]{helvet}
\usepackage{arydshln} 
%lines for tabes



%% Falls die automatische Worttrennung in W�rtern mit Umlauten
%% nicht funktionieren sollte oder der Text pixelig aussieht:
%% ==> Installieren Sie die cm-super Fonts (z.B. mit dem mikTeX Package Manager).
%% Eine nicht ganz vollwertige Alternative ist die Verwendung dieses Pakets:
\usepackage{ae, aeguill}


%% Packages f�r Grafiken & Abbildungen %%%%%%%%%%%%%%%%%%%%%%
\usepackage{graphicx} %%Zum Laden von Grafiken
%\usepackage{subfig} %%Teilabbildungen in einer Abbildung
%\usepackage{pst-all} %%PSTricks - nicht verwendbar mit pdfLaTeX

\usepackage{float} % f�r eine bessere Positionierung von Grafiken

\usepackage{wrapfig}

\usepackage{listliketab}  %f�r die getabbte itemize


\usepackage[usenames]{color}
\definecolor{rot}{RGB}{255,195,195}
\definecolor{dunkelblau}{RGB}{160,172,254}
\definecolor{hellblau}{RGB}{164,215,216}
\definecolor{gelb}{RGB}{254,247,205}
\definecolor{gruen}{RGB}{218,255,204}
\definecolor{lila}{RGB}{205,181,217}
\definecolor{grey}{RGB}{200,200,200}

\usepackage{colortbl}    %um Tabellen farbig zu machen

%% Beachten Sie:
%% Die Einbindung einer Grafik erfolgt mit \includegraphics{Dateiname}
%% bzw. �ber den Dialog im Einf�gen-Men�.
%% 
%% Im Modus "LaTeX => PDF" k�nnen Sie u.a. folgende Grafikformate verwenden:
%%   .jpg  .png  .pdf  .mps
%% 
%% In den Modi "LaTeX => DVI", "LaTeX => PS" und "LaTeX => PS => PDF"
%% k�nnen Sie u.a. folgende Grafikformate verwenden:
%%   .eps  .ps  .bmp  .pict  .pntg


%% Bibliographiestil %%%%%%%%%%%%%%%%%%%%%%%%%%%%%%%%%%%%%%%%%%%%%%%%%%
%\usepackage{natbib}



%%  (Franz Gritschneder hinzugefuegte Bibliotheken)

\usepackage{setspace}

\usepackage{fancyhdr}							%f�r Kopf- und Fusszeilen

\usepackage{booktabs}							% Absatz Tabellenumgebung





\usepackage{lettrine}   % f�r Initialen an einem Abschnitt z.B. \lettrine[lines=1]{H}{}auptaug 

%\usepackage[a4paper,
%left=3cm, right=3cm,
%top=3cm,
% bottom=2cm]{geometry}


\usepackage{yfonts}   %Initialen

%\usepackage[T1]{fontenc} % �ndert die Fontkodierung auf T1 Format
\usepackage{pdfpages}
%\ pdf einbinden

%\parindent0mm
\setlength{\parindent}{0pt}



%\usepackage[ps2pdf,%
%linktocpage,%
%colorlinks,%
%bookmarks,%
%bookmarksopen,%
%bookmarksnumbered]{hyperref}



\usepackage{multirow}    % f�r die Tabellenumgebung
%\usepackage{multicolumn} % f�r die Tabellenumgebung - 
\usepackage{tabularx}

\setcounter{secnumdepth}{3}  %gibt die Tiefe der Nummerierung f�r Unterpunkte an

%\chapter{Kapitel1}
%\section{Punkt1}
%\subsection{Unterpunkt1}
%\subsubsection{Unterpunkt2}
%\paragraph{Unterpunkt3}
%\subparagraph{Unterpunkt4


\newcommand{\entspr}{\ensuremath \widehat{=}}

%\setlength{\parindent}{1.4em}
\setlength{\parindent}{0pt} %kein Einzug auf der 1. Zeile eines Absatzes


\usepackage{textcomp}
 %Mathematikbefehle anzeigen
%\usepackage{mathtools} 
 \usepackage{amsmath}
 \numberwithin{equation}{chapter}
 \usepackage{amssymb}
 
 %Zur Erzeugung von Index
 \usepackage{makeidx}
 %Index erstellen
 \makeindex
 
  %captionbefehle einbingen
 \usepackage{caption} 
 \usepackage{array}
 
 \usepackage{pgfplots}				% für Plots direkt in Latex
 \usepackage{tikz}					% Umgebung für Plots
 \usetikzlibrary{patterns}			% Füllungen von Plots mit Mustern

 
 \usepackage{subcaption}
 \usepackage{caption}
 
 \usepackage{bm}					% bold text in math mode
 \usepackage{nicefrac}				% schräge Brüche in Text
 
 \usepackage[colorlinks=true,       %zur anzeige der bookmarks im pdf-ausgabefile
        linkcolor=black,
        citecolor=black,
%        linkcolor=blue,
%        citecolor=red,
        filecolor=black,
        %pagecolor=black,
        urlcolor=black,
        bookmarks=true,
        bookmarksopen=false,
        bookmarksopenlevel=5,
        bookmarksnumbered,
        plainpages=false,
        pdfpagelabels=true]{hyperref}
\usepackage{hyperref}
\hypersetup{
pdftitle = {TBD},
pdfsubject = {Automatisierte Erzeugung von Zustandsautomaten},
pdfauthor = {Jan Martin},
pdfkeywords = {Masterarbeit, tbd},
pdfcreator = {GNU Ghostscript 8.00},
pdfproducer = {LaTeX},
urlcolor = black
}

 

 
 

\makeatletter
\renewenvironment{thebibliography}[1]
{\chapter{Bibliography}%
%{\chapter{\refname}%
\@mkboth{\MakeUppercase\refname}{\MakeUppercase\refname}%
\list{\@biblabel{\@arabic\c@enumiv}}%
{\settowidth\labelwidth{\@biblabel{#1}}%
\leftmargin\labelwidth
\advance\leftmargin\labelsep
\@openbib@code
\usecounter{enumiv}%
\let\p@enumiv\@empty
\renewcommand\theenumiv{\@arabic\c@enumiv}}%
\sloppy
\clubpenalty4000
\@clubpenalty \clubpenalty
\widowpenalty4000%
\sfcode`\.\@m}
{\def\@noitemerr
{\@latex@warning{Empty `thebibliography' environment}}%
\endlist}
\makeatother


\renewcommand{\headrulewidth}{0.00pt}    %Dicke der Kopflinie
\renewcommand{\footrulewidth}{0.00pt}    %Dicke der Fusslinie

\setlength{\textheight}{630pt}%630
\setlength{\textwidth}{450pt}
%\setlength{\textwidth}{17cm}
\setlength{\headsep}{40pt}
\setlength{\headheight}{50pt}%45


% Kopf- und Fusszeile definieren
% Definition fuer zweiseitige Anordnung ( aus wikipedia.org )
% ---------------------------------------------------------------------------------------------------
%\lhead[<lh-even>]{<lh-odd>}   |	   \chead[<ch-even>]{<ch-odd>}    |    	\rhead[<rh-even>]{<rh-odd>}
% ....
% ...
% ..
% .
%\lfoot[<lf-even>]{<lf-odd>} 	 |     \cfoot[<cf-even>]{<cf-odd>} 	  |     \rfoot[<rf-even>]{<rf-odd>}
% ---------------------------------------------------------------------------------------------------
% \thepage 	number of the current page
% \leftmark 	current chapter name printed like "CHAPTER 3. THIS IS THE CHAPTER TITLE"
% \rightmark 	current section name printed like "1.6. THIS IS THE SECTION TITLE"
% \chaptername 	the name chapter in the current language. If this is English, it will display "Chapter"
% \thechapter 	current chapter number
% \thesection 	current section number





% ********************************************  K O P F *********************************************
%      links gerade Seitenzahl                                      links UNgerade Seitenzahl              
\lhead[ \thepage ]                                                  { }   
%      mitte gerade Seitenzahl                                      mitte UNgerade Seitenzahl
\chead[\leftmark]                                                   {\leftmark}
%      rechts gerade Seitenzahl                                     rechts UNgerade Seitenzahl
\rhead[         ]                                                   { \thepage} 



% ********************************************  F U S S *********************************************
%      links gerade Seitenzahl                                      links UNgerade Seitenzahl
\lfoot[]                                                            {} 	
%      mitte gerade Seitenzahl                                      mitte UNgerade Seitenzahl
\cfoot[]                                                            {}
%      rechts gerade Seitenzahl                                     rechts UNgerade Seitenzahl
\rfoot[]                                                            {}
% ****************************************************************************************************

\pagestyle{fancy}



\begin{document}

\hyphenation{Punkttypen Test-ab-de-ckung Test-endkri-te-ri-um Test-endkri-te-ri-en Test} %besondere Worttrennungen
%%%%%%%%%%%%%%%%%%%%%%%%%%%%%%%%%%%%%%%%%%%%%%%%%%%%%%%%%%%%%%%%%%%%%%%
%% Ihr Buch                                                          %%
%%%%%%%%%%%%%%%%%%%%%%%%%%%%%%%%%%%%%%%%%%%%%%%%%%%%%%%%%%%%%%%%%%%%%%%
%% Schmutztitel-Seite %%%%%%%%%%%%%%%%%%%%%%%%%%%%%%%%%%%%%%%%%%%%%%%%%
%\extratitle{Schmutztitel}

%% eigene Titelseitengestaltung %%%%%%%%%%%%%%%%%%%%%%%%%%%%%%%%%%%%%%%    
%\begin{titlepage}
%Einsetzen der TXC Vorlage "Deckblatt" m?glich
%\end{titlepage}

%\include{chapters/Deckblatt}

%% Angaben zur Standardformatierung des Titels %%%%%%%%%%%%%%%%%%%%%%%%
%\titlehead{Titelkopf}
%\subject{Typisierung}
%%%%%%%%%%%%%%%%%%%%%%%%%%%%%%%%%%%%%%%%%%%%%%%%%%%%%%
%\title{Der Name Ihrer Arbeit}
%%%%%%%%%%%%%%%%%%%%%%%%%%%%%%%%%%%%%%%%%%%%%%%%%%%%%%
%\author{Ihr Name}
%\and{Der Name des Co-Autoren}
%\thanks{Fußnote}			% entspr. \footnote im Flie?text
%\date{}					% falls anderes, als das aktuelle gewünscht
%\publishers{Herausgeber}

%% R?ckseite der Titelseite %%%%%%%%%%%%%%%%%%%%%%%%%%%%%%%%%%%%%%%%%%%
%\uppertitleback{Titelrückseitenkopf}
%\lowertitleback{Titelrückseitenfuß}

%% Widmungsseite %%%%%%%%%%%%%%%%%%%%%%%%%%%%%%%%%%%%%%%%%%%%%%%%%%%%%%
%\dedication{Widmung}

%\maketitle 						% Titelei wird erzeugt

%% %%%%%%%%%%%%%%%%%%%%%%%%%%%%%%%%%%%%%%%%%%%%%%%%%%%%%%%%%%%%%%%%%%%%

\begingroup %group removes the 1.page of chapter on right side -> empty pages
\renewcommand{\cleardoublepage}{}
\renewcommand{\clearpage}{}

\frontmatter   %einleitungsteil   % Vorspann (z.B. roemische Seitenzahlen)





\begin{titlepage}

\includepdf[fitpaper]{chapters/second.pdf}

%\vspace{10cm}
%%\vspace*{\fill}
%\begingroup
%\centering
%{\large
%\textbf{Eigenschaftsprognose von Kabelbündeln\\ mit modernen\\ Methoden der Ähnlichkeitsmechanik}\\}
%\vspace{3cm}
%
%Bei der Fakultät Luft- und Raumfahrttechnik und Geodäsie der Universität Stuttgart zur Erlangung des Abschlusses Master of Science (Msc.) eingereichte Abhandlung
%
%\vspace{3cm}
%vorgelegt von\\
%
%Benedikt Walter\\
%\vspace{5cm}
%
%Betreuer: PD Dr.-Ing. habil. Stephan Rudolph\\
%\vspace{2cm}
%
%Tag der Einreichung:\\
%\vspace{2cm}
%
%Institut für Statik und Dynamik der Luft- und Raumfahrtkonstruktionen\\
%Universität Stuttgart\\
%2015
%
%\endgroup

\end{titlepage}
	

\onehalfspacing % zeilen abstand 1.5
\thispagestyle{empty}
\includepdf[fitpaper]{chapters/second.pdf}
\thispagestyle{empty}
\newpage
\thispagestyle{empty}
%\includepdf[fitpaper]{images/aufgabe(ueberarbeitet).pdf}

%\chapter[Clustern]{}
%\vspace{-145pt}
%\section*{\centering Clustern ''ähnlicher'' Testfälle mittels eines Fallbasierten Schließverfahrens}

\begingroup
\begin{sloppy}
	{\centering\textbf{\large TBD}
		
		\vspace{6pt}
		%{\centering Maximilian Schilling
		%\vspace{20pt}
		{Jan Martin}
		\vspace{20pt}
		
		\textbf{\normalsize Aufgabenstellung}}
	
	\vspace{12pt}
	TBD
	
	\vspace{12pt}
	Die M.Sc.-Thesis umfasst dabei folgende Arbeitspunkte:
	\begin{itemize}
		\setlength{\itemsep}{-3pt}
		\item[-] TBD
		\item[-] TBD
		\item[-] TBD
		\item[-] TBD
		\item[-] TBD
		\item[-] TBD
		\item[-] TBD
	\end{itemize}
	
	Die M.Sc.-Thesis wird bei der Daimler AG in der Konzernforschung PKW am Standort Sindelfingen erstellt. Die beteiligten Partner sind die Universität Stuttgart (Institut für Statik und Dynamik der Luft- und Raumfahrttechnik) und die Daimler AG (RD/FIT).
	
	\vspace{12pt}
	Bearbeitungszeitraum: 03.11. -- 03.05.2017\\
	Prüfer: Priv.-Doz. Dr.-Ing. Stephan Rudolph (ISD)\\
	Betreuer: Benedikt Walter (RD/FIT)
	
	
	\vspace{12pt}
	Note:
\end{sloppy}


\endgroup

\newpage
\thispagestyle{empty} 
\includepdf[fitpaper]{chapters/second.pdf}
\thispagestyle{empty}
\newpage
\thispagestyle{empty}

%\vspace{550cm}
%\vspace*{\fill}
\begingroup
\large
\textbf{\LARGE Kurzfassung}\\
\parskip 12pt \\


\endgroup



\newpage
\thispagestyle{empty}

%\vspace{550cm}
%\vspace*{\fill}
\begingroup
\large

\textbf{\LARGE Abstract}\\
\parskip 12pt \\

\endgroup
\vspace*{\fill}
\newpage





\singlespacing %zeilenabstand 1.0
%% Erzeugung von Verzeichnissen %%%%%%%%%%%%%%%%%%%%%%%%%%%%%%%%%%%%%%%
\setcounter{page}{1} % seitenzahl ab inhatsverzeichniss
\addtocontents{toc}{\protect\thispagestyle{fancy}} 
\setcounter{secnumdepth}{3} % Nummerierungstiefe
\setcounter{tocdepth}{2}	% Tiefe/Ebenen Inhaltsverzeichnis
\tableofcontents			% Inhaltsverzeichnis
\newpage
\addtocontents{lof}{\protect\thispagestyle{fancy}}  
\listoffigures				% Abbildungsverzeichnis
\newpage
\addtocontents{lot}{\protect\thispagestyle{fancy}} 
\listoftables				% Tabellenverzeichnis
\endgroup



%% Der Text %%%%%%%%%%%%%%%%%%%%%%%%%%%%%%%%%%%%%%%%%%%%%%%%%%%%%%%%%%%
\onehalfspacing
%\doublespacing

%\chead[Danksagung]{Danksagung}
%\include{chapters/Danksagung}

%\include{chapters/eid}

%\cleardoublepage
%$$$$$$$$$$$$$$$$$$$$$$$$$$$$$$$$$$$$$$$$$$$$$$$$$$



\mainmatter
\newcounter{romcounter}
\setcounter{romcounter}{\value{page}}
\renewcommand{\thepage}{\arabic{page}}
\renewcommand{\chapterpagestyle}{empty} % first page of chapter now heading
\setcounter{page}{1}

\onehalfspacing

% ********************************************  K O P F *********************************************
%      links gerade Seitenzahl                                      links UNgerade Seitenzahl                  
\lhead[\thepage]														{}

%      mitte gerade Seitenzahl                                      mitte UNgerade Seitenzahl
%\chead[\rightmark ]                                                 {\leftmark}
\chead[\rightmark]													{\leftmark}
%      rechts gerade Seitenzahl                                     rechts UNgerade Seitenzahl
\rhead[]															{\thepage}

% ********************************************  F U S S *********************************************
%      links gerade Seitenzahl                                      links UNgerade Seitenzahl
\lfoot[]                                                            {} 	
%      mitte gerade Seitenzahl                                      mitte UNgerade Seitenzahl
\cfoot[]                                             				{}
%      rechts gerade Seitenzahl                                     rechts UNgerade Seitenzahl
\rfoot[]                                                            {}
% ****************************************************************************************************
\pagestyle{fancy}

%$$$$$$$$$$$$$$$$$$$$$$$$$$$$$$$$$$$$$$$$$$$$$$$$$$$$
% !TEX root = ../ausarbeitung.tex

\chapter{Einleitung}
Kinder sitzen oftmals mehrere Stunden verzweifelt vor ihren Mathe-Hausaufgaben. Das alles, um gelernten Stoff, durch weitere Übungen, zu festigen. Hat das Kind aber bereits nur noch das Spielen im Kopf, entwickelt sich die Hausaufgabenzeit zu einer langwierigen Arbeitszeit mit mäßigem Fortschritt. Aber auch in der Schule selbst können in der heutigen Zeit der Digitalisierung bessere Lernerfolge erzielt werden, wenn man sich die Technik hierfür zunutze macht. Mit dieser Bachelorarbeit möchte ich untersuchen, ob das Lernen der Addition in der Mathematik angenehmer für das Kind gestaltet werden und womöglich auch bessere Lernerfolge herbeiführen kann. Dies wird durch den Ansatz der Serious Games versucht, indem das Lernen mit dem Spielen kombinieren kombiniert wird. Dabei versuche ich das Konzept der Partnerzahlen in dieses Spiel einfließen zu lassen, wodurch bereits gute Lernerfolge in anderen Spielen erzielt wurden \cite{Jung2016}. Das Spiel umfasst zwei Versionen, die bis auf die Perspektive identisch sind. Ziel dieser Arbeit ist es, herauszufinden, ob dieses Mathespiel Kindern im Grundschulalter Spaß bereitet und welche Perspektive auf das Spielgeschehen sie besser finden.

Partnerzahlen, oder auch \qq{verliebte Zahlen} genannt, sind in der Mathematik ein beliebtes Mittel um die Addition zu lehren. Dabei wird dies oft im 10-er Zahlenraum für Natürliche Zahlen angewendet, um zwei Zahlen zu finden, die zusammen addiert 10 ergeben ( zum Beispiel 4 + 6 = 10, womit die Partnerzahl von 4 dem entsprechend 6 wäre um auf die Zahl 10 zu kommen). In diesem Spiel wird dieser Bereich, der gesuchten Zahl, kontinuirlich ausgeweitet. Anfangs wird der Zahlenraum von 3 bis 20 verwendet, bis hin aus dem Zahlenraum von 30 bis 100. Die untere Schranke des kleinsten Zahlenraums ist damit begründet, da hier nur wenig Variationen bestehen um diese Zahlen zu erreichen. Durch den Anstieg des Zahlenraums erhoffe ich mir ein anspruchsvolleres und erweiterbares Mathespiel entwickelt zu haben. Außerdem erhoffe ich mir einen interessanten Aufgabenbereich für das Spiel gewählt zu haben, der den Kindern es leichter macht die Addition zu verstehen und Spaß an dem Spiel zu haben.

Für das Spiel verwende ich den Ansatz der Serious Games, aber was sind Serious Games eigentlich? Für diese Art von Spiel gibt es zur heutigen Zeit noch weitere Bezeichnungen, wie 'Educational Game' , 'Game-Based-Learning' oder 'Edutainment'. Dabei ist es schwierig zu sagen, ob alle Begriffe die gleichen Spiele kategorisieren. Eine mögliche Kategorisierung orientiert sich am Verhältnis von didaktischen zu spielerischen Elementen \cite{Wechselberger2009}. Dabei sind Serious Games stark didaktisch ausgeprägt, während sich Edutainment stärker an unterhaltenden, spielerischen Elementen orientiert. Es gibt aber auch Definitionen, die allen Begriffen die gleiche Bedeutung zuordnen. Hawlitschek\cite{hawlitschek2013} (p.23) definiert digitale Lernspiele wie folgt:
\begin{quote}
\textit{... digitale Lernspiele sind Computerspiele,}
\begin{itemize}
\item \textit{die explizit und systematisch in Hinblick auf ein bestimmtes Lernziel und für den Einsatz in einem pädagogischen Kontext konzipiert wurden.}
\item \textit{die ein positives Spielerlebnis beim Spieler auslösen.}
\item \textit{deren Effektivität bei der Vermittlung der Lerninhalte nachgewiesen werden konnte.}
\end{itemize}
\end{quote}
Mit dieser Arbeit möchte ich ein solches digitales Lernspiel nach dieser Definition implementieren und vor allem den zweiten Punkt, des positiven Spielerlebnisses, untersuchen.\\
\\
In den Folgenden Kapiteln möchte ich aktuelle Erkenntnisse der Forschung präsentieren, sowie Aufbau und Evaluation des entwickelten Spieles.
% Main chapter 2:
% Grundlagen
\chapter{Grundlagen}
\label{chap:Grundlagen}
Das Kapitel \ref{chap:Grundlagen} behandelt die Grundlagen der vorliegenden Arbeit. Dazu zählen neben den Graphenbasierten Entwurfsprachen (Abschnitt \ref{sec:GBES}) die Finite-State-Machines (Abschnitt \ref{sec:FSM}) und das Requirements Engineering (Abschnitt \ref{sec:RE}).
	
\section{Graphenbasierte Entwurfssprachen}
\label{sec:GBES}
Graphenbasierte Entwurfssprachen ermöglichen es, "`\textit{das gesamte Wissen über den Entwurf eines Systems in einer digitalen, \mbox{konsistenten} und ausführbaren Graphenrepräsentation abzubilden}"' \cite{IIL17}. Grundlegend für die Modellierung in graphenbasierten Entwurfssprachen ist die Unified Modeling Language (UML). Ein kurzer Einblick erfolgt in Unterabschnitt \ref{subsec:UML}. Die UML bietet die Basis für die Systemmodellierung im Design Cockpit 43\textsuperscript{\textregistered} (Unterabschnitt \ref{subsec:DC43}). Einen Überblick über den Entwurf in graphenbasierten Entwurfssprachen wird Unterabschnitt \ref{subsec:GBE} dargelegt.

\subsection{Unified Modeling Language}
\label{subsec:UML}
\begin{quote}
	"`\textit{Seit Jahrhunderten stimmen Bauherren und Architekten Gebäudeanforderungen mithilfe
		von Grundrissen, Schnitten und Ansichten ab. Wie Bauzeichnungen für die Gebäudeplanung
		bilden UML-Diagramme im Kontext der Softwareentwicklung mittlerweile das zentrale
		Entwurfs- und Kommunikationsmittel.}"'\cite{RBF16}
\end{quote}
 Die Unified Modeling Language ist eine international genormte graphische Modellierungssprache zur  Entwicklung und Beschreibung von Systemen. Elementarer Bestandteil dieser sind somit die UML-Diagramme. Grundsätzlich lassen sich die 14 Diagramme in zwei Kategorien einteilen:\\
 
 \textbf{Strukturdiagram}: Dieses stellt die statische Struktur des Systems dar. Die Beschreibung erfolgt mithilfe von Klassen, Schnittstellen, Beziehungen, Konnektoren, Attributen und Operationen. Die wichtigste Ausprägung eines Strukturdiagramms, auch in Bedeutung auf der weiteren Verwendung im Design Cockpit 43\textsuperscript{\textregistered}, stellt das Klassendiagramm dar. Ein Beispiel kann Abbildung \ref{img:Ex_Class_Diagram} entnommen werden.
 \begin{figure}
 	\centering
 	\includegraphics[width=0.8\textwidth]{images/Ex_Class_Diagram.pdf}
 	\caption{Beispiel eines Klassendiagramms}
 	\label{img:Ex_Class_Diagram}
 \end{figure}\\
Das Diagramm besteht aus vier Klassen: Raum, Hörsaal, Vorlesung und Student. Bei der Klasse Raum handelt es sich um eine abstrakte Klasse, dies ist durch die kursive Schrift kenntlich gemacht. Abstrakte Klassen können nicht instanziiert werden d.h. es ist nicht möglich konkrete Objekte dieser Klasse zu erstellen. Die Klasse Hörsaal erbt von der Klasse Raum. Daraus folgt, dass der Hörsaal alle Attribute und Methoden (implementierte Operationen) übernimmt. Unterklassen können, bei Bedarf, die geerbten Attribute und Methoden überschreiben oder neue Attribute und Methoden hinzufügen. Bei der Klasse Hörsaal handelt es sich nicht um eine abstrakte Klasse, was zur Folge hat, dass eine Instanz (konkretes Objekt) dieser Klasse erstellt werden kann. Die Beziehung zwischen dem Hörsaal und der Vorlesung ist ein unidirektionale Assoziation mit der Multiplizität [0...*]. Dies bedeutet, dass ein Hörsaal keine bis beliebig viele Vorlesungen hat. Konkrete Beziehungen, soll heißen Instanzen einer Assoziation, werden Links genannt. Die Beziehung zwischen der Vorlesung und dem Student wird als bidirektionale Assoziation bezeichnet. Dies bedeutet, dass sowohl eine Navigierbarkeit von dem Studenten zur Vorlesung als auch von der Vorlesung zum Studenten besteht.  Die Beschreibung der Attribute, Methoden und Parameter erfolgt nach folgendem Schema:\\
\begin{figure} [H]
Attribute:\\
$[Sichtbarkeit] [/] name [:Typ] [Multiplizitaet] [=Vorgabewert][\{eigenschaftswert*\}]$\\
Methoden:\\
$[Sichtbarkeit] name [(\{Parameter\})][:Rueckgabetyp][\{eigenschaftswert*\}]$\\
Parameter:\\
$[Uebergaberichtung] name : Typ [Multiplizitaet] [=Vorgabewert][\{eigenschaftswer*\}]$\\
\end{figure}
Die Definition der Sichtbarkeiten sind Tabelle \ref{tab:VisUML} zu entnehmen.
\begin{table}[H]	
\begin{tabularx}{\textwidth}{p{0.08\textwidth}|p{0.15\textwidth}|p{0.8\textwidth}}
	Symbol & Bezeichnung & Eigenschaften \\ \hline
	$+$ & public & sichtbar für alle \\ \hline
	$\#$ & protected & sichtbar für alle innerhalb des gleichen Pakets;\newline sichtbar nur für Unterklassen außerhalb des Pakets \\ \hline
	$-$ & private & sichtbar nur innerhalb der Klasse \\ \hline
	$\tilde{}$ & package & sichtbar für alle innerhalb des gleichen Pakets	
\end{tabularx}
\caption{Sichtbarkeiten in der UML}
\label{tab:VisUML}
\end{table} 

\textbf{Verhaltensdiagram}: Dieses stellt das dynamische Verhalten eines Systems dar. Besondere Aufmerksamkeit soll hier, erneut in Bezug auf die weitere Verwendung im Design Cockpit 43\textsuperscript{\textregistered}, dem Aktivitätsdiagramm gewidmet werden. Zu den wichtigsten Elementen dieses gehören u.a. Start-, Endknoten, Aktionen und Entscheidungsknoten. Beispielhaft ist ein derartiges Diagramm in Abbildung \ref{img:Ex_Activity_Diagram} illustriert. 
 \begin{figure}
	\centering
	\includegraphics[width=0.8\textwidth]{images/Ex_Activity_Diagram.png}
	\caption{Beispiel eines Aktivitätsdiagramm \cite{RBF16}} 
	\label{img:Ex_Activity_Diagram}
\end{figure}\\
Das Ativitätsdiagramm beginnt mit dem Startknoten, der durch einen Vollkreis dargestellt ist. Von dort aus führt der Pfeil zu der ersten Aktion "`anfordern". Die graue Schrift an den Pfeilen beschreiben den Zustand in dem sich das System an dieser Stelle befindet. Der Entscheidungsknoten wird durch eine Raute symbolisiert. Die Bedingungen an ausgehenden Pfaden aus dem Entscheidungsknoten werden durch eckige Klammern angezeigt. Der Endknoten zeichnet sich im Gegensatz zu dem Startknoten durch eine zusätzlichen Hohlkreis um den Vollkreis aus.\\
Das Zustandsdiagramm, ein weiteres Verhaltensdiagramm, lässt sich aus dem Aktivitätsdiagramm ableiten, indem man die Pfeile zu Rechtecken macht und die Rechtecke zu Pfeilen. Dies ist im gegensätzlichen Fokus beider Diagramme begründet. Während das Aktivitätsdiagramm den Fokus auf die Aktionen legt, liegt jener des Zustandsdiagramms auf den Zuständen.\\

Für eine ausführlichere Beschreibung der Unified Modeling Language, ihre Diagramme und Begriffe sei beispielsweise auf "`Einführung in die UML"' von Randen, Bercker und Fieml \cite{RBF16} verwiesen.

\subsection{Design Cockpit 43\textsuperscript{\textregistered}}
\label{subsec:DC43}
Das Design Cockpit 43\textsuperscript{\textregistered} (DC43) liefert "`\textit{eine Softwareumgebung zur Erstellung und Ausführung von graphenbasierten Entwurfssprachen auf Basis der [im vorangegangen Abschnitt behandelten] Unified Modeling Language (UML)}"' \cite{IIL17}. Grundlage für das DC43 ist das Eclipse Framework, welcher mittels Erweiterungen eine flexible Anpassung an das jeweilige Problem ermöglicht. Zusätzlich bietet das Design Cockpit 43\textsuperscript{\textregistered} unterschiedlichste Schnittstellen zur "`Übersetzung"' der Unified Modeling Language in die sogenannten Domain Specific Languages der Zielsysteme,  wie zum Beispiel CAD, MKS und FEM. Die ermöglicht eine Analyse in den jeweiligen Fachdomänen mit dem Vorteil eines zentralen Datenmodells. Die wichtigsten Schlüsselkomponenten des DC43 sind:\\

\textbf{Klassendiagramm:} Das ,im vorherigen Abschnitt bereits angesprochene, Klassendiagramm enthält die verschiedenen Klassen (Vokabeln) mit ihren Attributen und stellt die Beziehungen der Klassen zueinander dar.\\

\textbf{Aktivitätsdiagramm:} Das, ebenfalls im vorherigen Abschnitt bereits besprochene, Aktivitätsdiagramm beinhaltet eine Abfolge von Aktionen (Rules) zur Transformation des Entwurfsgraphen. Dazu stehen vor allem sogenannte "`Java-Rules"' und die graphischen Regeln mit ihrer charakteristischen Linker- (if) und Rechter-Hand-Seite (then) zur Verfügung. Bei diesen erfolgt eine Modifaktion des Entwurfsgraphen gemäß der Rechten-Hand-Seite, insofern die Linke-Hand-Seite erfüllt ist. Ein Beispielhaft ist eine derartige Regel in Abbildung \ref{img:Graphical_Rule} abgebildet. In dieser wird, insofern eine Instanz der Gastturbinen-Klasse gefunden wird, Instanzen der Verdichter-, Brennkammer und Turbinen-Klasse erzeugt und in Beziehung zum Gastturbinen-Objekt gesetzt.
 \begin{figure}
	\centering
	\includegraphics[width=0.8\textwidth]{images/Graphical_Rule.pdf}
	\caption{Beispiel einer graphischen Regel innerhalb des Aktivitätsdiagramm im Design Cockpit 43 \textsuperscript{\textregistered}} \label{img:Graphical_Rule}
\end{figure}\\

\textbf{Lösungspfadgenerator:} Der Lösungspfadgenerator fungiert als Solver für die im Klassendiagramm als Gleichungen aufgeprägten Randbedingungen und im Aktivitätsdiagramm gesetzten Werten.\\

\textbf{Entwurfsgraph:} Der Entwurfsgraph repräsentiert die Topologie des Entwurfs mit mit seiner Geometrie etc. in einer abstrakten Graphendarstellung.\\

Auf das Zusammenspiel von Vokabeln und Regeln im Laufe des Entwurfsprozesses wird im folgenden Unterabschnitt noch einmal genauer eingegangen.

\subsection{Graphenbasierter Entwurf}
\label{subsec:GBE}
Im konventionellen Produktentwurf ergibt sich oftmals die Konfrontation mit der Situation, dass bereits zu einem frühen Zeitpunkt Entscheidungen gefällt werden müssen, welche einen hohen Anteil, der durch die Produktentwicklung, anfallenden Kosten bestimmen, ohne die Folgen richtig abschätzen zu können. Der graphenbasierte Entwurf schafft hier Abhilfe. Ein großer Vorteil von diesem ist, neben dem zentralen Datenmodell, die systematische Variation von Design-Varianten, welche eine substantielle methodische Hilfestellung bei der eingangs erwähnten Problematik bietet. Ein Überblick über die Verarbeitungskette im graphenbasierten Entwurf kann mittels Abbildung \ref{img:Graph_Based_Design} gewonnen werden.
\begin{figure}
	\centering
	\includegraphics[width=0.8\textwidth]{images/GBD_process_chain.pdf}
	\caption{Graphenbasierter Entwurfsprozess, nach \cite{Rud14}} !!!noch abändern, wenn engültiges Ablaufdiagramm bekannt!!!\label{img:Graph_Based_Design}
\end{figure}\\
Das Kernstück des graphenbasierten Entwurfs bilden dabei die Elemente im grauen Kasten. Im ersten Schritt bilden, die aus dem Klassendiagramm gewonnenen, Klassen mit ihren Beziehungen zueinander die Vokabeln. Zusammen mit den, aus dem Aktivitätsdiagramm extrahierten, Regeln, welche mittels Transformationen die Vokabeln in Instanzen ausprägen und deren Verlinkungen modifzieren, bilden diese das Produktionssystem. Ausgehend von dem Produktionssystems können mithilfe eines Compilers (DC43), unter Beachtung sämtlicher im Lösungspfadgenerator verarbeiteten Randbedingungen beziehungsweise Gleichungen, verschiedene Entwurfsvariationen erzeugt werden. Schließlich liefert der Entwurfsgraph eine abstrakte Repräsentation dises Entwurfs. Durch die, in Unterabschnitt \ref{subsec:DC43} erwähnten Schnittstellen, lässt sich dieser abstrakte Entwurfsgraph für weitere Untersuchungen in die jeweiligen fachspezifischen Analyseprogramme exportieren. Außerdem ist es natürlich auch noch möglich durch entsprechende Schnittstellen Daten aus anderen Programmen, wie zum Beispiel über das Requirements Interchange Format (ReqIF), zu importieren und im graphenbasierten Entwurf zu verarbeiten.\\
Variationen in Vokabeln und Regeln ermöglichen es Entwurfsalternativen, in einem Zeitaufwand der nur durch die Rechenzeit gegeben ist, zu erzeugen.
Eine Rückführung der Ergebnisse von den Analysetools mittels Bewertungsverfahren ermöglicht eine automatische Variation des Produktionssystems.


\section{Finite State Machine}
\label{sec:FSM}
Der Zustandsautomat (englisch: finite state machine, FSM) versteht sich als ein abgeschlossenes System welches durch eine Menge von Zuständen und deren Übergänge definiert wird. Die Impulse für einen Übergang können von dabei sowohl extern angeregt werden als auch spontan auftreten. In der vorliegenden Arbeit werden grundsätzlich vier verschiedene Übergänge unterschieden: 
\begin{itemize}
	\item[] \textbf{Spontane Transitionen}: Bei spontanen Transitionen sind keine Bedingungen an die Übergänge geknüpft. Diese können quasi instantan auftreten.
	\item[] \textbf{Eingangs-Angeregte Transitionen}:  Die Überführung auf den nächsten Zustand mittels einer Eingangs-Angeregten Transitionen kann nur eintreten insofern bestimmte Bedingungen bezüglich des Eingangs erfüllt sind.
	\item[] \textbf{Counter-Angeregte Transitionen}: Counter-Angeregte Transitionen beinhalten eine Bedingung bezüglich eines bestimmten Zählwerts um auf den nächsten Zustand zu führen.  
	\item[] \textbf{Timer-Angeregte Transitionen}: Bei Timer-angeregten Transitionen ist ein Timeout-Signal eines (spezifischen) Timers nötig, damit diese Transition begangen werden kann.
\end{itemize}
Möglich sind natürlich auch Kombinationen dieser.
%Eine Erweiterung des funktionellen Umfangs wird durch hinzufügen von einem oder mehreren externen Speichern erhalten. Diese ermöglichen es Informationen die nicht mit dem aktuellen Status in Verbindung stehen zu speichern \cite{Sta17}. 
In Abschnitt \ref{sec:FSM} werden die theoretischen Grundlagen zu den Finite State Machines  betrachtet. Dazu folgt zuerst eine kurze Definition von FSMs in Abschnitt \ref{(subsec:DefFSM)}. Anschließend wird eine Klassifikation der verschiedenen Arten von FSMs (Abschnitt \ref{(subsec:ClassFSM)}) dargelegt. Zusätzlich wird ein Algorithmus zur Zustandsminimierung in Abschnitt \ref{subsec:StateMinimization} angesprochen.

\subsection{Definition}
\label{(subsec:DefFSM)}
%Ein Zustandsautomat stellt das mathematische Modell einer sequentiellen Schaltung dar \cite{CK13}. [Sequentielle Schaltung erklären]
Cerny \cite{Cer80} schlägt charakteristische Funktionen vor um Mengen und ihr Beziehungen zueinander darzustellen. Bei einer gegebenen Relation $R \subseteq X \times Y$, wobei $X$ und $Y$ eine endliche Menge sind, ist die charakteristische Funktion von $R \chi_R : X \times Y \rightarrow B$ für jedes Paar $(x,y) \in X \times Y$ definiert wie folgt:
\begin{equation}
\chi_R(x,y)=\begin{cases}
0 & \text{falls x,y nicht in Beziehung R zueinander stehen}\\
1& \text{falls x,y in Bezihung R zueinander stehen} 
\end{cases}\ \text{.}
\label{eq:charFunkt}
\end{equation}
Die vorrangengangene Gleichung lässt sich auch auf n-dimensionale Relationen ausbauen.
Kam, Villa, Brayton und Sangiovanni-Vincentelli \cite{KVBS97} definieren einen Zustandsautomaten folgendermaßen :
Eine FSM wird durch ein 5-fach-Tupel 
\begin{equation}
M=\{S,I,O,T,R\}
\label{eq:DefFSM}
\end{equation} definiert. $S$ beschreibt dabei den finiten Zustandsraum, $I$ den finiten Eingangsraum und $O$ den finiten Ausgangsraum. $T$ bezeichnen die Zustands-Transitions-Beziehungen und sind definiert durch eine charakteristische Funktion $T$: 
\begin{equation}
T=I \times S \times S \times O \xrightarrow {} B
\label{eq:DefFSMT}
\end{equation} Bei einem Eingang $i$ kann die FSM bei einem momentanten Zustand $p$ nur in den nächsten Zustand $n$ mit den Ausgang $o$ übergehen insofern die charakteristische Transitionsfunktion (\ref{eq:DefFSMT}) $T(i,p,n,o) = 1$ ist, wobei mögliche ist, dass mehrere Transitionen für den momentanten Zustand  $n$ und den Eingang $i$ exisiteren. Die Zustands-Transitions-Beziehungen können als vollständig angesehen werden insofern die Bedingung 
\begin{equation}
\forall i \in I \forall p \in S \exists n \in S \exists o \in O \text{ sodass } T(i,p,n,o) = 1
\label{eq:CondFSMT}
\end{equation}
erfüllt ist.
\begin{equation}
R \subseteq S
\label{eq:DefFSMR}
\end{equation} bilden eine Menge von Reset-States, sprich Ausgangszuständen in welche die FSM (z.B. nach Beendigung dieser) zurückgesetzt werden kann.
Der Zustands-Transitions-Graph eines gegebenen Zustandsautomaten $M = \{S,I,O,T,R\}$ ist ein gekennzeichneter Graph
\begin{equation}
STG(M) = (V,E)
\label{eq:GraphFSMT}
\end{equation}
bei dem jeder Zustand $s \in S$ durch einen Knoten $V$ und jede Transition durch eine Kante $E$ gegeben ist.
% Zustandsautomat Abbildungen von Mealy und Moore-Typ Automaten später bei Erläuterung der Typen. Evtl Auch Zustandsüberganzsbild
\subsection{Klassifikation}
\label{(subsec:ClassFSM)}
Die Finite State Machines lassen sich je nach Ausführung beziehungsweise der Zustands-Transitions-Relationen in verschieden Klassen einordnen. Bei der in Abschnitt \ref{(subsec:DefFSM)} definiertem und allgemeinstem Zustandsautomaten handelt es sich um eine nicht-deterministische FSM, da bei dieser keinerlei Restriktionen an die Transitionen geknüpft sind. Nicht-deterministische Finite State Machines (NDFSM) zeichnen sich durch einen nicht eindeutigen nächsten Zustand $p$ bezüglich eines Paars aus Eingangssinal $i$ und aktuellen Zustand $n$ aus. Die Transition in den nächsten Zustand ist somit prinzipiell willkürlich und nicht deterministisch festgelegt. Wie bereits angeklungen beruht somit die Einordnung in die verschiedenen Klassen auf Restriktionen and die Transitionen. Die allgemeine Struktur von FSM ist in Abbildung \ref{img:StructFSM} dargestellt. $\delta / \Delta$ und $\lambda / \Lambda$ stehen hierbei für die Funktion/Relation des nächsten Zustands beziehungsweise des Ausgangs. Diese spielen bei der Klassifizerung eine wichtige Rolle und erfahren deshalb eine genauere Betrachtung in diesem Abschnitt. Eine Übersicht der Klassifikation von Zustandsautomaten ist Abbildung \ref{img:ClassFSM} zu entnehmen. Die im folgenden vorgestellten Defintionen zur Klassifikation von Zustandautomaten lassen sich ebenfalls auf Kam, Villa, Brayton und Sangiovanni-Vincentelli \cite{KVBS97} zurückführen.
\begin{figure}
	\centering
	\includegraphics[width=0.66\textwidth]{images/StructureStateMachine.pdf}
	\caption{Struktur einer Finite State Machine}
	\label{img:StructFSM}
\end{figure}

\subsubsection{Nicht-deterministische FSM}
\label{subsubsec:ClassNDFSM}
Eine nicht pseudo nicht-deterministische Finite State Machine (PNDFSM) ist definiert durch ein 6-fach-Tupel 
\begin{equation}
M = \{S,I,O,\delta,\Lambda,R\} \text{ .}
\label{eq:DefPNDFSM}
\end{equation}
Zustätzlich zu den aus Abschnitt \ref{(subsec:DefFSM)} bekannten Vektoren $S,I,O,R$ treten $\delta$ und $\Lambda$ auf. $\delta$ stellt mit
\begin{equation}
\delta : I \times S \times O \xrightarrow{} S
\label{eq:DefPNDFSMdelta}
\end{equation}
die Funktion des nächsten Zustands dar, wobei jedes 3-fach-Tupel aus Eingang $i$, aktueller Zustand $p$ und Ausgang $o$ auf einen eindeutigen nächsten Zustand $n$ abbildet. $\Lambda$ ist die charakteristische Ausgangsfunktion mit
\begin{equation}
\Lambda: I \times S \times O \xrightarrow{} B \text{ ,}
\label{eq:DefPNDFSMLambda}
\end{equation}
wobei jede Kombination von Eingang $i$ und aktuellen Zustand $p$ sich auf einen oder mehrere Ausgangswerte $o$ beziehen. Für die Zustands-Transitions-Relationen lässt sich somit folgende restriktierende Bedingung ableiten:
\begin{equation}
\forall i,p,o \exists! n \text{ sodass } T(i,p,n,o)=1
\label{eq:CondPNDFSMT} 
\end{equation}
Weitere nicht-deterministische Finite State Machines, wie die Mealy NDFSM, die Moore NDFSM und die Incomplete Specified FSM lassen sich von vornherein für die vorliegende Arbeit ausschließen. Dies ist darin begründet, dass es für ein automotives System höchst unpraktikabel erscheint, insofern die "`Soll-Reaktion"'  und somit der nächste Zustand nicht eindeutig festgelegt ist.
Für Definitionen zu diesen drei Klassen von Finite State Machines sei auf Abschnitt \ref{sec:NDFSM} im Appendix verwiesen.

\subsubsection{Deterministische FSM}
\label{subsubsec:ClassDFSM}
Den nicht-deterministischen FSM diametral gegenüber steht die Completely Specified FSM (CSFSM) oder auch einfach nur deterministische Finite State Machine (DFSM). Beschrieben wird diese durch folgendes 6-fach-Tupel:
\begin{equation}
M = \{S,I,O,\delta,\lambda,r\} \text{ .}
\label{eq:DefDFSM}
\end{equation}
$S$ steht hierbei wieder für den finiten Zustandsraum, $I$ für den Raum der Eingangssignale und $O$ für den Raum der Ausganssignale. $r \in S$ beschreibt im Gegensatz zu R nicht eine Menge von möglichen sondern einen eindeutigen Reset-State. Die Funktion für den nächsten Zustand ist definiert durch
\begin{equation}
\delta : I \times S \xrightarrow{} S \text{ ,}
\label{eq:DefDFSMdelta}
\end{equation}
wobei $n \in S$ der nächste Zustand bezüglich des aktuellen Zustands $p \in S$ und des Eingangs $i \in I$ ist, insofern gilt $n = \delta(i,p)$. Die Definition der Funktion für den Ausgang ist
\begin{equation}
\lambda : I \times S \xrightarrow{} O \text{ .}
\label{eq:DefDFSMlambda}
\end{equation}
Hier steht $o \in O$ für den Ausgang des aktuellen Zustands $p \in S$ bezüglich des Eingangs $i \in I$, falls gilt $o = \delta(i,p)$. Die Beziehung für die Zustands-Transitions-Relation ergibt sich zu
\begin{equation}
\forall (i,p) \in I \times S \exists! n \in S \text{ sodass } T(i,p,n,o)=1 \text{ .}
\end{equation}
Durch Spezialisierung der Funktion für den Ausgang lässt sich die Moore DFSM ableiten. Bei dieser gilt
\begin{equation}
\lambda : S \xrightarrow{} O \text{ .}
\label{eq:DefMooreDFSMlambda}
\end{equation}
$o \in O$ bezeichnet dabei für $o=\lambda(p)$ das Ausgangssignal des aktuellen Zustands $p \in S$.
\begin{figure}
	\centering
	\includegraphics[width=0.66\textwidth]{images/ClassificationFSMOverview.pdf}
	\caption{Übersicht der Klassifikation von Finite State Machines}
	\label{img:ClassFSM}
\end{figure}

\subsubsection{Definition von Zustand, Eingang, Ausgang und Transitionsfunktion und Ausgangsfunktion (MooreDFSM)}
\label{subsubsec:DefSIOT}
Eine  deterministische Moore Finite State Machine resultiert aus der Annahme eindeutiger Zustände, die hinreichend den Ausgang festlegen. Für diese lassen sich folgende Definitionen %nach Kam, Villa, Brayton und Sangiovanni-Vincentelli \cite{KVBS97} 
für den Zustand, den Eingang, den Ausgang, die Transitionsfunktion und die Ausgangsfunktion geben. Anhand der Abbildung \ref{img:Ex_FS_Diagram} werden die gegebenen Definitionen mit einem Beispiel erweitert. Vorweg sei bemerkt, dass es sich bei dem Eingangsraum, dem Zustandsraum und den Ausgangraum der vorrangegangenen Definitionen zur Klassifikation der Zustandsautomaten um generische Räume handeln kann. Somit liegt beispielweise keine Beschränkung des Eingangs oder des Ausgangs auf ein binäres Signal vor, sondern diese können symbolische Werte annehmen. \\

\textbf{Zustand (State):} \textit{Der Zustand beschreibt eindeutig die aktuelle Situation in der sich das System befindet. Als Zustandsmenge wird die endliche Gesamtheit aller Zustände die ein System einnehmen kann bezeichnet. Untermengen dieser sind der Initialzustand und die Menge akzeptierter Endzustände.}\\
Beispiele sind mit Abbildung \ref{img:Ex_FS_Diagram} durch die Zustände "`Einkaufswagen"', "`zu bezahlen"', "`bezahlt"' und "`versendet"' gegeben. "`Einkaufswagen"', dargestellt durch einen Vollkreis, ist dabei der Initialzustand und "`versendet"', zusätzlich mit ein umliegenden Ring illustriert, der akzeptierten Endzustand.\\

\textbf{Eingang (Input):} \textit{Der Eingang besteht aus einer endlichen Zeichenfolge erlaubter Eingabesymbole, dem sogenannten Eingabealphabet. Der Eingang koppelt über die Transitionsfunktion direkt in das System ein.}\\
Anhand des Beispiels in Abbildung \ref{img:Ex_FS_Diagram} könnte unter anderem das Drücken eines Bestellknopfs einen möglichen Eingang in die Finite State Machine, welche das System eines Bestellvorgangs abbildet, darstellen.\\


\textbf{Ausgang (Output):} \textit{ Der Ausgang besteht analog zum Eingang ebenfalls aus einer endlichen Zeichenfolge erlaubter Ausgabesymbole. Der Ausgang koppelt prinzipiell nicht in das System zurück und hat somit grundsätzlich nur auf die Ausgabe einen Einfluss.}\\
Ein möglicher Ausgang in Bezug auf Abbildung \ref{img:Ex_FS_Diagram} wäre somit die Anzeige einer Versandbestätiung, nachdem der akzeptierte Endzustand "`versendet"' erreicht worden ist.\\

\textbf{Transitionsfunktion (Transition Logic):} \textit{Die Transitionsfunktion regelt den Übergang von dem aktuellen Zustand zum nächsten Zustand. Ein Zustand kann über endliche viele Transitionen mit sich selbst oder anderen Zuständen in Verbindung gesetzt werden. Die Transitionsfunktion bildet bezüglich des Eingangs und dem aktuellen Zustand auf die entsprechende Transition und somit den nächsten Zustand ab.}\\
In Abbildung \ref{img:Ex_FS_Diagram} bildet die Transitionsfunktion beispielweise anhand des Eingangs der Bezahlung auf die Transition beziehungsweise Aktion bezahlen ab, welche das System von dem Zustand "`zu bezahlen"' in den Zustabd "`bezahlt"' überführt.\\

\textbf{Ausgangsfunktion (Output Logic):} \textit{Die Ausgangsfunktion bildet auf einen entsprechenden Ausgang beziehungsweise Ausgabesymbol ab. Bei Moore FSM besteht dabei nur eine Abhängigkeit auf den aktuellen Zustand.}\\
Beispielhaft bildet die Ausgangsfunktion in Abbildung \ref{img:Ex_FS_Diagram} anhand des aktuellen Status "`versendet"' auf den Ausgang der Anzeige einer Versandbestätigung ab.\\

Die Modellierung eines Systems ist somit durch eine Abfolge von Zuständen, die aufgrund des momentan anliegenden Eingangs, in definierte andere Zustände wechseln und einen entsprechendes Ausgang produzieren möglich.

\begin{figure}
	\centering
	\includegraphics[width=0.9\textwidth]{images/Ex_FS_Diagram.png}
	\caption{Zustandsdiagramm einer Bestellung \cite{RBF16}}
	\label{img:Ex_FS_Diagram}
\end{figure}

\subsection{Zustandsminimierung bei FSM}
\label{subsec:StateMinimization}
Zur Zustandsminimierung bei Finite State Machines existiert eine Vielzahl von Methoden. Czerwinski und Kania schlagen den in diesem Abschnitt vorgestellte schnelle und einfache Vorgehensweise zur symbolischen Zustandsminimierung vor \cite{CK13}.

\subsubsection{Algorithmus zur Zustandsminimierung}
\label{subsubsec:AlgStateMinimization}
Der erste Schritt zur Zustandsminimierung besteht hierbei die Finite State Machine in Form einer State Transition Table (Zustandsübergangstabelle, STT) darzustellen. Die STT (Abbildung \ref{img:FSM2STT}) enthält in der ersten Spalte das Eingangssginal, in der zweiten den aktuellen, in der dritten den nächsten Zustand und in der vierten das Ausgangssignal.
\begin{figure}
	\centering
	\includegraphics[width=0.66\textwidth]{images/STT.png}
	\caption{Zustandsübergangsgraph und zugehörige STT \cite{CK13}}
	\label{img:FSM2STT}
\end{figure}
Im nächsten Schritt wird die STT in partitionierte Tabellen überführt. Dabei werden die Transitionen (Zeilen in der STT) nach dem Ausgangssignal sortiert (Abbildung \ref{img:STT2PT}).
\begin{figure}
	\centering
	\includegraphics[width=0.66\textwidth]{images/PT.png}
	\caption{STT und zugehörige partitionierte Tabellen \cite{CK13}}
	\label{img:STT2PT}
\end{figure}
Es sei hier angemerkt, dass Czerwinski und Kania den Raum der Eingangssignale mit X bezeichnen (und nicht wie in vorangegangenen Defintionen mit I). Als inkompatibel zur Zustandsminimierung können nun diejenigen Spalten angesehen werden, falls mindestens eine Reihe mit zwei verschiedenen Zuständen existiert. Aus diesen partitionierten Tabellen lässt sich nun der Inkompatibilitätsgraph (Abbildung \ref{img:PT2ICGraph}) ableiten.
\begin{figure}
	\centering
	\includegraphics[width=0.66\textwidth]{images/ICGraph.png}
	\caption{Partitionierte Tabellen und der daraus abgeleitete Inkompatibilitätsgraph. St3 und St8 sind inkompatibel, da diese zwei unterschiedliche nächste Zustände besitzen. \cite{CK13}}
	\label{img:PT2ICGraph}
\end{figure}
Der Inkompatibilitätsgraph ist dabei ein ungerichteter Graph. Die Knoten symbolisieren die verschiedenen Zustände, die Kanten die Inkompatibilität. Die Grundidee besteht nun darin zweit adjazenten (mit einer Kante verbundenen) Knoten unterschiedliche Farben zuzuweisen. Dabei wird dem Knoten mit der höchsten Anzahl von Kanten eine Farbe zugewiesen. Dem adjanzenten Knoten wird diese Farbe als "nicht verfügbar" zugeteilt. Im Anschluss kann diese Kante dann entfernt werden. Der vollständige Algorithmus kann wir folgt formuliert werden:
\begin{figure}[H]
\begin{enumerate}
	\item Aufteilung der STT in partitionierte Tabellen
	\item Aufstellen des Inkompatibilitätsgraph
	\item Suchen des Knotens $\nu_i$ im Imkompatibilitätsgraph mit:
	\begin{enumerate}
		\item der größten Anzahl von nicht verfügbaren Farben
		\item der größten Anzahl von Kanten
	\end{enumerate}
	\item Zuteilen einer verfügbaren Farbe $A_i$ zum Knoten $\nu_i$
	\item Zuteilen der nicht verfügbaren Farbe $a_i$ zu den adjazenten Knoten zu $\nu_i$
	\item Reduzieren (entfernen der Kanten zu den adjazenten Knoten zu $\nu_i$) des Inkompatibilitätsgraph
	\item Bei Vorhandensein von nicht gefärbten Knoten zurück zu Punkt 3
	\item Ende
\end{enumerate}
\end{figure}
Ein Beispiel der Anwendung dieses Algorithmus kann Abbildung \ref{img:ICGraph_Example} entnommen werden. 
\begin{figure}
	\centering
	\includegraphics{images/ICGraph_Example.png}
	\caption{Anwendungsbeispiel des vorgestellten Algorithmus zur Zustandsminimierung. Kompatible Knoten sind hier mit der gleichen Farbe gekennzeichnet. \cite{CK13}}
	\label{img:ICGraph_Example}
\end{figure}

\section{Requirements Engineering}
\label{sec:RE}
\begin{quote}
	"`\textit{Die Anforderungen an ein neues [...] Produkt zu ermitteln, zu spezifizieren, zu analysieren, zu validieren und daraus eine fachliche Lösung abzuleiten bzw. ein Produktmodell zu entwickeln gehört mit zu den anspruchsvollsten Aufgaben}."' \cite{Bal09}
\end{quote} 
Zusammengefasst wird dies unter dem Begriff des Requirements Engineering. Zuerst wird in Abschnitt \ref{subsec:ProDev} auf die Produktentwicklung und ihre Modelle beziehungsweise Methoden eingegangen. Anschließend folgt in Abschnitt \ref{subsec:ClassReq} eine Klassifikation der Requirements. 
%Abschließend widmet sich Abschnitt \ref{subsec:Patterns} der Sprachverarbeitung der textualisierten Requirements. 
\subsection{Produktentwicklung}
\label{subsec:ProDev}
Dieser Unterabschnitt gibt einen Einblick in den allgemeinen Ablauf der Produktentwicklung beziehungsweise der Entwicklung eines Systems. Bei realen technischen Systemen ist schnell eine hohe Komplexitätsstufe erreicht. Dies lieferte, im Rahmen des Systems Engineering,  den Anlass zur Entwicklung verschiedener Phasenmodelle, welche bei der Produktentwicklung eines hoch komplexen Systems als Hilfestellung dienen sollen.\\
Bei dem klassischen Phasenmodell (auch "`Wasserfall"'-Modell genannt; Abbildung \ref{img:Waterfal_Model}) handelt es sich um das einfachste solcher Modelle. 
\begin{figure}
	\centering
	\includegraphics[width=0.6\textwidth]{images/WaterfallModel_simple.pdf}
	\caption{Vereinfachte Darstellung des Wasserfallmodells}
	\label{img:Waterfal_Model}
\end{figure}
Bei diesem werden einfach die Stufen der verschiedenen Phasen hinabgestiegen bis man bei der letzten angelangt ist. Die Ergebnisse der vorherigen Stufe sind dabei bindende Randbedingungen für die folgende Stufe.\\ 
%In den ersten drei Phasen wird das Lastenheft mit den Anforderungen erstellt. Dabei werden die Spezifikationen sukzessiv von der Produkt-, über die System- bis zur Kompontenebene verfeinert. Die Entwicklungsphase beinhaltet den Übergang von der abstrakten Spezifikation zum wirklichen Bauteil. In den letzten drei Phasen folgt die Integration der Komponenten zum Gesamtprodukt. Dabei werden zuerst die einzelnen Komponenten für sich,
%anschließend die zum System kombinierten Komponenten und abschließen die zum Produkt zusammengefügten (Teil-)Systeme getestet.
In der ersten Phase der Analyse und Definition erfolgt die Erstellung des Lastenhefts. Daraufhin folgt die Entwurfsphase, in welcher der Entwurf der Systemarchitektur stattfindet. Die Entwicklungsphase beinhaltet den Übergang von der abstrakten Spezifikation zum wirklichen Bauteil. In der letzten Phase, Integration, Test und Abnnahme, werden die Bauteile beziehungsweise Komponenten zum Gesamtsystem kombiniert und die Überprüfung dieses durchgeführt.\\ 
Eine Weiterentwicklung des Wasserfallmodells, und auch Stand der Technik im Systems Engineering, ist das V-Model (Abbildung \ref{img:Mod_V-Model}).
Zusätzlich zum Wasserfallmodell schlüsselt dieses die temporale Abfolge der verschiedenen Spezifikations und Testphasen auf und bildet die Teststufe direkt gegenüber der entsprechenden Spezifikationsstufe ab. Dadurch entsteht das für die Namensgebung maßgebliche V.
\begin{figure}
	\centering
	\includegraphics[width=0.99\textwidth]{images/Mod_V-Model.pdf}
	\caption{Das allgemeine V-Modell, nach \cite{SL05}}
	\label{img:Mod_V-Model}
\end{figure}
Zusätzlich enthält dieses auf der Spezifikationsseite bis hin zur Entwicklung Verfikationsschritte. Auf der Integrationsseite werden zu den Tests auf den verschieden Stufen des weiteren auch noch Integrationstests durchgeführt, welche überprüfen ob die Schnittstellen zwischen den einzelnen Elementen die Spezifikation erfüllen. Eine Validierung bezüglich der Requirements findet mittels den Tests statt.
%Bei der Verifikation wird ausgehend von der Entwicklung sukzessiv zürück auf die Anforderungen gehend bewertet ob den Spezifikationen gemäß entwickelt wurde. Die Validierung kontrolliert mithilfe der Tests ob die Spezifikationen eingehalten worden sind (Validierung nach Benz Vorlesung einfügen). Auf der Integrationsseite werden zusätzlich zu den Tests auf den verschieden Stufen auch noch Integrationstests durchgeführt, welche überprüfen ob die Schnittstellen zwischen den einzelnen Elementen die Spezifikation erfüllen.\\

Durch die Abbildung \ref{img:Mod_V-Model} zum V-Model wird sofort die elementare Wichtigkeit von Requirements ersichtlich. Durch die Verifikations- und Validierungsschritte ist es möglich die Korrektheit und Vollständigkeit des Phasenergebnisses relativ zu seiner direkten Spezifikation (Phaseneingangsdokumenten) nachzuweisen (Verifikation) \cite{SL05} und ob ein (Teil-) Produkt eine festgelegte (spezifizierte) Aufgabe tatsächlich löst (Validation) \cite{SL05}. Die Überprüfung, ob die Anforderungen die Funktionen des Systems beziehungsweise die Funktionsweise des Gesamtsystems korrekt beschreiben stellt jedoch ein Problem dar. Beispielsweise tritt eine unvollständige Spezifikation des Systems, welche nicht alle gewünschten (Teil-) Funktionen des Systems ausreichend beschreibt, womöglich erst bei den Produkttests beziehungsweise der Abnahme zu Tage.  Des weiteren können auch Uneindeutigkeiten durch rendundante, oder sogar sich wiedersprechenden Requirements auftreten.

Es sei an dieser Stelle vorweggenommen, dass sich die vorliegende Arbeit hauptsächlich mit der linken Seite des V-Modell, also den Spezifikationen beschäftigt. Mit der rechten Seite des V-Modell, zumindest einem Teil davon, genau genommen der Optimierung von Testfällen durch Entfernen redundanter Testschritte und dem dadurch benötigtem Clustering und Neuordnen dieser haben sich bereits Walter et al. \cite{WSPR17} befasst.

%validation
%Ref: ISO 9000
%Confirmation by examination and through provision of objective evidence that the requirements for a specific
%intended use or application have been fulfilled.

%verification
%Ref: ISO 9000
%Confirmation by examination and through provision of objective evidence that specified requirements have been
%fulfilled.
 
\subsection{Klassifikation von Requirements}
\label{subsec:ClassReq}
Dieser Unterabschnitt legt eine Möglichkeit der Klassifzierung von Requirements dar.\\ Anforderungen werden traditionsweise in zwei Klassen unterteilt, funktionale Requirements und nicht-funktionale Requirements. \\
Es besteht grundsätzlich großes Einvernehmen über die Defintion der funktionalen Anforderungen. Eine übliche Definition wäre "`eine Funktion, welche ein System in der Lage sein muss auszuführen"' \cite{IEEE90}.\\
Ganz anders stellt sich die Lage bei nicht-funktionalen Anforderungen dar. Zu dem Thema der nicht-funktionalen Requirements gibt es zwar eine Vielzahl von Veröffentlichungen, aber im Allgemeinen besteht hier kein Konsensus über die Definition dieser \cite{Gli07}. Glinz ersetzt den Ausdruck der nicht-funktionalen Requirements indem er diesen weiter aufgliedert in Performance Requirements, Quality Requirements und Constraints. Eine Übersicht über die Klassifzierung von Requirements wird in Abbildung \ref{img:OV_Class_Req} gegeben.\\
\begin{figure}
	\centering
	\includegraphics[width=0.6\textwidth]{images/Overview_Class_Req.png}
	\caption{Überblick über die Klassifzierung von nicht-funktionalen Requirements nach Glinz \cite{Gli07}}
	\label{img:OV_Class_Req}
\end{figure} 
Im Gegensatz zu den funktionalen Andorderungen stellen die Performance und Quality Anforderungen keine Funktionen dar, sondern sind als Attribute des Systems anzusehen. Die Performance Requirements beinhalten beispielsweise vor allem zeitliche und räumliche Begrenzungen. Die Quality Requirements beschreiben unter anderem Attribute wie Brauchbarkeit und Sicherheit. Constraints stellen ihrem Namen entsprechend (zum Beispiel: pyhsikalische oder kulturelle) Randbedingung für das System auf.

\subsection{Grundprinzipien an Requirements}
\label{subsec:PrincReq}
Dieser Unterabschnitt stellt einige Grundprinzipien an Requirements vor, die eine sinnvolle und effektive Arbeit mit Anforderungen gewährleisten.
Zunächst soll vorangestellt werden, dass Anforderungen nicht das Verhalten, sondern Eigenschaften, Effekte und Auswirkungen beschreiben sollten. Dies wird mit der Möglichkeit verschiedener Verhalten, welche das gewünschte Resultat liefern, begründet, sowie auch verschiedene Programme das gewünschte Ergebnis berechnen können \cite{Jac17}.\\
Zusätzlich sollten folgende Grundprinzipien, welche von der IEEE Computer Society und dem IEEE-SA Standards Board gelistet \cite{II98} und von ISO/IEC/IEEE \cite{III11} definiert werden, auf Requirements angewendet werden können:
\begin{itemize}
	\item[] \textbf{Notwendig}: Das Requirement definiert eine essentiele Funktion, Charakteristik, Zwang und/oder Gütefaktor.
	\item[] \textbf{Implementationsfrei}: Das Requirement gibt keine spezifische Implementation vor oder stützt sich auf eine spezifische Implementation.
	\item[] \textbf{Eindeutig}: Das Requirement ist so formuliert, dass es nur auf eine bestimmte Weise interpretiert werden kann.
	\item[] \textbf{Konsistent}: Die Konsistenz steht mit keinen anderen Requirements im Konflikt.
	\item[] \textbf{Vollständig}: Das Requirement benötigt keine Erweiterungen, da es messbar ist und die Funktionen und Charakteristiken ausreichend beschrieben sind um die Ansprüche der Teilhaber zu erfüllen.
	\item[] \textbf{Singulär}: Die textuale Beschreibung des Requirements beinhaltet nur ein Requirement.
	\item[] \textbf{Realisierbar}: Es besteht die technische Möglichkeit das Requirement umzusetzen.
	\item[] \textbf{Rückverfolgbar}: Das Requirement ist sowohl auf die Dokumente, in welchen die Ansprüche der Teilhaber niedergeschrieben sind zurückführbar, als auch auf spezifischere Requirements nachverfolgbar.
	\item[] \textbf{Nachweisbar}: Es besteht die Möglichkeit zu überprüfen, ob das System das spezifizierte Requirement erfüllt. Die Nachweisbarkeit wird verbessert, falls das Requirement messbar ist.
\end{itemize}
Es sei an dieser Stelle vorweggenommen, dass mit der, in dieser Arbeit, entwickelten Methodik die Notwendigkeit, Konsistenz und Rückverfolgbarkeit systematisch überprüft werden können. Die Nachweisbarkeit ist mittels der Validierung über Tests gegeben und die Validation der Vollständigkeit obliegt dem Anwender, da dieser beurteilen muss ob das System den Ansprüchen entsprechend funktioniert. Den Erhalt der Singularität und der Eindeutigkeit wird im nächsten Unterabschnitt (\ref{sec:Patterns}) untersucht.

\section{Formalisierung von textuellen Requirements (Sprachverarbeitung)}
\label{sec:Patterns}
\begin{quote}
	"`\textit{Die Fähigkeit, mit natürlicher Sprache zu kommunizieren, ist eine der erstaunlichsten Eigenschaften des Menschen. [...] Der wissenschaftliche und technische Fortschritt der letzten Jahrzehnte hat dazu geführt, dass Maschinen immer besser darin werden, Teile der menschlichen Sprachkompetenz nachzuahmen. Die Maschinelle Sprachverarbeitung befasst sich also mit der so genannten Computerlinguistik und untersucht die Sprache aus einem besonderen Blickwinkel. Ihr geht es darum, die sprachlichen Gesetzmäßigkeiten explizit zu machen, um auf dieser Basis Rechnersysteme zu erstellen, die Sprache verstehen und produzieren können.}"' \cite{IMS17}
\end{quote}
Der vorliegende Abschnitt befasst sich mit Überführung von natürlicher Sprache in maschinelle Operationen. Dazu werden zuerst spezialisierte Sprachmuster (Unterabschnitt \ref{subsec:specPatterns}) und anschließend generische Sprachmuster zur automatischen Sprachverarbeitung (Unterabschnitt \ref{subsec:genPatterns}) vorgestellt. Zwischen den beiden Sprachmustern erfolgt in Unterabschnitt \ref{subsec:Logic} ein kurzer Einschub in die Logik, da das generische Pattern auf dieser aufbaut. Daraufhin werden diese kurz diskutiert (Unterabschnitt \ref{subsec:discPatterns}) und letzendlich wird die der vorliegenden Arbeit zugrundeliegende Vearbeitungskette zur Sprachverarbeitung in Unterabschnitt \ref{subsec:FormProcLang} eingeführt.
\subsection{Spezialisierte Sprachmuster}
\label{subsec:specPatterns}
Spezialisierte Spachmuster (im folgenden kurz: Muster) sind meistens für einen spezifischen Anwendungsfall entwickelt und zugeschnitten. Im folgenden werden zwei spezialisierte Muster vorgestellt.\\
Die "`EARS"' Pattern entwickelt von Mavin und Wilkinsons \cite{MW10} und die "`MASTeR"' Schablonen bereitgestellt von den SOPHISTen \cite{JPQRSSV16}.\\

Zuerst wird sich nun dem "`Easy Approach to Requirements Syntax"' (EARS) zugewandt. Mavin et al. beklagen sich über zahlreiche Probleme mit natürlich sprachlichen Requirements, die da wären Untestbarkeit, ungeeigneter Implementierung, zu viele Worte, Dopplungen, Versäumnix, Komplexität, Vagheit und Mehrdeutigkeit. Zur Beherrschung dieser Probleme schlagen diese ein kleine Anzahl von einfachen Requirement-Strukturen als effiziente Möglichkeit vor um das Schreiben guter Anforderungen zu verbessern. Die Muster mit ihren Erklärungen, Aufbau und Beispiel sind in Tabelle \ref{tab:EARSTemplates} bereitgestellt.
\begin{table}[]
	\centering
	\begin{tabularx}{\textwidth}{p{0.13\textwidth}|p{0.13\textwidth}|p{0.66\textwidth}} 
		\hline
		\multicolumn{3}{c}{Muster}  \\ \hline
		\multirow{3}{0.0\textwidth}{Allgegen-wärtig} & Erklärung & Requirements haben weder Vorbedingung noch Auslöser und sind immer aktiv \\ \cline{2-3}
		 & Aufbau  & <system name> shall <system response> \\ \cline{2-3}
		 & Beispiel & The control system shall indicate the engine oil quantity to the aircraft \\ \hline
		\multirow{3}{0.0\textwidth}{Ereignis-Gesteuert} & Erklärung & Requirements werden nur initiiert sobald ein auslösendes Ereignis an den Syste-Grenzen eintritt \\ \cline{2-3}
		& Aufbau  & When <optional preconditions> <trigger> the <system name> shall <system response> \\ \cline{2-3}
		& Beispiel & When continuous ignition is commanded by the aircraft, the control system shall switch on continuous ignition \\ \hline
		\multirow{3}{0.0\textwidth}{Ungewolltes Verhalten} & Erklärung & Requirements definieren die erforderliche Antwort des Systems um ein ungwollten Zustand zu mildern, oder um das System davon abzuhalten in den ungewollten Zustand überzugehen. Diese sind äquivalent zu den Ereignis-Gesteuerten Requirements, kennzeichen aber durch die If-Then Schlüsselworte die Handhabung ungewollten Verhaltens \\ \cline{2-3}
		& Aufbau  & IF <optional preconditions> <trigger>, THEN the <system name> shall <system response> \\ \cline{2-3}
		& Beispiel & If the computed airspeed fault flag is set, then the control system shall use modelled airspeed \\ \hline
		\multirow{3}{0.0\textwidth}{Zustands-Gesteuert} & Erklärung & Requirements sind aktiv während das System in einem definierten Zustand ist. \\ \cline{2-3}
		& Aufbau  & WHILE <in a specific state> the <system name> shall <system response> \\  \cline{2-3}
		& Beispiel & While the aircraft is in-flight, the control system shall maintain engine fuel flow above XXlbs/sec \\ \hline
		\multirow{3}{0.0\textwidth}{Optionale Merkmale} & Erklärung & Requirements sind nur anwendbar in Systemen, welche ein bestimmtes Merkmal haben \\  \cline{2-3}
		& Aufbau  & WHERE <feature is included> the <system name> shall <system response> \\ \cline{2-3}
		& Beispiel & Where a control system component acts as a firewall, the component shall be Fireproof \\ \hline
	\end{tabularx}
	\caption{EARS-Muster mit Erklärungen, Aufbau und Beispiel, nach \cite{MW10}}
	\label{tab:EARSTemplates}
\end{table}
%\begin{table}	
%	\begin{tabularx}{\textwidth}{|p{0.13\textwidth}|p{0.27\textwidth}|p{0.20\textwidth}|p{0.29\textwidth}|}
%		\hline
%		Muster & Erklärung & Aufbau & Beispiel \\ \hline
%		Allgegen-wärtig & Requirements haben weder Vorbedingung noch Auslöser und sind immer aktiv & <system name> shall <system response> & The control system shall indicate the engine oil quantity to the aircraft \\ \hline
%		Ereignis-Gesteuert & Requirements werden nur initiiert sobald ein auslösendes Ereignis an den Syste-Grenzen eintritt& When <optional preconditions> <trigger> the <system name> shall <system response>& When continuous ignition is commanded by the aircraft, the control system shall switch on continuous ignition\\ \hline
%		Ungewolltes Verhalten & Requirements definieren die erforderliche Antwort des Systems um ein ungwollten Zustand zu mildern, oder um das System davon abzuhalten in den ungewollten Zustand überzugehen. Diese sind äquivalent zu den Ereignis-Gesteuerten Requirements, kennzeichen aber durch die If-Then Schlüsselworte die Handhabung ungewollten Verhaltens & IF <optional preconditions> <trigger>, THEN the <system name> shall <system response> & If the computed airspeed fault flag is set, then the control system shall use modelled airspeed \\ \hline
%		Zustands-Gesteuert& Requirements sind aktiv während das System in einem definierten Zustand ist. & WHILE <in a specific state> the <system name> shall <system response> & While the aircraft is in-flight, the control system shall maintain engine fuel flow above XXlbs/sec\\ \hline
%		Optionale Merkmale& Requirementssind nur anwendbar in Systemen, welche ein bestimmtes Merkmal haben & WHERE <feature is included> the <system name> shall <system response> & Where a control system component acts as a firewall, the component shall be Fireproof  \\ \hline	
%	\end{tabularx}
%	\caption{EARS-Muster mit Erklärungen, Aufbau und Beispiel, nach \cite{MW10}}
%	\label{tab:EARSTemplates}
%\end{table}
Laut Mavin et al. kann mithilfe dieser Patterns die Anzahl der benötigen Worte und die Komplexität der Beschreibung eines Requirements deutlich reduziert werden. Außerdem soll die Identifikation redundanter oder fehlender Anforderungen vereinfacht werden.\\

Die Satzschablonen "`Mustergültige Anforderungen - die SOPHIST Templates für Requirements"' (MASTeR) stellen ebenfalls Vorlagen für hochqualitative Requirements bereit. "Die Arbeit mit einer Satzschablone für Anforderungen trägt ihren Teil dazu bei, dass Anforderungssätze
einheitlich strukturiert sind und so gewisse Defekte (wie z. B. das Fehlen des
Akteurs) von vornherein vermieden werden`"' \cite{JPQRSSV16}. Beispielhaft ist in Abbildung \ref{img:Func_MASTeR} der sogenannte Funktions-Master, eine Schablone zur Formulierung funktionaler Anforderungen abgebildet.
\begin{figure}
	\centering
	\includegraphics[width=0.99\textwidth]{images/Sophist_Patterns.png}
	\caption{FunktionsMASTeR zur Formulierung funktionaler Anforderungen \cite{JPQRSSV16}}
	\label{img:Func_MASTeR}
\end{figure}
Die rechtliche Verbindlichkeit ist dabei vereinfacht wie folgt festgelegt \cite{JPQRSSV16}. 
\begin{itemize}
	\item[] \textbf{Muss}: Alle Anforderungen, die mit MUSS formuliert sind, sind verpflichtend
	in der Umsetzung. Die Abnahme eines Produkts kann verweigert werden, sollte das System einer MUSS-Anforderung nicht entsprechen.
	\item[] \textbf{Sollte}: Formulierungen mit SOLLTE stellen einen Wunsch eines Stakeholders
	dar. Sie sind nicht verpflichtend und müssen nicht erfüllt werden. Allerdings erhöht ihre Umsetzung die Zufriedenheit der Stakeholder und ihre Dokumentation verbessert die Zusammenarbeit und Kommunikation zwischen Stakeholdern und Entwicklern/Auftragnehmern.
	\item[] \textbf{Wird}: Mit WIRD dokumentieren Sie die Absicht eines Stakeholders. Eine mit WIRD formulierte Anforderung dient als Vorbereitung für eine in der Zukunft liegende Integration einer Funktion. Sie ist verpflichtend in der Umsetzung zu berücksichtigen, auch wenn ihre Realisierung zunächst nicht getestet wird.
\end{itemize}
Oft sind bestimmte Funktionen eines Systems an Bedingungen gebunden. Um dies in den Schablonen abzufangen, kann dem System dabei noch eine Bedingung vorangestellt werden, welche wiederrum durch eine Schablone, dem sogenannten BedingungsMASTeR (Abbildung \ref{img:Cond_MASTeR}), beschrieben wird. 
\begin{figure}
	\centering
	\includegraphics[width=0.99\textwidth]{images/Sophist_Patterns2.png}
	\caption{BedingungsMASTeR zur Formulierung von Bedingungen \cite{JPQRSSV16}}
	\label{img:Cond_MASTeR}
\end{figure}
Für nicht-funktionale Requirements steht als Universal-Schablone der EigenschaftsMASTeR zur Verfügung (Abbildung \ref{img:Prop_MASTeR}).
\begin{figure}
	\centering
	\includegraphics[width=0.99\textwidth]{images/Sophist_Patterns3.png}
	\caption{EigenschaftsMASTeR zur Formulierung nicht-funktionaler Anforderungen \cite{JPQRSSV16}}
	\label{img:Prop_MASTeR}
\end{figure}
Laut den Sophisten ist es anhand dieser Schablonen möglich "`die natürlichsprachliche Dokumentaion von Anforderungen zu verbessern"' \cite{JPQRSSV16} und fundamentale Fehler von Beginn an zu vermeiden.

\subsection{Einschub: Logik}
\label{subsec:Logic}
Im folgenden werden zwei Unterformen der Logik behandelt. Die Prädikatenlogik (First Order Logic, FOL) mit der konjunktive Normalform (Conjunctive Normal Form, CNF) als Darstellung dieser und die Lineare Temporal Logik (Linear Temporal Logic, LTL).
\begin{quote}
	"`\textit{Die Prädikatenlogik ist eine an logische Grundkonzepte angelehnte Form der Wissensrepräsentation. Die am häufigsten verwendete Prädikatenlogik erster Ordnung umfasst: Variablen, Konstanten, Funktionen, Prädikate und Ausdrücke mit Verknüpfungsapparaturen und Quantoren. Ein Prädikat ist eine Aussage mit einer Menge von Parametern, der aufgrund vorhandener Datenobjekte (Faktenwissen) ein Wahrheitswert zugeordnet werden kann.}"' \cite{LS17}
\end{quote}
Die First Order Logic macht Gebrauch von folgenden Arten logischer Symbole um First-Order Terme und Formeln zusammenzusetzen \cite{BM11}:
\begin{itemize}
	\item[] \textbf{Verknüpfungsoperatoren}: Diese sind nötig um aus einfachen Formeln komplexere zu erzeugen. Eine Auswahl dieser sind Tabelle \ref{tab:junctors} zu entnehmen
	\item[] \textbf{Variablen}: Eine endliche Menge von symbolischen Variablen.
	\item[] \textbf{Quantoren}: In der Prädikatenlogik kommen zwei Quantoren zum Einsatz: $\forall$ um eine Aussage über alle Variablen in der Domäne zu machen und $\exists$ um etwas über mindestens eine Variable mit einen bestimmten Attribut auszusagen.
	\item[] \textbf{Klammern}: Klammer-Symbole können um symbolische Folgen gesetzt werden und so als "`Zeichensetzung"' dienen.
\end{itemize}
\begin{table}[]
	\centering
	\begin{tabular}{|c|c|c|c|}
		\hline
	Operationen & Junktoren  & Bezeichnung & Beispiel \\ \hline
	Negation & NOT & $\lnot$ oder $!$ & NOT A \\ \hline
	Konjunktion	& AND & $\land$ & A AND B \\ \hline
	Disjunktion & OR  & $\lor$ & A OR B \\ \hline
	Implikation	& IMPLIES & $\rightarrow$ & A I B \\ \hline
	\end{tabular}
	\caption{Verknüpfungsoperationen der Prädikatenlogik}
	\label{tab:junctors}
	\end{table}
Eine konjunktive Normalforn zeichent sich dadurch aus, dass sie eine Konjunktion von Disjunktionstermen ist. Disjunktionsterme sind wiederrum Disjunktionen von Variablen oder negierten Variablen \cite{BN77}. Ausgedrückt als Formel folgt
\begin{equation}
CNF = \bigwedge_{i} \bigvee_{j} (\lnot) x_{ij} \text{,}
\label{eq:CNF}
\end{equation}
wobei $x_{ij}$ für eine Variable steht. Mithilfe der konjunktiven Normalform ist es möglich eine eindeutige Darstellung von Ausdrücken in der Prädikatenlogik zu erhalten.\\

Die Lineare Temporal Logik hat ihren Urspung in der von Arthur Prior entwickelten Tense Logic (TL). Priors Motivation zur Entwicklung der Tense Logic war größtenteils philosphisch \cite{Zal17}. Diesen störte die Unandwendbarkeit einer formalen logischen Notation auf philosophische Probleme und Lösungen bezüglich der Zeit. Die ursprüngliche Tense Logic von Prior bestand aus vier zeitlichen Operatoren $P$ ("`It has at some time been the case that …"'), $F$ ("`It will at some time be the case that …"'), $H$ ("`It has always been the case that …"') und $G$ ("`It will always be the case that …"'). Es sei vorweggenommen, dass im weiteren Verlauf der vorliegenden Arbeit die Operatoren $P$ und $H$ eine untergeornete Rolle spielen, da technische Systeme im Allgemeinen als eine vorwärtsgerichte Folge von Systemzuständen, sozusagen Reaktionen auf bestimmte Aktionen, anzusehen sind und deshalb Aussagen über die Vergangenheit nur wenig sinnhaft sind. Implizit war dieser Logik schon ein linearer Zeitfluss zugrunde gelegt. Prior selbst fügte noch Erweiterungen zu seiner ursprünglichen Sprache hinzu. Darunter fällt auch der Operator $X$ ("`neXttime"'). Als einer der wichtigsten Erweiterungen zählen die Operatoren $S$ ("`Since"') and $U$ ("`Until"') die von Hans Kamp in seiner Doktorarbeit eingeführt wurden. Im weiteren Verlauf der Arbeit ist, aus bereits genannten Gründen, erneut nur $U$ von Interesse. Die Sprachliche Bedeutung von $\phi U \psi$ ist dabei "`$\phi \text{ will be true until a time when } \psi \text{ is true}$"'. Weiter bewies Kamp die Fähigkeit von $U$ alle vorwärtsgerichteten Logiken, wie zum Beispiel $F$, darstellen zu können. Aufbauend auf diesen Arbeiten entwickelte sich die Lineare Temporal Logik, welche zuerst von Pnueli vorgeschlagen und von Gabbay zum ersten Mal explizit axiomatisiert wurde. Die LTL besteht nur aus vorwärtsgerichten Operatoren. Eine Übersicht über diese bietet Tabelle \ref{tab:operators_LTL}.
\begin{table}[]
	\centering
	\begin{tabular}{|c|c|c|c|}
		\hline
		Operationen & Symbol  & Bezeichnung & Beispiel \\ \hline
		Global & G & $\square$ oder $!$ & G A \\ \hline
		neXt	& X & $\circ$ & X B \\ \hline
		Until & U  & $U$ & A $U$ B \\ \hline
		Future bzw. Maybe	& F bzw. M & $\diamond$ & F A bzw. M A \\ \hline
	\end{tabular}
	\caption{Operatoren der Linearen Tempoal Logik um zeitliche Zusammenhänge darzustellen}
	\label{tab:operators_LTL}
\end{table}\\
Für eine ausführliche Einführung in die vorgestellte Logik sei beispielweise auf  die Literatur von Prior \cite{BP58}, \cite{Pri67} und \cite{PH03} und Hans Kamp \cite{Ka68} verwiesen. Ebenfalls bietet die Stanford Encyclopedia of Philosophy \cite{Zal17} einen guten Überblick zu der Thematik.\\ 

\subsection{Generisches Sprachmuster}
\label{subsec:genPatterns}
Generische Sprachmuster sind, im Gegensatz zu speziellen Pattern, nicht nur für bestimmte Zwecke entworfen und zugeschnitten, sondern über ein breites Spektrum anwendbar.\\
Dwyer et al. \cite{DAC98} entwickelten das Specification Pattern System (SPS). Motivation zur Entwicklung war, dass formale Methoden den Anwendern substantielle Vorteile zur Verfügung stellen. Die Barriere zur Anwendung diesbezüglicher Methoden sehen Dwyer et al. aber vor allem in der manuellen Übersetzung der Zusammenhänge in den Logikraum. Diese Transformation ist aber zur weiteren maschinellen Verarbeitung unabdingbar. Die SPS soll zur Überwindung dieser Barriere beitragen, indem es natürlichsprachlichen Beschreibungen eine direkte Abbildung in den Logikraum (u.a. Lineare Temporal Logik) zuweist. Disbezüglich formte sich die Idee, die aus der Softwareentwicklung bekannten Entwurfs-Muster aufzugreifen, die Notation von Muster auf Spezifikationen anzupassen und anschließend anzuwenden.\\
Ein Specification Pattern ist eine generalisierte Beschreibung einer allgemein vorkommenden Anforderung bezüglich zulässigen Zustands- oder Ereignisfolgen in einem finiten Zustandsmodell eines Systems. Weiter beschreibt dieses die grundlegende Struktur einiger Aspekte des Verhaltens eines Systems und bietet Ausdrücke dieses Verhaltens in einer Palette von Formalismen \cite{DAC98}. Die Specification Pattern von Dwyer et al. werden als Linear Temporal Logic (LTL), Computation Tree Logic(CTL), Graphical Interval Logic (GIL), Quantified Regular Expressions (QRE), INCA Queries, Action Computation Tree Logic und Regular Alternation-Free Mu-Calculus bereitgestellt. Es sei an dieser Stelle vorweggenommen, dass in der vorliegenden Arbeit nur von der Abbildugn der SPS in LTL Gebrauch gemacht worden ist. Die Lineare Temporal Logik zeichnet sich durch die sequenzielle Abfolge von Zuständen oder Ereignissen entlang eines Pfades aus. Dies erscheint deutlich praktikabler zur Darstellung einer Funktion als zum Beispiel die Computation Tree Logik, welche darauf abzielt den formal beschriebenen Zusammenhang auf allen permutierten Pfaden der Zustände oder Ereignissen nachzuprüfen.\\    
Die Einteilung erfolgt dabei in elf Basis-Pattern, welche nochmal in fünf Geltungsbereiche unterteilt werden. Beispielhaft ist das sogenannte Response-Pattern in Abbildung \ref{img:Dwyer_Response_Pattern} dargestellt. Für eine vollständige Beschreibung aller Patterns sei auf die Internetseite des Specification Pattern System verwiesen \cite{DACDAP17}.\\
\begin{figure}
	\centering
	\includegraphics[width=0.99\textwidth]{images/Dwyer_Response_Pattern.png}
	\caption{Response Pattern \cite{DAC98}}
	\label{img:Dwyer_Response_Pattern}
\end{figure}
Die Abbildung der Pattern unter anderem von SPS zu LTL wurde dabei empirisch validiert. Darunter fallen Reviews unter Projekt-Mitgliedern, die Zuhilfenahme von Experten bezüglich der speziellen Patterns und die Ausführung in Finite State Verification Tools um kleine finite Zustandstransitionsmodelle zu erzeugen, welche befriedigende Sequenzen von Zuständen und Ereignissen lieferten \cite{DACDAP17}.
\subsection{Diskussion der Sprachmuster}
\label{subsec:discPatterns}
Prinzipiell lässt sich schlussfolgern, dass spezialisierte Pattern sehr gut für die Zwecke geeignet sind, für welche sie entwickelt wurden. Im Falle des EARS ist von Mavin et al. \cite{MW10} eine deutliche Reduzierung der Wortanzahl und der Komplexität von Requirements gezeigt worden. Gleichermaßen eignen sich die MASTeR-Schablonen der Sophisten um Requirements einheitlich zu strukturieren, gewisse Defekte, wie das Fehlen von einem Akteur, und sprachliche Effekte zu vermeiden \cite{JPQRSSV16}. Ebenfalls sind beide Muster in der Lage sowohl funktionale, als auch nicht-funktionale Requirements zu beschreiben und zeitliche Bedingungen zu berücksichtigen.\\ 
Im Vergleich zu generischen Pattern sind spezialisierte Pattern hinsichtlich ihres Zweck sowohl bei der Schwierigkeit der Anwendung als auch bei der Umsetzung eben jenes Zwecks überlegen.\\
Vorteilhaft verhalten sich die generischen Muster gegenüber den spezialisierten Mustern jedoch bei der maschinellen Sprachverarbeitung. Die Problematik bei spezialisierten und der Gewinn der generischen Pattern ist in Abbildung \ref{img:NL2Logic} illustriert.\\
\begin{figure}
	\centering
	\includegraphics[width=0.6\textwidth]{images/NL2Logic.pdf}
	\caption{Abbildungsproblematik vom natürlichsprachlichen Raum (NL) in den Logikraum (Logic)}
	\label{img:NL2Logic}
\end{figure}
Die spezialisierten Muster liegen im Raum der allgemeinen natürlichen Sprache (NL). Von diesem sind sowohl injektive als auch surjektive Abbildungen in den Logikraum (Logic) möglich. Jedoch sind die Abbildung im allgemeinen vom NL in Logic niemals bijektiv. Somit ist eine automatische maschinelle Sprachverarbeitung aus dem allgemeinen natürlich sprachlichen Raum schwer vorstellbar und nach heutigen Stand der Technik nicht umsetzbar. Bei den Specification Pattern System von Dwyer et al. \cite{DAC98} handelt es sich um eine Teilmenge der NL. Die Lineare Temporal Logik ist wiederrum eine Teilmenge des Logikraum. Für das SPS ist eine eindeutige Entsprechung im Logikraum zum Beispiel der LTL empirisch nachgewiesen worden. Somit lässt sich der Gebrauch von generischen Pattern rechtfertigen, auch wenn die Erstellung von Spezifikation in diesen anfänglich weniger intuitiv sein kann.
\subsection{Formalisierungs-Prozess}
\label{subsec:FormProcLang}
Die, der vorliegenden Arbeit zugrundeliegende, Verarbeitungskette zur Sprachverarbeitung wurde von Walter et al. \cite{WHPR17} entwickelt und ist in Abbildung \ref{img:Form_Proc} veranschaulicht.\\
\begin{figure}[H]
	\centering
	\includegraphics[width=0.99\textwidth]{images/formalization-process0.pdf}
	\caption{Formalisierungs Prozess der Sprachverarbeitung \cite{WHPR17}}
	\label{img:Form_Proc}
\end{figure}
Da sowohl in den Veröffentlichungen von Walter et al. \cite{WHPR17} und \cite{WSPR17} als auch in der vorliegenden Arbeit die Formalisierung auf einen natürlichsprachlichen Anforderungssatz eines realen Systems beruht, ist eine manueller Übergang zu den Specification Pattern System notwendig.
Die Abbildung von SPS zu LTL kann, wie bereits besprochen, maschinell erfolgen.\\
In der Natur des Requirements steckt die Beschreibung eines Teilsystems beziehungsweise einer Systemfunktion. In einem technischen Systemen gibt es per Definition keinen vorgeschriebenen Abfolge von Systemzuständen, sondern abhängig von aktuellen Zustand und Eingang sind verschiedene folgende Zustände möglich. Aus dieser Entscheidungsvielfalt folgt sofort eine Vielzahl von möglichen Pfaden innerhalb eines technischen Systems.\\
\begin{figure}
	\centering
	\includegraphics[width=0.99\textwidth]{images/elimination-of-operators-combined.pdf}
	\caption{Auflösung der linear temporal logischen Operatoren $G$,$U$ und $X$ \cite{WHPR17}}
	\label{img:Elim_LTL}
\end{figure}
Das Naturell von Tests ist jedoch eine vorgeschriebe sequenzielle Abfolge von Testschritten. Somit kann bei der Test-Formalisierung von einer vorwärts gerichteten verketteten Liste von Zuständen ausgegangen werden. Der linear temporal logische Ausdruck enthält sowohl zeitliche als auch Randbedingungen bezüglich der Reihenfolge. Walter et al. \cite{WHPR17} haben unter Einhaltung dieser Bedingungen und der verketteten Liste als zugrunde liegende Datenstruktur die Weiterverarbeitung des formalen Ausdrucks von Linear Temporal Logic in First Order Logic gezeigt. In der FOL sind die zeitlichen Randbedingungen nicht mehr explizit sondern nur noch implizit gegeben. Diese Umwandlung wird beispielhaft in Abbildung \ref{img:Elim_LTL} an der Auflösung der Operatoren $G$, $U$ und $X$ gezeigt. Vor $Par1$ steht der globale Operator $G$, was zur Folge hat, dass dieser in jedem Zustand gelten muss. Des Weiteren gilt $Par2$ nur bis $Par3$. Dies ist durch den Operator $U$ kenntlich gemacht. Der Operator $X$ steht für ein Eintreten im nächsten Zustand. Der Zusammenhang ist an $Par4$ und $Par5$ zu erkennen.\\
Die Vereinfachung auf ein einfach verkettete Liste ist bei Anforderungen jedoch nicht tragbar. Hier bedarf es einer flexibleren Lösung und dementsprechend muss die Ausprägung der LTL-Ausdrücke in FOL angepasst werden. Eine nähere Beleuchtung des modifzierten Prozesses erfolgt in Kapitel \ref{chap:Model} und speziell der Abbildung von LTL auf FOL in Unterabschnitt \ref{subsec:ltl2fol_lib}.\\
Durch einen Sortier-Algorithmus enthält man schlussendlich ausgehend von der First Order Logic die konjunktive Normalform. Anhand der eindeutigen Darstellung in der konjunktiven Normalform ist ein Vergleich zu anderen Ausdrücken in der Prädikatenlogik möglich.

% Main chapter 3:
% Modell 

\chapter{Prozess}
\label{chap:Model}
Das Kapitel \ref{chap:Model} befasst sicht mit der Modellbildung bezüglich der entwickelten Methodik. Das Ziel des Verfahrens ist die Gewinnung der systembeschreibenden Zustandsautomaten aus formalisierten Anforderungen. Dazu wird in Abschnitt \ref{sec:Overview_Model} zuerst eine Übersicht über dieses gegeben. Anschließend werden in Abschnitt \ref{sec:lib_algo} die verwendeten Bibliotheken und Algorithmen angesprochen. Nachdem in Abschnitt \ref{sec:support_proc} die unterstützenden Prozesse abgehandelt worden sind folgt in Abschnitt \ref{sec:core_proc} der Kernprozess. Schlussendlich werden in Abschnitt \ref{sec:ext_FSM} noch zwei sturkturelle Erweiterungen zur klassischen Finite State Machine eingeführt.

\section{Übersicht über den Gesamtprozess}
\label{sec:Overview_Model}
Die Intention des Abschnittes \ref{sec:Overview_Model} ist es die Zusammenhänge zwischen den einzelnen Bausteinen des Verfahren herzustellen (Unterabschnitt \ref{subsec:ClassDiagram_Model}) und deren Zusammenwirken mittels des Ablaufdiagramms (Unterabschnit \ref{subsec:Flowchart_Model}) darzustellen. Abschließend folgt in Unterabschnitt \ref{subsec:ex_exp_proc} eine kurze Einführung zweier Beispiel-Requirements.
\subsection{Klassendiagramm des Modells}
\label{subsec:ClassDiagram_Model}
Abschnitt \ref{subsec:UML} hat sich mit den Grundlagen des Klassendiagramms als Ausprägung eines Strukturdiagramms beschäftigt. Das Klassendiagramm stellt die statische Struktur, insbesondere die Klassen und ihre Beziehungen zueinander dar. Das Klassendiagramm, welches dem in dieser Arbeit entwickelten Modell zugrunde liegt, ist in Abbildung \ref{img:ClassDiagram_Model} aufgezeigt.\\
\begin{figure}
	\centering
	\includegraphics[width=0.8\textwidth]{images/ClassDiagram.pdf}
	\caption{Klassendiagramm des Modells}
	\label{img:ClassDiagram_Model}
\end{figure}
Das Management der Anforderungen erfolgt in IBM\textsuperscript{\textregistered} Rational\textsuperscript{\textregistered} Doors\textsuperscript{\textregistered}. Die Bereitstellung für das Design Cockpit 43\textsuperscript{\textregistered} erfolgt via dem RequirementsInterchangeFormat (ReqIF). Aufgrund der Zuweisung des Objekt-Typs in Doors\textsuperscript{\textregistered} werden beim Import der ReqIF-Datei automatisch die entsprechenden Instanzen der Klassen der linken Spalte des Klassendiagramms (ReqIFSys, VehicleFunction, Requirement, TestCase, BaseScenario, TestStep) erzeugt. Ein ReqIFSys[tem] hat dabei beliebig viele VehicleFunction[s], welche wiederrum beliebig viele Requirement[s] besitzen. Jedem Requirement können beliebig viele TestCase[s] zugeordnet werden. Ein TestCase besteht aus aus beliebig vielen BaseScenario[s] und TestStep[s]. Die Struktur des aus der ReqIF-Datei entstehendem Systems ist jedoch nicht der Fokus dieser Arbeit und somit sei für eine genauere Beschreibung dieses  auf das Paper von Walter et al. \cite{WHPR17} verwiesen. Die Elimination von redundanten Testfällen (TestCase[s]) und das infolge dessen notwendige Clustering und Neuanordnen dieser ist ebenfalls Gegenstand bereits von Walter et al. behandelt worden \cite{WSPR17} und wird deshalb ebenso nicht im Rahmen der vorliegenden Arbeit näher beleuchtet. Eine ausführliche Beschäftigung mit den übrigen Klassen im Diagramm erfolgt in den nächsten Abschnitten innerhalb dieses Kapitels.
Neben den VehicleFunction[s] hat das ReqIfSys[tem] auch noch 1 bis N SystemStateMachine[s], wobei N eine natürliche Zahl ist. Mehrere systembescheibende Zustandsautomaten eines Systems können entstehen, indem die im Lastenheft aufgeführte Requirements mehr als ein unabhängiges System beschreiben. Die SystemStateMachin baut sich aus mindestens einer normalerweise jedoch N AtomicRequirementStateMachine[s] auf. Die AtomicRequirementsStateMachine[s] enstehen aus der Atomarisierung der RequirementStateMachine[s]. Daraus folgt sofort, dass auch jede RequirementStateMachine 1 bis N AtomicRequirementStateMachine[s] besitzt. Eine RequirementStateMachine ist ein Zustandsautomat, der ein einziges Requirement beschreibt. Somit hat jedes Requirement genau eine RequirementStateMachine. Sowohl die AtomicRequirementStateMachine als auch die RequirementStateMachine setzten sich jeweils aus 1 bis N RequirementStates und Requirement State Transitions zusammen. RequirementStateTransition[s] beschreiben den Übergang  von einem Zustand (RequirementState) innerhalb der RequirementStateMachine zum nächsten. Da Zustände abhängig vom Eingang (Input) unterschiedliche nächste Zustände, oder im Falle eines Endzustandes gar keinen, haben können die Transition jedoch einen eindeutigen Übergang liefers soll, besitzt ein RequirementState 0 bis N RequirementStateTransition[s], eine RequirementStateTransition jedoch nur 1 RequirementState. Ein RequirementState besitzt als Bausteine 1 bis N atomare Zustände (AtomicStates). Analog zu der Requirement-Ebene setzt sich auch die AtomicStateMachine aus sowohl aus 1 bis N AtomicState[s] als auch aus 1 bis N AtomicStateTransitions zusammen. Ebenfalls kann ein AtomicState 0 bis N AtomicStateTransitions haben, die AtomicStateTransition hingegen nur 1 AtomicState. Der atomare Zustand wird durch 1 Parameter und 1 Value aus dem Werteraum dieses Parameters vollständig beschrieben. 
\subsection{Ablaufdiagramm des Prozesses}
\label{subsec:Flowchart_Model}
Das Ablaufdiagramm (Abbildung \ref{img:Flowchart_Model}) beschreibt den Prozess von der natürlich sprachlichen Andorderung bis hin zum systembeschreibenden Zustandsautomaten. Eine nähere Betrachtung der einzelnen Schritte beziehungsweise der Übergänge von einem Zwischenergebnis zum nächsten wird in den folgenden Abschnitten behandelt.\\
\begin{figure}
	\centering
	\includegraphics[width=0.99\textwidth]{images/Ablaufdiagramm171205.pdf}
	\caption{Ablaufdiagramm des Modells}
	\label{img:Flowchart_Model}
\end{figure}
Am Anfang stehen dabei die Requirements in natürlicher Sprache. Die Beschreibung der natürlichsprachlichen Requirements in Specification Patterns System erfolgt manuell. Die Atomarisierung-Bibliothek legt die atomarisierbaren Zustände fest. Durch einen Substitutionsvorgang und mit Hilfe einer Abbildungsbibliothek erfolgt der Übergang der Requirements-Beschreibung von Specification Patterns in Lineare Temporal Logik (LTL). Die Resultate der Substitution werden in einer Substitut-Datenbank abgelegt. Aus dieser werden die enthaltenen Parameter mit ihren Werteräume extrahiert. Anhand der Paarungen von Parametern und Werten sind die Atomic State Machines erstellbar. Bevor die Abbildungsbibliothek von LTL nach First Order Logic (FOL) in das System einkoppelt, können durch Anwenden einen Sortieralgorithmus die Ausdrücke von FOL in eine Konjunktive Normalform (CNF) überführt werden. Diese bilden zusammen mit den Beschreibungen der Requirements in Linearer Temporal Logik und den Atomic State Machines die Requirement State Machines. Die Atomarisierung dieser in Atomic Requirement State Machine erfolgt unter erneuter Anwendung der Atomarisierungs-Bibliothek und der Substitut-Datenbank. Schlussendlich werden durch eine Synthese die Atomic Requirement State Machines zu mindestens einer System State Machine zusammengefügt.
\subsection{Beispiel-Requirements zur Erläuterung des Prozesses}
\label{subsec:ex_exp_proc}
Inhalt dieses Unterabschnittes ist es zwei beispielhafte Requirements vorzustellen. Diese sollen in den folgenden Abschnitten beziehungsweise Unterabschnitten herangezogen werden um die aufgeführten Erläuterungen bezüglich der einzelnen Schritte zu unterstreichen. Dabei werden die Requirements sowohl in natürlicher Sprache als auch in Specification Pattern angegeben.
\subsubsection{Beispiel-Requirement 1}
\label{subsubsec:ex_req_1}
Die natürlichsprachliche Beschreibung des ersten Beispiel-Requirements ist wie folgt:
\begin{quote}
	"`Durch drehen des Schlüssels im Zündschloss in die Position ON wird die Spannungsversorgung sowohl für die Zündung als auch für die Verbraucher eingeschaltet."' 
\end{quote}
Ausgedrückt im Specification Pattern System ergibt sich das Requirement zu:
\begin{quote}
	"`(PowerSupplyIgnition[ON] and PowerSupplyConsumer[ON]) is true after IgnitionKeyLockPosition[ON] "' 
\end{quote}
\subsubsection{Beispiel-Requirement 2}
\label{subsubsec:ex_req_2}
Die natürlichsprachliche Beschreibung des zweiten Beispiel-Requirements ist wie folgt:
\begin{quote}
	"`Nachdem der Lichtdrehschalter in die Position Außenlicht ein bewegt worden ist darf das Ablendlicht nicht aktiviert werden bevor die Spannungsversorgung für die Verbraucher und die Zündung anliegt."' 
\end{quote}
Ausgedrückt im Specification Pattern System ergibt sich das Requirement zu:
\begin{quote}
	"`(PowerSupplyIgnition[ON] and PowerSupplyConsumer[ON]) precedes (LowBeamHeadlightLeft[ON] and LowBeamHeadlightRight[ON]) after RotaryLightSwitchPosition[ExteriorLightON] "' 
\end{quote}
\section{Bibliotheken und Algorithmen}
\label{sec:lib_algo}
Ziel dieses Abschnittes ist es die, für den Prozess, benötigen Blbliotheken und Algorithmen kurz einzuführen. Dazu wird zuerst die Atomarisierungsbibliothek (Unterabschnitt \ref{subsec:atomization_lib}) angesprochen. Daraufhin folgt die Abbildungbibliothek von SPS zu LTL in Unterabschnitt \ref{subsec:sps2ltl_lib}. Abschließend wird die Abbildungsbibliothek von LTL zu FOL (Unterabschnitt \ref{subsec:ltl2fol_lib}) und der damit eng verwobene Sortier-Algorithmus von FOL zu CNF (Unterabschnitt \ref{subsec:fol2cnf_sort_algo}).
\subsection{Atomarisierungs-Bibliothek}
\label{subsec:atomization_lib}
Die Atomarisierungs-Bibliothek ist eine kleine, aber wichtige Biblbiothek. Diese gibt an welche Requirement States, die (wie bereits besprochen in Unterabschnitt \ref{subsec:ClassDiagram_Model}), aus Atomic State zusammengesetzt sind, sich in ihre atomaren Bestandteile zerlegen lassen. Für das Precedence-Pattern gelten die Regeln aus Tabelle \ref{tab:atom_precedence} zur Atomarisierung. Demnach lässt sich die Variable P (Platzhalter für einen Requirement State) atomarisieren. Eine vollständige Auflistung aller Specification Pattern ist Tabelle \ref{tab:atom_lib} in Unterabschnitt \ref{subsec:app_atomization_lib} des Appendix \ref{chap:AppendixB} zu entnehmen.\\
\begin{table}[H]
	\centering
	\begin{tabularx}{\textwidth}{p{0.22\textwidth}|p{0.22\textwidth}|p{0.22\textwidth}|p{0.22\textwidth}}
		\hline
		Specification\newline Pattern & Ausdruck & atomarisierbar & nicht \newline atomarisierbar \\ \hline
		Precedence & S precedes P [...] & P & S \\ 
	\end{tabularx}
	\caption{Atomarisierungs-Bibliothek bezüglich des Precedence-Pattern}
	\label{tab:atom_precedence}
\end{table}
Die drei theoretischen möglichen Atomarisierungen bezüglich des zweiten Beispiel-Requirements (Unterabschnitt \ref{subsubsec:ex_req_2}) wären:
\begin{itemize}
	\item \textbf{AtomicPattern I}: (PowerSupplyIgnition[ON]) precedes (LowBeamHeadlightLeft[ON] and LowBeamHeadlightRight[ON]) after\newline RotaryLightSwitchPosition[ExteriorLightON]\\ \textbf{AtomicPattern II}: (PowerSupplyConsumer[ON]) precedes (LowBeamHeadlightLeft[ON] and LowBeamHeadlightRight[ON]) after\newline RotaryLightSwitchPosition[ExteriorLightON]
	\item \textbf{AtomicPattern I}: (PowerSupplyIgnition[ON] and PowerSupplyConsumer[ON]) precedes LowBeamHeadlightLeft[ON] after\newline RotaryLightSwitchPosition[ExteriorLightON]\\ \textbf{AtomicPattern II}: (PowerSupplyIgnition[ON] and PowerSupplyConsumer[ON]) precedes (LowBeamHeadlightRight[ON]) after\newline RotaryLightSwitchPosition[ExteriorLightON]
	\item \textbf{AtomicPattern I}: PowerSupplyIgnition[ON] precedes LowBeamHeadlightLeft[ON] before RotaryLightSwitchPosition[ExteriorLightON]\\ 
	\textbf{AtomicPattern II}: PowerSupplyIgnition[ON] precedes LowBeamHeadlightRight[ON] after RotaryLightSwitchPosition[ExteriorLightON]\\
	\textbf{AtomicPattern III}: PowerSupplyConsumer[ON] precedes LowBeamHeadlightLeft[ON] before RotaryLightSwitchPosition[ExteriorLightON]\\ 
	\textbf{AtomicPattern IIII}: PowerSupplyConsumer[ON] precedes LowBeamHeadlightRight[ON]) after RotaryLightSwitchPosition[ExteriorLightON]
\end{itemize}
Die dritte Möglichkeit wird dabei generell ausgeschlossen, da in diesem Fall die Abhängigkeiten zwischen der Spannungsversorgung und des Ablendlichts bekannt sein müssen. Diese gehen aber aus den Specification Patterns nicht hervor. Beispielsweise macht eine Abhängigkeit des linken Ablendlichts von wahlweise der Spannungsversorgung der Verbaucher oder derjenigen der Zündung wenig Sinn.\\
Die zweite Möglichkeit wird auch nicht als zulässig erklärt. Ebenfalls wird dies mit der fehlenden Information der Specification Patterns bezüglich der Abhängigkeiten begründet. Für den menschlichen Benutzer ist eine Abhängigkeit bezüglich der Spannungsversorgung der Verbaucher und des Ablendlichts schnell offensichtlich, maschinell ist dies aber schwierig zu entscheiden.\\
Somit bleibt nur noch die erste Möglichkeit. Es kann gefahrlos angenommen werden, dass weder dass linke noch das rechte Ablendlicht aktiviert werden können bevor die Stromversorgung anliegt.
\subsection{Abbildungs-Bibliothek von SPS zu LTL}
\label{subsec:sps2ltl_lib}
Die Abbildungs-Bibliothek von SPS zu LTL bereitgestellt von Dwyer et al. ist bereits in Unterabschnitt \ref{subsec:genPatterns} vorgestellt worden. Tabelle \ref{tab:extract_sps2ltl_lib} liefert einen Ausschnitt aus dieser. Für die Bibliothek sei auf Internetseite des Specification Pattern Systems verwiesen \cite{DACDAP17}.\\
\begin{table}[H]
	\centering
	\begin{tabularx}{\textwidth}{p{0.20\textwidth}|p{0.29\textwidth}|p{0.40\textwidth}}
		\hline
		Specification\newline Pattern & Ausdruck & LTL \\ \hline
		Universality & P is true after Q & $\square$(Q $\rightarrow$ $\square$(P)) \\ 
	\end{tabularx}
	\caption{Abbildungs-Bibliothek von SPS zu LTL (Ausschnitt) \cite{DACDAP17}}
	\label{tab:extract_sps2ltl_lib}
\end{table}
Beispielhaft ergibt sich für den SPS Ausdruck des ersten Beispiel-Requirements der LTL-Ausdruck zu:
\begin{quote}
	"`$\square$(IgnitionKeyLockPosition[ON] $\rightarrow$ $\square$(PowerSupplyIgnition[ON] and PowerSupplyConsumer[ON])) "' 
\end{quote}
\subsection{Abbildungs-Bibliothek von LTL zu FOL (FSM)}
\label{subsec:ltl2fol_lib}
Die Abbildungs-Bibliothek von LTL zu FOL regelt den Übergang von der Notation der Linearen Temporal Logik in Prädikatenlogik. Walter et al. \cite{WHPR17} hat bereits im Rahmen der Formalisierung von Tests die Ausprägung linear temporal logischer Ausdrücke auf First Order Logic auf Basis einer vorwärts gerichteten verketteten Liste illustriert. Auf Grundlage der Erkenntnisen aus Unterabschnitt \ref{subsec:FormProcLang} bedarf es bei Requirements einer flexibleren Lösung.\\
Ausgehend von der Abbildung linear temporal logischer Ausdrücke auf Büchi Automaten (beispielsweise von Gastin et al. \cite{GP01} oder Lu et al. \cite{LL12} behandelt) und einem zustandsorientiertem Fokus ist im Rahmen der vorliegenden Arbeit eine Abbildungsvorschrift von LTL zu FOL entwickelt worden. Ein Auszug aus dieser Abbildungs-Bibliothek ist Tabelle \ref{tab:extract_ltl2fol_lib} entnehmen. Die vollständige Abbildungsbibliothek aller Specification Pattern ist Tabelle \ref{tab:ltl2fol_lib} in Unterabschnitt \ref{subsec:app_ltl2fol_lib} des Appendix \ref{chap:AppendixB} zu entnehmen.
\begin{table}[H]
	\centering
	\begin{tabularx}{\textwidth}{|p{0.47\textwidth}|p{0.475\textwidth}|}
		\hline
		LTL & FOL (FSM) \\ \hline
		G(Q I G (P)) & \includegraphics[width=0.3\textwidth]{images/P_is_true_after_Q.png} \\ \hline
		G NOT Q OR M (NOT P U \newline(S OR G NOT P)) & \includegraphics[width=0.3\textwidth]{images/S_precedes_P_after_Q.png} \\ \hline
		\multicolumn{2}{|c|}{\hspace{0.5cm}\includegraphics[width=0.3\textwidth]{images/LTL2FSM_description.pdf}} \\ \hline
	\end{tabularx}
	\caption{Abbildungs-Bibliothek von LTL zu FOL (Ausschnitt)}
	\label{tab:extract_ltl2fol_lib}
\end{table}
Die First Order Logic beziehungsweise Finite State Machine Repräsentation der beiden Beispiel-Requirements sind Abbildung \ref{imag:ex_req_FSM} zu entnehmen.
\begin{figure}
	\centering
	\includegraphics[width=0.6\textwidth]{images/Ex_Req_1.pdf}\\
	\includegraphics[width=0.6\textwidth]{images/Ex_Req_2.pdf}
	\caption{Finite State Machines der Beispiel-Requirements}
	\label{imag:ex_req_FSM}
\end{figure}
\subsection{Sortier-Algorithmus zum Erhalt der CNF aus der FOL}
\label{subsec:fol2cnf_sort_algo}
Der Sortier-Algorithmus zum Erhalt der CNF formt die Ausdrücke aus den verschiedenen Zuständen in die konjunktive Normalform um. Der Ausdruck in FOL muss dabei Gleichung \ref{eq:CNF} genügen.
Beispielhaft wäre der CNF-Ausdruck des Zustands "`\textit{NOT (LowBeamHeadlightLeft[ON] and LowBeamHeadlightRight[ON])}"' der zweiten Beispiel-FSM "`\textit{(NOT LowBeamHeadlightLeft[ON] and NOT LowBeamHeadlightRight[ON])}"'. 
\section{Unterstützungsprozess}
\label{sec:support_proc}
\subsection{Substitution und Substitut-Datenbank}
\label{subsec:substitution_database}
\subsection{Parameter-Liste}
\label{subsec:parameter_list}
\subsection{Atomic State Machine}
\label{subsec:ASM}
\section{Kernprozess}
\label{sec:core_proc}
\subsection{Requirement in natürlicher Sprache}
\label{subsec:req_NL}
\subsection{Requirement in Specification Pattern}
\label{subsec:req_SPS}
\subsection{Requirement in LTL}
\label{subsec:req_LTL}
\subsection{Requirement State Machine}
\label{subsec:RSM}
\subsection{Atomic Requirement State Machine}
\label{subsec:ARSM}
\subsection{System State Machine}
\label{subsec:SSM}
\section{Strukturelle Erweiterungen der FSM}
\label{sec:ext_FSM}
\subsection{Counter}
\label{subsec:ext_FSM_counter}
\subsection{Timer}
\label{subsec:ext_FSM_timer}
% Main chapter 4:
% Validierung

%Ausblick

\chapter{Diskussion und Validierung}
\label{chap:Validation}
\section{Annahmen und Konfiguration des Systems}
\label{sec:Assumptions_Setup}
\section{Evaluation und Validierung der Prozessergebnisse}
\label{sec:eval__valid_Results}
\section{Diskussion und Zusammenfassung der Prozessergebnisse}
\label{sec:disc_concl_Results}
redundante im schlimmsten Fall widersprechende Requirements, unvollständig spezifiziert, System tut nicht was es soll --> Abdeckung mit automatischer Erzeugung von Zustandsautomaten aus Requirements

Dabei wird insbesondere übersehen, daß darin
enthaltene Mängel später nur schwierig und äußerst kostenintensiv beseitigt
werden können [Boe 81].
Fehleinschätzung und unzureichende Beachtung des Requirements-Engineering
zieht weitreichende Konsequenzen nach sich. Zu nennen sind z.B. die fehlerhafte
Übermittlung von Anforderungen, unkontrolliertes Management durch unrealistische
Kostenschätzungen und Zeitpläne, ein inadäquates Verständnis der Benutzerbedürfnisse
sowie vor allem unvollständige und inkonsistente Spezifikationen und
damit häufiges Fehlverhalten von Software.
Aber selbst wenn in einem Projekt dem Requirements-Engineering der gebührende
Stellenwert eingeräumt wird, tauchen viele Probleme auf, die in prinzipiellen
Mängeln bezüglich durchgängiger Methoden, geeigneter Beschreibungsmittel und
unterstützender Werkzeuge ihre Ursache haben. 

Testpfade im Zustandsautomaten --> man sieht welche tests von welchen requirements abgedeckt werden, ob alle tests von den requirements abgedeckt werden, welche requirements von keinen test überprüft werden...

Dwyer Patterns, da walter bei tests benutzt und funktionsweise gezeigt und formale beschreibung

%% Main chapter 5:
% Approximationsmodell



\chapter{Approximationsalgorithmus}
\label{chap:Approximationsalgorithmus}

Das Kapitel \ref{chap:Approximationsalgorithmus} (Approximationsalgorithmus) stellt einen Algorithmus vor, mithilfe dessen aus einem Testdatensatz neue Fälle approximiert und zum Testdatensatz adaptiert werden können. Diese Verdichtung der Datenpunkte könnte man später als Grundlage für eine Regression nutzen, um so das Prognosemodell aus Kapitel \ref{chap:Modell} (Prognosemodell) mit echten, gemessenen Daten zu validieren und zu einem ganzheitlichen Modell zu vereinigen.




\section{Ansatz}
\label{sec:Ansatz}
In Kapitel \ref{chap:Modell} (Prognosemodell) wurde ein Prognosemodell für die mechanischen Eigenschaften (Flächenträgheitsmoment, Biege- und Torsionssteifigkeit) von Kabelbündeln erarbeitet. Dieses Modell wurde in Kapitel \ref{chap:Validierung} (Validierung) soweit validiert, dass für alle mechanischen Eigenschaften ein Korridor gefunden wurde welcher eine obere und untere Grenze für die realen mechanischen Eigenschaften darstellt. \\
\parskip 12pt \\
Unabhängig von theoretischen Modellen können Messungen durchgeführt werden um Informationen über reale Zusammenhänge zu erlangen. Eine wichtige Aufgabe liegt darin die vorhandenen Messdaten sinnvoll zu nutzen und einzusetzen. Es besteht die Möglichkeit innerhalb der Messwerte einen funktionalen Zusammenhang zur Beschreibung der mechanischen Eigenschaften zu finden. Gelingt dies, existiert sowohl die exakte (reale) Beschreibung des Problems basierend auf realen Messwerten, sowie die mathematische und mechanische Beschreibung in Form des theoretischen Prognosemodells, vereint in einem ganzheitlichen Modell. Die Messwerte führen auf einen Kopplungsfaktor (Fudge factor). Dadurch kann der Korridor auf eine einzelne Funktion reduziert werden. Gelingt dies, liegt ein Modell vor, welches sowohl die theoretisch korrekten Formeln sowie auch die real auftretenden Effekte beinhaltet. Um diese Vereinigung durchzuführen wird ein Set an Messdaten benötigt.\\
\parskip 12pt \\
Um funktionale Zusammenhänge in den Messdaten zu finden, kann grundsätzlich eine Regression durchgeführt werden. Hierbei wird jedoch nur sehr unspezifisch auf das Problem eingegangen. Mithilfe von Dimensionsanalyse und der Technik des Fallbasierten Schließens können die Problemeigenschaften sehr genau berücksichtigt werden und helfen, das Ergebnis der Approximation stark zu verbessern.\\
\parskip 12pt \\
Die Dimensionsanalyse reduziert den Lösungsraum von $n$ (Anzahl der Problemvariablen) auf $m$ (Anzahl dimensionslose Kennzahlen) Dimensionen. Somit entsteht eine Verdichtung der Messwerte. Es werden weniger Messwerte benötigt um die qualitativ gleichwertige Approximation durchzuführen. Das Fallbasierte Schließen nutzt ein vorhandenes Set von Mess- bzw. Trainingsdaten um neue Fälle zu adaptieren. Hierbei werden im Trainingsdatensatz ähnliche Fälle zum zu adaptierenden Fall gesucht. Dies geschieht über das Ähnlichkeitsmaß (siehe \cite{LSB06} und \cite{Her04}). Es kann hierbei eine weitere Eigenschaft der Dimensionsanalyse ausgenutzt werden.\\
\parskip 12pt \\
Wie in \cite{RH98a} und \cite{Rud02} beschrieben, stellen die dimensionslosen Kennzahlen die Bewertungseigenschaften eines Falles dar. Da sich die dimensionslosen Kennzahlen aus den physikalischen Eigenschaften ergeben, sind diese nicht frei wählbar und führen somit auf eine objektive Bewertung. Um das Ähnlichkeitsmaß zu bestimmen, wird der Unterschied ($\delta$ Variablenwert) zwischen Trainingsfall und zu adaptierendem Fall in jeder Eigenschaft (Variable) bestimmt. Da jede Eigenschaft eine eigene Dimension darstellt, wird faktisch das euklidische Distanzmaß als Ähnlichkeitsmaß genutzt. Alle Eigenschaften sind gleich gewichtet (jede andere Gewichtung wäre eine rein subjektive Entscheidung), daher ist die Inverse Distanzgewichtung die korrekte Methode um die Ähnlichkeit von Punkten im Pi-Raum zu ermitteln.\\
\parskip 12pt \\
Das Fallbasierte Schließen (siehe Unterabschnitt \ref{subsec:fallbasiertes-schliessen}) approximiert neue Fälle basierend auf den existierenden Testfällen. Da das euklidische Distanzmaß genutzt wird, kann zur Ermittlung der besten (ähnlichsten) Testpunkte eine Voronoi Diskretisierung genutzt werden (siehe Unterabschnitt \ref{subsec:Voronoi}). Hierbei wird der Raum in Gebiete nächster Nachbarn zerlegt. Der zu adaptierende Fall liegt in einem Gebiet. Dieses Gebiet ist einem Punkt zugeordnet, welches aufgrund der Konstruktion des Verfahrens der nächste Nachbar zum zu adaptierenden Fall ist. Würde man nur auf dem nächsten Nachbarn basierend approximieren, würde faktisch eine Klassifizierung stattfinden. Es werden daher weiter Punkte benötigt. Es zeigt sich, dass die $k$ nächsten Testpunkte nicht zwingend die besten Approximationspunkte sind. Um zwischen Punkten zu mitteln, müssen diese den zu optimierenden Punkt geometrisch umschließen. Dies ist für die $k$ nächsten Nachbarn nicht gegeben. Eine Approximation mit den $k$ nächsten Nachbarn führt auf hinreichend falsche Ergebnisse\\
\parskip 12pt \\
Mit der Technik der Delauny Triangulation (siehe Unterabschnitt \ref{subsec:Delauny}) werden die drei Punkte im $\mathbb{R}^2$ (Im $\mathbb{R}^3$ äquivalent vier Punkte und Tetraeder)  gefunden, welche das Dreieck aufspannen, die den zu adaptierenden Punkt minimal umschließen. Aus den drei Punkten kann mithilfe der Inversen Distanzgewichtung (siehe Unterabschnitt \ref{subsec:inversedist}) der neue Punkt approximiert und zur Falldatenbank hinzugefügt werden. (Falls mehr als drei Punkte zur Approximation genutzt werden sollen, kann das Verfahren auf die Nachbardreiecke der Delauny Triangulation erweitert werden. Dies Erweiterung wird hier nicht durchgeführt, soll aber als mögliche Verbesserung angesprochen werden.(siehe Unterabschnitt \ref{subsec:mehrpunkte}))\\
\parskip 12pt \\
Im folgenden soll nun ein Approximationsalgorithmus entwickelt werden, welches das Problem in den dimensionslosen Raum überführt und aus dem gegebenen Set an Testdaten über einen Fallbasierten Schließmechanismus (CBR) neue Fälle adaptiert. Diese Verdichtung der Testdaten kann dann genutzt werden um funktionale Zusammenhänge zu finden und schlussendlich das theoretische Prognosemodell mit den realen Messdaten zu vereinen. Dieses vereinte Modell basiert zum einen auf den theoretischen physikalischen und mechanischen Grundlagen und problemrelevanten Eigenschaften und berücksichtigt zudem die realen Messwerte und beobachteten Zusammenhänge.

\section{Algorithmus}
\label{sec:Algorithmus}

Im folgenden soll der Approximationsalgorithmus vorgestellt werden. Der Code wurde in Matlab2015a implementiert. Für die Validierung des Modells ist die Performance in Matlab ausreichend. Bei großen Datensets wird dies nicht mehr gegeben sein. Es wird daher empfohlen in anderen, problemspezifischen Umgebungen zu arbeiten. \\
\parskip 12pt \\
Der Pseudocode zeigt das Modell für den Validierungsfall. Es liegt kein Testdatenset vor, sondern dieses wird künstlich erzeugt. Die Validierungs- und Approximationsfälle werden ebenfalls künstlich erzeugt. Diese Methode eignet sich um die Güte und den Fehler des Modells zu ermitteln. Für reale Anwendungen muss der Code insofern abgeändert werden, dass es in diesem Fall keine Erzeugung von Testfällen gibt. Die Fehlerberechnung entfällt aufgrund nicht vorhandener Validierungsfälle.\\
\parskip 12pt \\
Die Input-Variablen geben die Zahl der Trainings- und Approximationspunkte vor. Die Größe der Arrays für die physikalischen Variablen und dimensionslosen Kennzahlen werden manuell angegeben. Das Modell kann in sofern erweitert werden, dass die Dimensionsanalyse automatisiert abläuft. In diesem Fall liegen die Informationen für die Arraygröße ohne Eingabe direkt vor.

\begin{table}[H]
\centering % centering table
\begin{tabular}{|l|l|l|}
\hline \hline
\multicolumn{3}{|l|}{input variables}                                      \\ \hline
\multirow{9}{*}{}  & \multicolumn{2}{l|}{set variables}                   \\ \cline{2-3} 
                   & \multicolumn{2}{l|}{\multirow{5}{*}{}} \\
                   & \multicolumn{2}{l|}{n - number of physical variables}                     \\
                   & \multicolumn{2}{l|}{d - number of dimensions ($\Pi$'s)}                     \\
                   & \multicolumn{2}{l|}{m - number of test cases}                     \\
                   & \multicolumn{2}{l|}{k - number of approximation cases}                     \\
                   & \multicolumn{2}{l|}{}                     \\ \cline{2-3} 
\hline
\end{tabular}
\caption{Algorithmus - Eingabevariablen} %title of the table
\label{tab:algInp}
\end{table} 

Es werden zufällige Trainingsdaten erstellt wobei die letzte Variable und die dimensionslosen Kennzahlen errechnet werden. Des weiteren werden die Validierungsfälle nach dem selben Prinzip erstellt. Die Approximationsfälle entstehen aus den Validierungsfällen wobei die letzte Variable und das letzte $Pi$ nicht übernommen werden.
\begin{table}[H]
\centering % centering table
\begin{tabular}{|l|l|l|}
\hline \hline
\multicolumn{3}{|l|}{generate data}                                      \\ \hline
\multirow{9}{*}{}  & \multicolumn{2}{l|}{generate test cases}                   \\ \cline{2-3} 
                   & \multicolumn{2}{l|}{\multirow{6}{*}{}} \\
                   & \multicolumn{2}{l|}{for i = 1:1:n}                     \\
                   & \multicolumn{2}{l|}{\hspace{5 mm}generate random values for $variable_1$ to $variable_{n-1}$}                     \\
                   & \multicolumn{2}{l|}{\hspace{5 mm}calculate $variable_n$}                     \\
                   & \multicolumn{2}{l|}{\hspace{5 mm}calculate $\Pi_1$ to $\Pi_m$}                     \\
                   & \multicolumn{2}{l|}{end}					     \\ \cline{2-3} 
                   & \multicolumn{2}{l|}{}                     \\ \cline{2-3} 
                   & \multicolumn{2}{l|}{generate validation cases}                     \\ \cline{2-3} 
                   & \multicolumn{2}{l|}{\multirow{5}{*}{}} \\
                   & \multicolumn{2}{l|}{for i = 1:1:k}                     \\ 
                   & \multicolumn{2}{l|}{\hspace{5 mm}generate random values for $variable_1$ to $variable_{n-1}$}                     \\
                   & \multicolumn{2}{l|}{\hspace{5 mm}calculate $variable_n$}                     \\
                   & \multicolumn{2}{l|}{\hspace{5 mm}calculate $\Pi_1$ to $\Pi_m$}                     \\
                   & \multicolumn{2}{l|}{end}                     \\ \cline{2-3} 
                   & \multicolumn{2}{l|}{}                  \\ \cline{2-3} 
                                      & \multicolumn{2}{l|}{generate approximation cases}                     \\ \cline{2-3} 
                   & \multicolumn{2}{l|}{\multirow{4}{*}{}} \\
                   & \multicolumn{2}{l|}{for i = 1:1:k}                     \\ 
                   & \multicolumn{2}{l|}{\hspace{5 mm}copy $variable_1$ to $variable_{n-1}$ from validation}                     \\
                   & \multicolumn{2}{l|}{\hspace{5 mm}copy $\Pi_1$ to $\Pi_{m}$ from validation}                     \\
                   & \multicolumn{2}{l|}{end}                     \\ \cline{2-3} 
\hline
\end{tabular}
\caption{Algorithmus - Datengenerierung} %title of the table
\label{tab:algGenerate}
\end{table} 


Basierend auf den Trainingsfällen wird der Raum mit einer Delauny Triangulation diskretisiert. Daraus kann als komplementär die Voronoi Zerlegung abgeleitet werden. Über die Delauny Triangulation wird später das beste Dreieck zum Approximieren ausgewählt. 
\begin{table}[H]
\centering % centering table
\begin{tabular}{|l|l|l|}
\hline \hline
\multicolumn{3}{|l|}{discretization}                                      \\ \hline
\multirow{9}{*}{}  & \multicolumn{2}{l|}{space discretization}                   \\ \cline{2-3} 
                   & \multicolumn{2}{l|}{\multirow{6}{*}{}} \\
                   & \multicolumn{2}{l|}{dt delauny triangulation ($\Pi_2$ to $\Pi_m$)}                     \\
                   & \multicolumn{2}{l|}{va voronoi areas ($\Pi_2$ to $\Pi_m$)}                     \\
                   & \multicolumn{2}{l|}{for i = 1:1:k}                     \\
                   & \multicolumn{2}{l|}{\hspace{5 mm}delete approximation case if not element of one triangle of dt}                    \\
                    & \multicolumn{2}{l|}{\hspace{5 mm}(point is at the far side and can't be approximated)}                      \\
                    & \multicolumn{2}{l|}{end}                      \\ \cline{2-3} 
\hline
\end{tabular}
\caption{Algorithmus - Diskretisierung} %title of the table
\label{tab:algRaum}
\end{table}

Die Approximation erfolgt über das euklidische Distanzmaß. Zuerst wird die Distanz jedes Punkts des Dreiecks bestimmt, hieraus entsteht das Gewichtungsmaß als Reziproke der Distanz. Die Funktionswerte der Punkte des ausgewählten Dreiecks werden mit den Gewichtungsfaktoren verrechnet und so der fehlende Wert approximiert. Dieser Programmteil stellt das Kernstück des Approximationsalgorithmus dar. Die Wahl des Gewichtungsverfahrens ist opportunistisch und kann mit Begründung auch geändert werden.
\begin{table}[H]
\centering % centering table
\begin{tabular}{|l|l|l|}
\hline \hline
\multicolumn{3}{|l|}{approximate}                                      \\ \hline
\multirow{9}{*}{}  & \multicolumn{2}{l|}{euclidean distance}                   \\ \cline{2-3} 
                   & \multicolumn{2}{l|}{\multirow{5}{*}{}} \\
                   & \multicolumn{2}{l|}{for i = 1:1:k}                     \\
                   & \multicolumn{2}{l|}{\hspace{5 mm}calculate euclidean distance (approximation case to dt($\Pi_1$ to $\Pi_d$)}                     \\
                   & \multicolumn{2}{l|}{\hspace{5 mm}calculate inverse distance weight for dt($\Pi_1$ to $\Pi_d$)}                     \\
                   
                   & \multicolumn{2}{l|}{end}					     \\ \cline{2-3} 
                   & \multicolumn{2}{l|}{}                     \\ \cline{2-3} 
                   & \multicolumn{2}{l|}{calculate value}                     \\ \cline{2-3} 
                   
				& \multicolumn{2}{l|}{\multirow{8}{*}{}} \\
                   & \multicolumn{2}{l|}{for i = 1:1:k}                     \\ 
                   & \multicolumn{2}{l|}{\hspace{5 mm}for j = i:1:d}                     \\
                   & \multicolumn{2}{l|}{\hspace{10 mm}calculate euclidean distance (approximation case to dt($\Pi_1$ to $\Pi_d$)}                     \\
                   & \multicolumn{2}{l|}{\hspace{10 mm}calculate inverse distance weight for dt($\Pi_1$ to $\Pi_d$)}                     \\
                   & \multicolumn{2}{l|}{\hspace{5 mm}end}                     \\
                   & \multicolumn{2}{l|}{\hspace{5 mm} approximation case $\Pi_1$ = $\frac{\sum \Pi_{1_i} weight_i}{\sum weight_i}$  }                    \\
                   & \multicolumn{2}{l|}{end}                     \\ \cline{2-3}                                      
                   
\hline
\end{tabular}
\caption{Algorithmus - Approximation} %title of the table
\label{tab:algapprox}
\end{table} 




Es wird der Fehler des approximierten Wertes für die dimensionslose Kennzahl und die physikalische Variable mit Hilfe des Validierungsfalls berechnet.
\begin{table}[H]
\centering % centering table
\begin{tabular}{|l|l|l|}
\hline \hline
\multicolumn{3}{|l|}{validation}                                      \\ \hline
\multirow{9}{*}{}  & \multicolumn{2}{l|}{error}                   \\ \cline{2-3} 
                   & \multicolumn{2}{l|}{\multirow{4}{*}{}} \\
                   & \multicolumn{2}{l|}{for i = 1:1:k}                     \\
                   & \multicolumn{2}{l|}{\hspace{5 mm}calculate error approximation case $\Pi_1$}                    \\
                    & \multicolumn{2}{l|}{\hspace{5 mm}calculate error approximation case $variable_1$}                      \\
                    & \multicolumn{2}{l|}{end}                        \\ \cline{2-3} 
\hline
\end{tabular}
\caption{Algorithmus - Validierung} %title of the table
\label{tab:algVal}
\end{table}





Es werden drei Plots dargestellt. Zum einen die Raumdiskretisierung mit Delauny Triangulation und Voronoi Gebieten. Außerdem die Darstellung der Trainings-, Validierungs- und Approximationsfälle sowie der Delauny Dreiecke für die Approximation. Der dritte Plot stellt den Fehler der approximierten Fälle dar.
\begin{table}[H]
\centering % centering table
\begin{tabular}{|l|l|l|}
\hline \hline
\multicolumn{3}{|l|}{plot}                                      \\ \hline
\multirow{9}{*}{}  & \multicolumn{2}{l|}{figure1: Raumdiskretisierung (2D if d = 3)}                   \\ \cline{2-3} 
                   & \multicolumn{2}{l|}{\multirow{5}{*}{}} \\
                   & \multicolumn{2}{l|}{plot dt delauny triangulation ($\Pi_2$ to $\Pi_d$)}                     \\
                   & \multicolumn{2}{l|}{plot va voronoi areas ($\Pi_2$ to $\Pi_d$)}                     \\
                   & \multicolumn{2}{l|}{plot approximation cases ($\Pi_2$ to $\Pi_d$)}                     \\
                   & \multicolumn{2}{l|}{mark triangles for approximation cases}                     \\ \cline{2-3} 
                   & \multicolumn{2}{l|}{}                     \\ \cline{2-3} 
                   & \multicolumn{2}{l|}{figure2: test-,validate- and approximate cases (2D if d = 3)}                     \\ \cline{2-3} 
                   & \multicolumn{2}{l|}{\multirow{5}{*}{}} \\
                   & \multicolumn{2}{l|}{plot test cases ($\Pi_1$ to $\Pi_d$)}                     \\ 
                   & \multicolumn{2}{l|}{plot validation cases ($\Pi_1$ to $\Pi_d$)}                     \\
                   & \multicolumn{2}{l|}{plot approximation cases ($\Pi_1$ to $\Pi_d$)}                     \\
                   & \multicolumn{2}{l|}{mark triangles for approximation cases}                       \\ \cline{2-3} 
                   & \multicolumn{2}{l|}{}                  \\ \cline{2-3} 
                                      & \multicolumn{2}{l|}{figure3: error plot (3D if d = 3)}                     \\ \cline{2-3} 
                   & \multicolumn{2}{l|}{\multirow{2}{*}{}} \\
                   & \multicolumn{2}{l|}{plot error ($\Pi_1$ to $\Pi_d$)}                                          \\ \cline{2-3} 
\hline
\end{tabular}
\caption{Algorithmus - Plot} %title of the table
\label{tab:algplot}
\end{table} 


\section{Anwendungsbeispiel - Stoßdämpfer}
\label{sec:anwendungBSP}
\subsection{Anforderungen}
\label{subsec:anforderungenBSP}
Um den Mechanismus des Approximationsalgorithmus vorzustellen, soll ein praktisches Beispiel genutzt werden. Die Anforderungen an den Anwendungsfall sind zum einen ein bekannter physikalischer Zusammenhang und zum anderen eine Reduktion des Lösungsraums auf $m = 3$ (drei dimensionslose Kennzahlen). Prinzipiell ist jedes Problem, auch $m \neq 3$ möglich, jedoch lässt sich das Problem mit $m = 3$ noch visuell darstellen und ist trotzdem hinreichend komplex. Im folgenden soll das Problem ``Stoßdämpfer - Eine schnell bewegte Masse trifft auf ein Hindernis`` vorgestellt werden.\\
\parskip 12pt \\
Das gewählte Problem ist aus \cite{Her04} übernommen und wird dort aus \cite{Szi89} zitiert. Das Stoßdämpfer-Beispiel wurde aufgrund der oben genannten Anforderungen ausgewählt und steht in keinem physikalischen Zusammenhang zur Fragestellung der Eigenschaften von Kabelbündeln. Ziel ist es den Approximationsalgorithmus vorzustellen. Die physikalische Beschreibung, die Dimensionsanalyse und die dimensionslosen Kennzahlen werden daher aus \cite{Szi89} respektive \cite{Her04} übernommen. Alle weiteren Analysen basieren auf eigenen, über einen Zufallsgenerator erzeugten Zahlen und Testfällen und sind losgelöst von der Betrachtung in \cite{Her04}.

\subsection{physikalische Beschreibung}
\label{subsec:physBeschr}
Beim Stoßdämpfer-Problem trifft eine schnell bewegte Masse auf ein Hindernis. Der Stoßdämpfer reduziert die Aufschlagskraft der Masse.
\begin{figure}[h]
	\centering
		\includegraphics[width=1\textwidth]{images/daempfer.jpg}
	\caption[Anwendung - Beispiel: Stoßdämpfer]{Anwendung - Beispiel: Stoßdämpfer \cite{Szi89}}
	\label{fig:daempfer}
\end{figure}
 Die bewegte $Masse[kg]$ mit der Geschwindigkeit $v_0[\frac{m}{s}]$ wird über den Stoßdämpfer mit Federkonstante $c_1[\frac{kg}{s^2}]$ gebremst. Die Kompression der Feder ist gegeben durch $\lambda[m]$. Die Struktur wird ebenfalls als Feder betrachtet, mit Federkonstante $c_2[\frac{kg}{s^2}]$. Es wird angenommen, dass immer die maximale Aufschlagskraft $F[\frac{kg \cdot m}{s^2}]$ der Masse wirkt. 

 
\begin{table}[h]
\centering % centering table
\begin{tabular}{|c|c|c|} % creating eight columns
\hline\hline %inserting double-line
physikalische Variable & Symbol & Dimension  \\ 
\hline % inserts single-line
Aufschlagskraft & F & $\frac{kg \cdot m}{s^2}$ \\
Geschwindigkeit Masse & $v_0$ & $\frac{m}{s}$ \\
Federkonstante Stoßdämpfer & $c_1$ & $\frac{kg}{s^2}$\\
Federkonstante Struktur & $c_2$ & $\frac{kg}{s^2}$\\
lineare Kompression Stoßdämpfer & $\lambda$ & m \\
Masse & m & kg \\
\hline % inserts single-line
\end{tabular}
\label{tab:stoss}
\caption{Anwendung - Beispiel: Stoßdämpfer} %title of the table
\end{table}  
 

 
Das System ist konservativ, es findet kein Energiefluss über die Systemgrenzen statt. Die Deformation der Feder ist linear.

\subsection{Dimensionsanalyse}
\label{subsec:dimAnBsp}

Mit der in Unterabschnitt \ref{subsec:Pi} vorgestellten Methode werden aus den dimensionsbehafteten Größen ($F[\frac{kg\cdot m}{s^2}]$, $v_0[\frac{m}{s}]$, $c_1[\frac{kg}{s^2}]$, $c_2[\frac{kg}{s^2}]$, $\lambda[m]$, $Masse[kg]$) die dimensionslosen Kennzahlen ($\Pi_1, \Pi_2, \Pi_3$) gebildet. Mit dem funktionalen Zusammenhang aus Formel \eqref{f5b}, sowie $n = 6$ und $r = 3$ ergibt sich $m = 3$. Speziell ergeben sich die folgenden Kennzahlen:

\begin{equation}\label{f69}
\Pi_1 = \frac{F}{c_2 d}
\end{equation}
\begin{equation}\label{f70}
\Pi_2 = \frac{v_0}{\lambda}\sqrt{\frac{M}{c_2}}
\end{equation}
\begin{equation}\label{f71}
\Pi_3 = \frac{c_1}{c_2} 
\end{equation}
Aus dem allgemeinen physikalischen Zusammenhang
\begin{equation}\label{f72}
F = \sqrt{c_2(M v_0^2-c_1 \lambda^2)}
\end{equation}
ergibt sich der dimensionslose Zusammenhang der Pi's (siehe Abbildung \ref{fig:piDat})
\begin{equation}\label{f73}
\Pi_1 = \sqrt{\Pi_2^2  - \Pi_3}
\end{equation}

\begin{figure}[h]
	\centering
		\includegraphics[width=1\textwidth]{images/flaeche-pi.jpg}
	\caption[Anwendung - Graph im dimensionslosen Raum]{Anwendung - Graph der Pi's im dimensionslosen Raum (Gleichung \eqref{f73})}
	\label{fig:piDat}
\end{figure}
\newpage
\subsection{Durchführung}
\label{subsec:durchBsp}

Ziel ist es für neue, unbekannte Fälle  (Vorgabe aller Variablen bis auf eine (hier $F$)) ohne das Wissen über den physikalischen Zusammenhang, die fehlende Variable $F$ zu approximieren. Hier werden im Modell für das Anwendungsproblem Testfälle erzeugt.\\ Es gibt hierbei drei Datensets:\begin{itemize}
\item Trainingsfälle: Die Variablen $v_0, c_1, c_2, \lambda, M$ werden als Zufallszahlen $x_i$ mit \\$x~=~\{x~\in~\mathbb{R}~|~0~\le~x~\leq 10\}$ erzeugt. F wird basierend darauf berechnet. Um sicherzustellen, dass F ebenfalls positiv ist, werden nur Fälle berücksichtigt bei denen $M v_0^2 > c_1 \lambda^2$ gilt. In realen Anwendungen werden die Trainingsfälle aus Messdaten gebildet.\\
\parskip 12pt \\
\item Validierungsfälle: Analog zu den Trainingsfällen werden Validierungsfälle erzeugt. Diese Fälle dienen im weiteren der Validierung der Approximationsfälle und zur Berechnung der Fehler. Bei realen Problemstellungen existieren diese Fälle nicht. Es können jedoch Trainingsfälle genutzt werden und als Validierungs- und Approximationsfälle genutzt werden um die Güte des Modells zu ermitteln und Informationen über den Fehler zu erhalten.
\item Approximationsfälle: Die zu approximierenden Fälle sind identisch mit den Approximationsfällen (identische Zahlenwerte für ($v_0, c_1, c_2, \lambda, M$)), allerdings  wird die fehlende Variable $F$ nicht berechnet sondern über das Modell approximiert. Anschließend können die approximierten Werte mit den Werten der Validierungsfälle verglichen werden um die Güte des Modells und den Fehler zu bestimmen. In realen Anwendungen sind Approximationsfälle die Fälle welche den Zugewinn an Wissen und Erkenntnis darstellen.
\end{itemize}



\section{Anwendung mit Testfall}
\label{sec:Anwendung Testfall}
Das in Abschnitt \ref{sec:anwendungBSP} vorgestellte Beispiel eines Stoßdämpfers wird nun genutzt um den in Abschnitt \ref{sec:Algorithmus} vorgestellten Approximationsalgorithmus in einer praktischen Anwendung zu zeigen. Es werden hierfür $m = 1000$ Trainings- sowie $k = 5$ Validierungsfälle genutzt. Für alle physikalischen Variablen ($n = 1$ bis $n_{ges}-1 = 5$) werden wie vorher beschrieben Zufallszahlen $x_i$ mit $x=\{x \in \mathbb{R} | 0 \le x \leq 10\} $ erzeugt. Es werden lediglich Fälle aussortiert bei denen $(M v_0^2 - c_1 \lambda^2)$ negativ ist, da in diesem Fall imaginäre Kräfte entstehen würden.\\
\parskip 12pt \\
Tabelle \ref{tab:anwDatdimbe} zeigt die fünf zufällig erzeugten Testfälle in dimensionsbehafteten Variablen. Die Variable F soll über das Verfahren approximiert werden. Zusätzlich enthält Tabelle \ref{tab:anwDatdimbe} die dimensionsbehafteten Variablen für die drei jeweils zur Approximation benutzten Trainingspunkte. (Es liegen die Variablen für alle $m = 1000$ Trainingspunkte vor, aus Gründen der Übersicht wird hier auf die spätere Auswahl der Trainingspunkte vorgegriffen.)

\begin{table}[H]
\centering % centering table
\scalebox{0.85}{
\begin{tabular}{|r|c|r|r|r|r|r|}
\hline\hline
\# & F & $v_0$ & $c_1$ & $c_2$ & $\lambda$ & M \\ \hline \hline
1        & ? &     0.3764  &  0.4531  &  3.0392  &  1.2020 & 124.7417    \\ \hline
   &      9.3526   & 7.1958  &  0.4772 &   8.6173  &  0.5030    &   0.1984    \\ \hline
   &     5.7139   & 3.7477  &  0.9574  &  3.7809   & 0.7605   &    0.6542   \\ \hline
   &        7.8536  &  3.0958  &  0.1640  &  1.6523   & 2.5623   &    4.0071   \\ \hline \hline
2  &  ? &      9.9932 &   5.1846  &  2.7251 &   3.1945   & 0.7837    \\ \hline
   &      7.2597  &  0.0651  &  7.8243  &  4.1417   & 1.8979 &   9661.6000          \\ \hline
   &      9.9857  &  0.2766  &  8.8869   & 4.7090   & 1.5533  &   557.0012      \\ \hline
   &        9.2959  &  3.3385   & 3.0252 &   1.3698   & 6.1769    &  16.0158   \\ \hline \hline
3  &   ?   &     1.2065  &  6.5139  &  1.3716 &   3.6574   & 68.3730    \\ \hline
   &         5.4779  &  4.8008  &  8.3612  &  1.9814   & 1.3633  &     1.3314   \\ \hline
   &     6.7962  &  0.6228 &   9.1909 &   1.8433   & 4.6881   &  585.3864    \\ \hline
   &   2.6456  &  9.3552   & 9.6888  &  2.1572   & 1.6193   &    0.3273 \\
\hline \hline
4  &     ? &     1.9613 &   0.5559 &   1.1878  &  1.6266  & 16.0034       \\ \hline
   &        4.7928 &   1.7165  &  0.5900  &  8.0707   & 0.1153   &    0.9687   \\ \hline
   &        6.6314  &  7.0918  &  5.8253  &  5.9962   & 0.2313   &    0.1520  \\ \hline
   &         9.3903  &  8.1846 &   4.0871 &   7.1609   & 0.3169   &    0.1899    \\
\hline \hline
5  &      ? &  0.0018 &  9.0307  &  3.6583  &  2.8787 & $2.4177 \cdot 10^7$       \\ \hline
   &          8.8362  &  4.4708  & 8.9423  &  3.5928   & 6.9438   &   22.6591   \\ \hline
   &         5.0640  &  9.2929  &  5.5282  &  2.2414   & 6.8714    &   3.1550  \\ \hline
   &         9.3722  &  2.9950  &  9.1107  &  3.6970   & 5.3263    &  31.4631   \\      
\hline
\end{tabular}}
\caption{Anwendung - Datenset (dimesionsbehaftet)} %title of the table
\label{tab:anwDatdimbe}
\end{table}

\begin{table}[H]
\centering % centering table
\scalebox{0.85}{
\begin{tabular}{|r|c|r|r|c|c|}
\hline\hline
\# & $\Pi_1$ & $\Pi_2$ & $\Pi_3$&  distance & weight \\ \hline \hline
1  & ?      &    2.0061  &  0.1491       & -        & -      \\ \hline
   &    2.1578 &   2.1706  &  0.0554     &  0.1893            &  5.2828      \\ \hline
   &    0.9236  &  1.6559  &  1.8891       &  0.1129            &  8.8547     \\ \hline
   &     2.0279 &   2.8866 &   4.2198       &  0.1342        &  7.4532      \\ \hline \hline
2  & ?      &       1.6776  &  1.9025       & -        &   -     \\ \hline
   &      5.1489 &   5.1560  &  0.0731   &  0.0255            &   39.2137         \\ \hline
   &      0.3542   & 1.6169   & 2.4890    &  0.2596            &    3.8520        \\ \hline
   &        1.9871 &   2.0499  &  0.2532     &  0.3503        &   2.8546     \\ \hline \hline
3  &  ?      &       2.3291 &   4.7490    &   -       &  -      \\ \hline
   &        1.3652 &   1.9368  &  1.8872    &  0.7687            &   1.3009         \\ \hline
   &       0.7864 &   2.3674 &   4.9860     &   0.2401           &    4.1647        \\ \hline
   &      4.7811  &  4.8816  &  0.9715      &   0.2694       &    3.7123    \\
\hline \hline
4  &   ?     &       4.4258  &  0.4680        &  -        & -       \\ \hline
   &       0.3288  &  1.6045  &  2.4664      &  0.8301            &   1.2047         \\ \hline
   &       1.8550  &  1.8816 &   0.0992     &  0.6792            &    1.4724        \\ \hline
   &       1.0987 &   1.8481   & 2.2085      &  0.2422        &   4.1281     \\
\hline \hline
5  &    ?    &        1.6096 &   2.4685      &    -      &   -     \\ \hline
   &        0.7574  &  2.2505   & 4.4913     &   0.0217           &  46.0380        \\ \hline
   &        4.1381  &  4.2065  &  0.5708   &  0.0055            &    182.9383       \\ \hline
   &        0.4760  &  1.6404  &  2.4644    &  0.0311        &   32.1721     \\      
\hline
\end{tabular}}
\caption{Anwendung - Datenset (dimesionslos)} %title of the table
\label{tab:anwDatdimlos}
\end{table}


Die dimensionslosen Kennzahlen $\Pi_1, \Pi_2, \Pi_3$ werden basierend auf den Formeln in Abschnitt \ref{sec:anwendungBSP} errechnet. Tabelle \ref{tab:anwDatdimlos} zeigt diese Größen sowie die Kennzahlen der Trainingspunkte welche die Grundlage der Approximation bilden. Diese Trainingsfälle werden Über die Delauny Triangulation zu den Approximationspunkten zugeordnet. Aus diesen wird über das euklidische Distanzmaß der fehlende Wert in $\Pi_1$ errechnet. Das vollständige Pi-Set kann dann zur Berechnung des fehlenden physikalischen Wertes $F$ genutzt werden.\\
\parskip 12pt \\
\begin{figure}[h]
	\centering
		\includegraphics[width=1\textwidth]{images/Datenmodell2legextra.jpg}
	\caption[Anwendung - Datenmodell]{Anwendung - Datenmodell}
	\label{fig:anwDatmod}
\end{figure}


Die graphische Darstellung der  zufällig erzeugten Punkte erfolgt in Abbildung \ref{fig:anwDatmod}. Hierbei werden die Trainings-, Validierungs- und approximierten Daten sowie die relevanten Dreiecke der Triangulation dargestellt. Die in Abbildung \ref{fig:piDat} dargestellte Oberfläche ist durch die Trainingspunkte in Abbildung \ref{fig:anwDatmod} ersichtlich.\\
\parskip 12pt \\
Aufgrund der Nutzung von Zufallszahlen für die Trainingspunkte, kommt es zu einer Ballung um den Punkt ($\Pi_2$ = 1 /$\Pi_3$ = 1) Dies hat jedoch keine negativen Auswirkungen. Die Abbildung \ref{fig:anwDatmod} zeigt die fünf approximierten Punkte mit dem jeweiligen Validierungspunkt (Kontrollpunkt über  Formelbeziehung gerechnet). Der Fehler der Approximation entspricht dem Unterschied von approximiertem Punkt und Validierungspunkt in $\Pi_1$-Koordinate. Die zur Approximation ausgewählten Trainingspunkte ergeben sich aus dem Delauny Dreieck.
\begin{figure}[h]
	\centering
		\includegraphics[width=1\textwidth]{images/Diskretisierung2leg.jpg}
	\caption[Anwendung - Diskretisierung]{Anwendung - Diskretisierung}
	\label{fig:anwDisk}
\end{figure}
Die Abbildung \ref{fig:anwDisk} zeigt die Diskretisierung über Delauny Triangulation und Voronoi Gebiete im 2D ($\Pi_2$ / $\Pi_3$). Die $\Pi_1$-Koordinate wird approximiert und ist daher für die Diskretisierung nicht relevant. In Abbildung \ref{fig:anwDisk} ist ebenfalls wieder die Ballung der Punkte um den Punkt {($\Pi_2$ = 1 / $\Pi_3$ = 1)} zu sehen. Die Delauny Triangulation ergibt hier eine sehr dichte Wolke. Auf der rechten Seite der Abbildung ist die Diskretisierung von Voronoi Gebieten und Delauny Triangulation anschaulich. Links zeigt sich, dass der Mangel an Punkten in einem Gebiet und eine ungünstige Verteilung zu sehr unschönen Voronoi Diskretisierungen führen können. Dies  ist besonders in Randgebieten oft der Fall. Aus der 2D-Diskretisierung können die vorher angesprochenen besten Delauny Dreiecke bestimmt werden. Für die zu approximierenden Werte sowie die besten Dreiecke, werden in Tabelle \ref{tab:anwDatdimbe} alle dimensionsbehafteten und in Tabelle \ref{tab:anwDatdimlos} alle dimensionslosen Größen dargestellt. Hierbei entspricht die erste Zeile dem zu approximierenden Wert und die drei folgenden Zeilen den drei Punkten des besten Dreiecks. Zusätzlich werden die Distanz und die daraus resultierende Gewichtung für die Approximation angegeben. Man beachte, dass das Gewicht $w$ reziprok zur Distanz $d$ ist.\\
\parskip 12pt \\
Aus den $\Pi_1$-Werten und den Gewichten der drei besten Punkte wird nach dem Prinzip der Inversen Distanzgewichtung (siehe Unterabschnitt \ref{subsec:inversedist}) der $\Pi_1$-Wert des zu approximierenden Punktes ermittelt. Um den ursprünglich gesuchten Wert F zu ermitteln müssen die Daten in den dimensionsbehafteten Raum rücktransformiert werden. Dabei ergibt sich der fehlende Wert. Auf die Rücktransformation soll hier verzichtet werden, das Prinzip wurde im Unterabschnitt \ref{subsec:Pi} vorgestellt. Die Fehlerauswertung für die fünf Approximationsfälle wird in Tabelle \ref{tab:anwFehler} gezeigt. Approximation und Validierung stimmen aufgrund der Methode in $\Pi_2$ und $\Pi_3$ exakt überein. Die Abweichung in $\Pi_1$ stellt den Fehler des Verfahrens dar. Dieser wird nach dem Prinzip der Gleichung \eqref{f68} berechnet.

\begin{table}[H]
\centering % centering table

\begin{tabular}{|r|l|l|l|c|}
\hline \hline
\#       & $\Pi_1$ & $\Pi_2$ & $\Pi_3$ & Err. $\Pi_1 [\%]$ \\ \hline \hline
App 1 &    1.9833  &  2.0061  &  0.1491    &     0.7448              \\ \hline
Val 1    &    1.9686  &  2.0061  &  0.1491    &       -        \\ \hline\hline
App 2 &      0.9715  &  1.6776  &  1.9025        &    1.7366           \\ \hline
Val 2    &        0.9549  &  1.6776  &  1.9025       &      -         \\ \hline \hline
App 3 &     0.9507  &  2.3291  &  4.7490       &    15.6601           \\ \hline
Val 3    &         0.8219  &  2.3291 &   4.7490       &       -        \\ \hline \hline
App 4 &      4.4561 &   4.4258   & 0.4680      &    1.9095           \\ \hline
Val 4    &        4.3726  &  4.4258  &  0.4680       &       -        \\ \hline \hline
App 5 &        0.3514 &   1.6096  &  2.4685        &    0.4888           \\ \hline
Val 5    &        0.3497  &  1.6096  &  2.4685        &       -        \\ \hline
\end{tabular}
\caption[Anwendung - Fehler]{Anwendung - Fehler $\Pi_1$} %title of the table
\label{tab:anwFehler}
\end{table}

Insgesamt kann mit diesem Datenset gezeigt werden, dass der Approximationsmechanismus funktioniert. Der Fehler liegt bei vier der fünf Fälle im Bereich $\le 2\%$. Lediglich Testfall 3 weist einen Fehler von $15\%$ auf Allerdings muss angemerkt werden, dass die erzeugten fünf Testfälle nicht repräsentativ sind und in diesem Fall ein sehr gutmütiges Fehlerverhalten vorliegt. Für große Datensets liegt der Fehler in der Regel für $90\%$ der Approximationspunkte bei $\le 10\%$. Dies soll jedoch nicht genau untersucht werden, da der absolute Approximationsfehler nur in Relation zu anderen Verfahren und Approximationsmethoden eine sinnvolle Aussage über die Güte des Verfahrens zulässt. Hier soll nur die prinzipielle Funktionsweise des Modells gezeigt werden. Außerdem wird klar, dass die approximierten Werte allgemein sinnvoll sind und das Verfahren korrekt funktioniert.

\section{Erweiterungen und Alternativen}
\label{sec:erweiterungen}

Die genutzten Approximationen wurde bisher aus exakt drei Punkten errechnet. Eine potenzielle Erweiterung besteht darin weitere Punkte einzubeziehen. Dies soll in Unterabschnitt \ref{subsec:mehrpunkte} diskutiert werden. Das gewichtete Distanzmaß wurde im Modell bisher aus opportunistischen Gründen genutzt. In Unterabschnitt \ref{subsec:gewichtung} soll diskutiert werden welche alternativen Gewichtungsfunktionen genutzt werden könnten.
\subsection{Zusätzliche Approximationspunkte}
\label{subsec:mehrpunkte}
Die bisher gewählten Punkte umschließen den zu approximierenden Punkt minimal. Es erscheint eventuell vorteilhafter zusätzliche Punkte ebenso nach dem Prinzip zu wählen, dass die Punkte den zu approximierenden Punkt umschließen. Es kann außerdem angenommen werden, dass näher liegende Punkte grundsätzlich auf eine bessere Approximation führen. Die Anforderung des Umschließens und der minimalen Fläche führen darauf, dass die besten neuen Punkte, diejenigen Punkte sind, welche die Nachbardreiecke des ersten Delauny Dreiecks sind. Jede Seite des Dreiecks ist Element eines weiteren Dreiecks. Dieses wird durch zwei Punkte aus dem ersten Dreieck, sowie einem neuen Punkt aufgespannt. Es werden daher insgesamt drei neue Punkte zur Approximation gefunden.

\begin{figure}[h]
	\centering
		\includegraphics[width=1.2\textwidth]{images/Voronoi5.jpg}
	\caption[Anwendung - Zusatzpunkte ]{Anwendung - Zusatzpunkte \cite{For15}}
	\label{fig:mehrPkt}
\end{figure}

Jedes der neu gewonnen Dreiecke hat zwei freiliegende Kanten. Es ergeben sich sechs ($3 \cdot 2 = 6$) neue Dreiecke mit bis zu sechs neuen Approximationspunkten. (Es kann vorkommen, dass ein neuer Punkt Element mehrerer Dreiecke ist und somit weniger als sechs neue Punkte gewonnen werden. Hierzu sind weitere Überlegungen nötig, da das gleichmäßige Umschließen nicht mehr vollständig gegeben ist. Es bleibt daher eine offene Frage, wie in diesem Fall am besten approximiert werden kann)


\subsection{Alternative Gewichtungsmethoden}
\label{subsec:gewichtung}
Nachdem in Unterabschnitt \ref{subsec:mehrpunkte} die relevanten Approximationspunkte bestimmt wurden, sollen nun weitere Gewichtungsmethoden diskutiert werden. Hierbei werden zwei unterschiedliche Möglichkeiten diskutiert. Es werden verschiedene Methoden zur Distanzermittlung genannt und außerdem wird kurz auf die gängigen Verfahren in den FEM hingewiesen. Stehen die Punkte zur Approximation fest, wird immer über ein arithmetisches Mittel approximiert. Der ungewichtete Mittelwert stellt das einfachste Mittel dar, weißt allerdings auch einen sehr großen Fehler auf. Deshalb wird dieses Mittel nicht weiter berücksichtigt.  Wie im Approximationsalgorithmus des Anwendungsbeispiels in Abschnitt \ref{sec:Anwendung Testfall} soll nun auch das gewichtete arithmetische Mittel genutzt werden. Die Gewichtsfunktionen beziehen sich hierbei auf das Ähnlichkeitsmaß. Im dimensionslosen Raum korrelieren Ähnlichkeitsmaß und Distanzmaß, daher kann nun über das Distanzmaß diskutiert werden. Einen allgemeinen Ansatz stellt hier die Minkowski-Norm dar.
\begin{equation}\label{f74}
d_{Minkowski} = \sqrt[r]{\sum_{j=1}^n|x_{p_j}-x_{i_j}|^p}
\end{equation}
wobei $r$ = Potenz der Wertepaardifferenzen, $x_p$ =  Approximationspunkt,\\ $x_i$ = Validierungspunkt $i$ und $j$ = Dimension. Als Sonderfälle lassen sich mit $r = 1$ das City-Block Distanzmaß
\begin{equation}\label{f75}
d_{City-Block} = \sqrt{\sum_{j=1}^n|x_{p_j}-x_{i_j}|^1}
\end{equation}
und mit $r = 2$ das bisher verwendete euklidische Distanzmaß ableiten.
\begin{equation}\label{f76}
d_{Euklid} = \sqrt[p]{\sum_{j=1}^n|x_{p_j}-x_{i_j}|^2}
\end{equation}
Als Alternative kann noch die Chebyshev-Distanz genannt werden. Hierbei wird die größte Einzelabweichung in einer Variablen (Dimension) als Distanzmaß genutzt.
\begin{equation}\label{f77}
d_{Chebyshev} = max|x_{p_j}-x_{i_j}|
\end{equation}
Zusätzlich zu Gewichtungsmethoden kann die numerische Problemlösung genannt werden. Aus der Diskretisierung über die Triangulation entsteht faktisch ein FEM-Problem. Das relevante Dreieck kann aus dem ($x,y$)-Raum (hier ($\Pi_2 , \Pi_3$)-Raum) in den ($\zeta, \eta$)-Raum transformiert werden. Es können zusätzliche Knoten eingeführt werden. Dies ermöglicht zusätzlich zu linearen auch quadratische Basisfunktionen und Ansätze höherer Ordnung zum lösen zu nutzen. Aufgrund der tiefe dieses Themas soll auf  \cite{Hah14} und \cite{MW09} verwiesen werden. Es ist ohne weitere tiefe Untersuchungen nicht möglich konkrete Aussagen über die Güte der genannten Distanzmaße und FEM-Lösungen zu treffen.

% !TEX root = ../ausarbeitung.tex

\begin{abstract}
\section*{Zusammenfassung}
Das so genannte 'Serious Games' keine Neuheit mehr zur heutigen Zeit sind, lässt sich leicht herausfinden. Sie helfen uns neue Dinge leichter zu lernen als über langweilige Papieraufgaben, die keinerlei Spaß bereiten. Diese Spiele wurden bisher meist für andere Fächer als der Mathematik evaluiert. Es gibt zwar Belege dafür, dass man auch in der Mathematik ein solches Spiel erfolgreich anwenden kann um schneller Fortschritt bei Kindern zu erzielen, aber die Frage ist ob diese Spiele nicht nur einen schnelleren Fortschritt gewährleisten, sondern auch Spaß bereiten. Dieser Frage möchte Ich in dieser Arbeit auf den Grund gehen. \\
Über die Entwicklung eines eigenen Spiels, welches über das Prinzip der Partnerzahlen die Addition lehrt, möchten wir diese Frage beantworten. Für diesen Zweck verwenden wir über den Game Experience Questionaire (GEQ), der für unsere Frage den passenden Nutzertest mit den nötigen Kategorien bereitstellt. Im Laufe der Arbeit entstanden zwei Versionen des Spiels mit unterschiedlichen Perspektiven. Über den GEQ konnten wir ebenfalls Tendenzen aufzeigen, welche dieser Versionen besser ist.
%Wissenschaftliche Arbeiten fangen normalerweise mit einer kurzen Zusammenfassung an. Deshalb sollte Ihre Arbeit ebenfalls eine solche Zusammenfassung enthalten. Die Zusammenfassung hat einen ähnlichen Inhalt wie die Motivation, nur viel kürzer. Sie soll kurz beschreiben
%\begin{itemize}
%\item worum es in der Arbeit geht (was war das zu lösende Problem?),
%\item welche Methoden zur Problemlösung angewendet wurden,
%\item wie das ganze evaluiert wurde,
%\item evtl. welches Ergebnis/ Schlussfolgerungen sich daraus ergeben.
%\end{itemize}

%\hfil\rule{0.4\textwidth}{0.4pt}

%Dieses Dokument soll als Ausgangs-Template für Bachelorarbeiten dienen. Gleichzeitig soll es zeigen, wie so ein \qq{fertiges Dokument} aussehen könnte. Um die Seiten gefüllt zu bekommen, wurde Blindtext verwendet.

%Die kurzen Beschreibungen zu den Abschnitten (jeweils über dem Querstrich) wurden \cite{alexandrakirsch2016} entnommen. Diese \qq{Hinweise zum Erstellen von Bachelor-/Masterarbeiten im Arbeitsbereich Mensch-Computer-Interaktion und Künstliche Intelligenz} sind aber auch darüber hinaus zu empfehlen.
\end{abstract}
%$$$$$$$$$$$$$$$$$$$$$$$$$$$$$$$$$$$$$$$$$$$$$$$$$$$$$


%Anhang  Beschriftung mit "`Anhang A"' bzw. "`A."
\appendix	
%    \captionsetup{list=no}	   
\chapter{Appendix A}
\label{chap:AppendixA}
\section{Nicht-Deterministische FSM}
\label{sec:NDFSM}
Die Mealy NDFSM und die Moore NDFSM lassen sich durch das 6-fach-Tupel
\begin{equation}
M = \{S,I,O,\Delta,\Lambda,R\}
\label{eq:DefMNDFSM}
\end{equation}
beschreiben. Beide Zustandsautomaten haben dabei die Relation für den nächsten Zustand
\begin{equation}
\Delta: I \times S \times S \xrightarrow{} B
\label{eq:DefMNDFSMDelta}
\end{equation}
gemein, bei der jede Kombination aus Eingangsignalen $i$ und und aktuellen Zuständen $p$ sich auf eine nicht leere Menge von nächsten Zuständen bezieht. Die charakteristische Funktion für die Ausgangs-Beziehung ergibt sich bei Mealy NDFSM zu 
\begin{equation}
\Lambda: I \times S \times O \xrightarrow{} B \text{ .}
\label{eq:DefMealyNDFSMLambda}
\end{equation}
Jede Kombination aus Eingangssignal $i$ und aktuellen Zustand $p$ bezieht sich hierbei auf eine nicht leere Menge von Ausgangsignalen $o$. Für die Zustands-Restriktions-Relation ergibt sich
\begin{equation}
(i,p,n,o) \in I \times S \times S \times O \text{ sodass } T(i,p,n,o)=1 \text{ falls } \Delta(i,p,n)=1 \text{ und } \Lambda(i,p,o)=1 \text{ .}
\label{re:CondMealyNDFSMT}
\end{equation}
Als ein Sonderfall der Mealy NDFSM ergibt sich $\Lambda$ bei der Moore NDFSM zu
\begin{equation}
\Lambda: \times O \xrightarrow{} B \text{ .}
\label{eq:DefMMooreNDFSMLambda}
\end{equation} 
Dabei bezieht sich jeder aktuelle Zustan $n$ auf eine nicht leere Menge von Ausgangsignalen $o$. Weiterhin folgt 
\begin{equation}
(i,p,n,o) \in I \times S \times S \times O \text{ sodass } T(i,p,n,o)=1 \text{ falls } \Delta(i,p,n)=1 \text{ und } \Lambda(p,o)=1 \text{ .}
\label{eq:CondMooreNDFSMT}
\end{equation}
Bei einer Incompletely Specified FSM (ISFSM) besziehen sich $\Delta$ (\ref{eq:DefMNDFSMDelta}) und $\Lambda$ (\ref{eq:DefMealyNDFSMLambda}, \ref{eq:DefMMooreNDFSMLambda}) entweder auf einen möglichen nächsten Zustand $n$ beziehungsweise Ausgang $o$ oder auf alle Zustände beziehungweise Ausgangssignale.
\chapter{Appendix B}
\label{chap:AppendixB}
\section{Blbliotheken}
\label{sec:libraries}
In diesem Teil des Anhangs sind die vollständigen, in der vorliegenden Arbeiten angewandten Bibliotheken, dargestellt.
\subsection{Atomarisierungs-Bibliothek}
\label{subsec:app_atomization_lib}
Im folgenden ist die vollständige Atomarisierungs-Bibliothek (Tabelle \ref{tab:atom_lib}) mit Auflistung aller Specification Patterns gegeben. Diese gibt an welche Variablen, die als Platzhalter für Requirement States dienen, sich atomarisieren lassen. Kurz gesagt führt sie auf welche Requirement States sich in ihre Atomic States aufspalten lassen.
\begin{table}[H]
	\centering
	\begin{tabularx}{\textwidth}{p{0.22\textwidth}|p{0.22\textwidth}|p{0.22\textwidth}|p{0.22\textwidth}}
		\hline
		Specification\newline Pattern & Ausdruck & atomarisierbar & nicht \newline atomarisierbar \\ \hline
		Absence & P is false [...] & P & - \\ \hline
		Existence & P becomes true [...] & P & - \\ \hline
		Bounded Existence & transitions to P-States occur at most $n$ times [...] & P & - \\ \hline
		Universality & P is true [...] & P & - \\ \hline
		Precedence & S precedes P [...] & P & S \\ \hline
		Response & S responds to P [...] & S & P \\ \hline
		Precedence Chain & S,T precedes P [...] & P & S,T \\ \hline
		Precedence Chain & P precedes S,T [...] & T & P,S \\ \hline
		Response Chain & P responds to S,T [...] & P & S,T \\ \hline
		Response Chain & S,T responds to P [...] & T & P,S \\ \hline
		Constrained Chain Patterns & S,T without Z responds to P [...] & T & P,S,Z \\ 
	\end{tabularx}
	\caption{Atomarisierungs-Bibliothek}
	\label{tab:atom_lib}
\end{table}
%\subsection{Abbildungs-Bibliothek von SPS zu LTL}
%\label{subsec:app_sps2ltl_lib}
\subsection{Abbildungs-Bibliothek von LTL zu FOL}
\label{subsec:app_ltl2fol_lib}
\begin{table}[H]
	\centering
	\begin{tabularx}{\textwidth}{|p{0.47\textwidth}|p{0.475\textwidth}|}
		\hline
		LTL & FOL (FSM) \\ \hline
		G(Q I G (P)) & \includegraphics[width=0.3\textwidth]{images/P_is_true_after_Q.png} \\ \hline
		G NOT Q OR M (NOT P U \newline(S OR G NOT P)) & \includegraphics[width=0.3\textwidth]{images/S_precedes_P_after_Q.png} \\ \hline
		\multicolumn{2}{|c|}{\hspace{0.5cm}\includegraphics[width=0.3\textwidth]{images/LTL2FSM_description.pdf}} \\ \hline
	\end{tabularx}
	\caption{Abbildungs-Bibliothek von LTL zu FOL}
	\label{tab:ltl2fol_lib}
\end{table}
%\renewcommand{\thepage}{\roman{page}}   %%Umstellung auf Roman-Stile
%\setcounter{page}{\value{romcounter}} 


%\bibliographystyle{alpha}
\bibliographystyle{alphadin}
\nocite{*}
\bibliography{bib/lit}



\backmatter					
% Nachspann die Nummerierung der Gliederungseinheiten im Text und im Inhaltsverzeichnis unterdrückt

\pagestyle{empty}
\thispagestyle{empty}

%\vspace{550cm}
%\vspace*{\fill}
\begingroup
\large
\begin{sloppy}
	\textbf{\Large \textbf{Rechteeinräumung zur Aufstellung in der Bibliothek}}\\	
	\parskip 12pt \\
	\scalebox{0.96}{\textbf{Einräumung\hspace{-1.2pt} von\hspace{-1.2pt} Nutzungsrechten\hspace{-1.2pt} an\hspace{-1.2pt} einer\hspace{-1.2pt} Bachelorthesis\hspace{-1.2pt} für\hspace{-1.2pt} Zwe-}}
	\mbox{\textbf{cke der Forschung, der Lehre, des Studiums und der Bibliothek}}\\
	\parskip 12pt\\
	Hiermit übertrage ich, Jan Martin, der \mbox{Universität Stuttgart} das Eigentum an einer von mir der Bibliothek des Instituts für Statik und \mbox{Dynamik} der Luft- und Raumfahrtkonstruktionen kostenlos zur Verfügung\hspace{4pt} gestellten\hspace{3pt} Exemplars\hspace{3pt} meiner\hspace{3pt} Bachelorarbeit\hspace{3pt} mit\hspace{3pt} dem\hspace{3pt} Titel
	\begin{center}\textbf{TBD}\end{center}
	
	und räume der Universität Stuttgart, Instituts für Statik und Dynamik der Luft- und Raumfahrtkonstruktionen, an dieser Arbeit und den im Rahmen dieser Arbeit entstandenen Arbeitsergebnissen ein kostenlos, zeitlich und räumlich unbeschränktes, einfaches Nutzungsrecht für Zwecke der Forschung, der Lehre, des Studiums und der Nutzung der Arbeit für Zwecke der Institutsbibliothek ein.\\
	\parskip 12pt\\
	Mir ist bekannt, dass die Erfassung meiner Arbeit im Online-Katalog der Bibliothek eine dauerhafte, weltweite Sichtbarkeit der bibliographischen Daten der Arbeit (Titel, Autor, Erscheinungsjahr, etc.) bedeutet.\\
	\parskip  24pt\\
	
	\noindent\rule{6cm}{0.4pt}\\
	Ort, Datum, Unterschrift
	
\end{sloppy}	
	
\endgroup
\vspace*{\fill}



\thispagestyle{empty}
\newpage
%\addcontentsline{toc}{chapter}{Index}
\renewcommand{\indexname}{Index}
\printindex


\end{document}
%% Basierend auf einer TeXnicCenter-Vorlage von Mark M�ller
%%%%%%%%%%%%%%%%%%%%%%%%%%%%%%%%%%%%%%%%%%%%%%%%%%%%%%%%%%%%%%%%%%%%%%%

% W�hlen Sie die Optionen aus, indem Sie % vor der Option entfernen  
% Dokumentation des KOMA-Script-Packets: scrguide

%%%%%%%%%%%%%%%%%%%%%%%%%%%%%%%%%%%%%%%%%%%%%%%%%%%%%%%%%%%%%%%%%%%%%%%
%% Optionen zum Layout des Buchs                                     %%
%%%%%%%%%%%%%%%%%%%%%%%%%%%%%%%%%%%%%%%%%%%%%%%%%%%%%%%%%%%%%%%%%%%%%%%
\documentclass[
%a5paper,							% alle weiteren Papierformat einstellbar
%landscape,						% Querformat
12pt,								% Schriftgr��e (12pt, 11pt (Standard))
BCOR 1cm,							% Bindekorrektur, bspw. 1 cm
%DIVcalc,							% f�hrt die Satzspiegelberechnung neu aus
twoside ,							% s. scrguide 2.4
%oneside,							% einseitiges Layout
%twocolumn,						% zweispaltiger Satz
%openany,							% Kapitel k�nnen auch auf linken Seiten beginnen
openright,							% Kapitel k�nnen nur auf rechten Seiten beginnen
%openleft,
%halfparskip*,				% Absatzformatierung s. scrguide 3.1
%headsepline,					% Trennline zum Seitenkopf	
%footsepline,					% Trennline zum Seitenfu�
%notitlepage,					% in-page-Titel, keine eigene Titelseite
%chapterprefix,				% vor Kapitel�berschrift wird "Kapitel Nummer" gesetzt
%appendixprefix,				% Anhang wird "Anhang" vor die �berschrift gesetzt 
%normalheadings,			% �berschriften etwas kleiner (smallheadings)
%idxtotoc,						% Index im Inhaltsverzeichnis
%liststotoc,					% Abb.- und Tab.verzeichnis im Inhalt
%bibtotoc,						% Literaturverzeichnis im Inhalt
%leqno,								% Nummerierung von Gleichungen links
%fleqn,								% Ausgabe von Gleichungen linksb�ndig
%draft								% �berlangen Zeilen in Ausgabe gekennzeichnet
]
{scrbook}

\usepackage{chngcntr}
\counterwithout{equation}{chapter}
%\pagestyle{empty}		% keine Kopf und Fu�zeile (k. Seitenzahl)
%\pagestyle{headings}	% lebender Kolumnentitel  

%% Deutsche Anpassungen %%%%%%%%%%%%%%%%%%%%%%%%%%%%%%%%%%%%%
%\usepackage[english]{babel}
\usepackage[ngerman]{babel}
\usepackage[utf8]{inputenc}
%\usepackage[ansinew]{inputenc}
%\usepackage[latin1]{inputenc}
%\usepackage[T1]{fontenc}
%\usepackage[scaled]{helvet}
\usepackage{arydshln} 
%lines for tabes



%% Falls die automatische Worttrennung in W�rtern mit Umlauten
%% nicht funktionieren sollte oder der Text pixelig aussieht:
%% ==> Installieren Sie die cm-super Fonts (z.B. mit dem mikTeX Package Manager).
%% Eine nicht ganz vollwertige Alternative ist die Verwendung dieses Pakets:
\usepackage{ae, aeguill}


%% Packages f�r Grafiken & Abbildungen %%%%%%%%%%%%%%%%%%%%%%
\usepackage{graphicx} %%Zum Laden von Grafiken
%\usepackage{subfig} %%Teilabbildungen in einer Abbildung
%\usepackage{pst-all} %%PSTricks - nicht verwendbar mit pdfLaTeX

\usepackage{float} % f�r eine bessere Positionierung von Grafiken

\usepackage{wrapfig}

\usepackage{listliketab}  %f�r die getabbte itemize


\usepackage[usenames]{color}
\definecolor{rot}{RGB}{255,195,195}
\definecolor{dunkelblau}{RGB}{160,172,254}
\definecolor{hellblau}{RGB}{164,215,216}
\definecolor{gelb}{RGB}{254,247,205}
\definecolor{gruen}{RGB}{218,255,204}
\definecolor{lila}{RGB}{205,181,217}
\definecolor{grey}{RGB}{200,200,200}

\usepackage{colortbl}    %um Tabellen farbig zu machen

%% Beachten Sie:
%% Die Einbindung einer Grafik erfolgt mit \includegraphics{Dateiname}
%% bzw. �ber den Dialog im Einf�gen-Men�.
%% 
%% Im Modus "LaTeX => PDF" k�nnen Sie u.a. folgende Grafikformate verwenden:
%%   .jpg  .png  .pdf  .mps
%% 
%% In den Modi "LaTeX => DVI", "LaTeX => PS" und "LaTeX => PS => PDF"
%% k�nnen Sie u.a. folgende Grafikformate verwenden:
%%   .eps  .ps  .bmp  .pict  .pntg


%% Bibliographiestil %%%%%%%%%%%%%%%%%%%%%%%%%%%%%%%%%%%%%%%%%%%%%%%%%%
%\usepackage{natbib}



%%  (Franz Gritschneder hinzugefuegte Bibliotheken)

\usepackage{setspace}

\usepackage{fancyhdr}							%f�r Kopf- und Fusszeilen

\usepackage{booktabs}							% Absatz Tabellenumgebung





\usepackage{lettrine}   % f�r Initialen an einem Abschnitt z.B. \lettrine[lines=1]{H}{}auptaug 

%\usepackage[a4paper,
%left=3cm, right=3cm,
%top=3cm,
% bottom=2cm]{geometry}


\usepackage{yfonts}   %Initialen

%\usepackage[T1]{fontenc} % �ndert die Fontkodierung auf T1 Format
\usepackage{pdfpages}
%\ pdf einbinden

%\parindent0mm
\setlength{\parindent}{0pt}



%\usepackage[ps2pdf,%
%linktocpage,%
%colorlinks,%
%bookmarks,%
%bookmarksopen,%
%bookmarksnumbered]{hyperref}



\usepackage{multirow}    % f�r die Tabellenumgebung
%\usepackage{multicolumn} % f�r die Tabellenumgebung - 
\usepackage{tabularx}

\setcounter{secnumdepth}{3}  %gibt die Tiefe der Nummerierung f�r Unterpunkte an

%\chapter{Kapitel1}
%\section{Punkt1}
%\subsection{Unterpunkt1}
%\subsubsection{Unterpunkt2}
%\paragraph{Unterpunkt3}
%\subparagraph{Unterpunkt4


\newcommand{\entspr}{\ensuremath \widehat{=}}

%\setlength{\parindent}{1.4em}
\setlength{\parindent}{0pt} %kein Einzug auf der 1. Zeile eines Absatzes


\usepackage{textcomp}
 %Mathematikbefehle anzeigen
%\usepackage{mathtools} 
 \usepackage{amsmath}
 \numberwithin{equation}{chapter}
 \usepackage{amssymb}
 
 %Zur Erzeugung von Index
 \usepackage{makeidx}
 %Index erstellen
 \makeindex
 
  %captionbefehle einbingen
 \usepackage{caption} 
 \usepackage{array}
 
 \usepackage{pgfplots}				% für Plots direkt in Latex
 \usepackage{tikz}					% Umgebung für Plots
 \usetikzlibrary{patterns}			% Füllungen von Plots mit Mustern

 
 \usepackage{subcaption}
 \usepackage{caption}
 
 \usepackage{bm}					% bold text in math mode
 \usepackage{nicefrac}				% schräge Brüche in Text
 
 \usepackage[colorlinks=true,       %zur anzeige der bookmarks im pdf-ausgabefile
        linkcolor=black,
        citecolor=black,
%        linkcolor=blue,
%        citecolor=red,
        filecolor=black,
        %pagecolor=black,
        urlcolor=black,
        bookmarks=true,
        bookmarksopen=false,
        bookmarksopenlevel=5,
        bookmarksnumbered,
        plainpages=false,
        pdfpagelabels=true]{hyperref}
\usepackage{hyperref}
\hypersetup{
pdftitle = {TBD},
pdfsubject = {Automatisierte Erzeugung von Zustandsautomaten},
pdfauthor = {Jan Martin},
pdfkeywords = {Masterarbeit, tbd},
pdfcreator = {GNU Ghostscript 8.00},
pdfproducer = {LaTeX},
urlcolor = black
}

 

 
 

\makeatletter
\renewenvironment{thebibliography}[1]
{\chapter{Bibliography}%
%{\chapter{\refname}%
\@mkboth{\MakeUppercase\refname}{\MakeUppercase\refname}%
\list{\@biblabel{\@arabic\c@enumiv}}%
{\settowidth\labelwidth{\@biblabel{#1}}%
\leftmargin\labelwidth
\advance\leftmargin\labelsep
\@openbib@code
\usecounter{enumiv}%
\let\p@enumiv\@empty
\renewcommand\theenumiv{\@arabic\c@enumiv}}%
\sloppy
\clubpenalty4000
\@clubpenalty \clubpenalty
\widowpenalty4000%
\sfcode`\.\@m}
{\def\@noitemerr
{\@latex@warning{Empty `thebibliography' environment}}%
\endlist}
\makeatother

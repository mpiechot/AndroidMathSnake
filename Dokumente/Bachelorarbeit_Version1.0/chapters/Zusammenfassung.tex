% !TEX root = ../ausarbeitung.tex

\begin{abstract}
\section*{Zusammenfassung}
Das so genannte 'Serious Games' keine Neuheit mehr zur heutigen Zeit sind, lässt sich leicht herausfinden. Sie helfen uns neue Dinge leichter zu lernen als über langweilige Papieraufgaben, die keinerlei Spaß bereiten. Diese Spiele wurden bisher meist für andere Fächer als der Mathematik evaluiert. Es gibt zwar Belege dafür, dass man auch in der Mathematik ein solches Spiel erfolgreich anwenden kann um schneller Fortschritt bei Kindern zu erzielen, aber die Frage ist ob diese Spiele nicht nur einen schnelleren Fortschritt gewährleisten, sondern auch Spaß bereiten. Dieser Frage möchte Ich in dieser Arbeit auf den Grund gehen. \\
Über die Entwicklung eines eigenen Spiels, welches über das Prinzip der Partnerzahlen die Addition lehrt, möchten wir diese Frage beantworten. Für diesen Zweck verwenden wir über den Game Experience Questionaire (GEQ), der für unsere Frage den passenden Nutzertest mit den nötigen Kategorien bereitstellt. Im Laufe der Arbeit entstanden zwei Versionen des Spiels mit unterschiedlichen Perspektiven. Über den GEQ konnten wir ebenfalls Tendenzen aufzeigen, welche dieser Versionen besser ist.
%Wissenschaftliche Arbeiten fangen normalerweise mit einer kurzen Zusammenfassung an. Deshalb sollte Ihre Arbeit ebenfalls eine solche Zusammenfassung enthalten. Die Zusammenfassung hat einen ähnlichen Inhalt wie die Motivation, nur viel kürzer. Sie soll kurz beschreiben
%\begin{itemize}
%\item worum es in der Arbeit geht (was war das zu lösende Problem?),
%\item welche Methoden zur Problemlösung angewendet wurden,
%\item wie das ganze evaluiert wurde,
%\item evtl. welches Ergebnis/ Schlussfolgerungen sich daraus ergeben.
%\end{itemize}

%\hfil\rule{0.4\textwidth}{0.4pt}

%Dieses Dokument soll als Ausgangs-Template für Bachelorarbeiten dienen. Gleichzeitig soll es zeigen, wie so ein \qq{fertiges Dokument} aussehen könnte. Um die Seiten gefüllt zu bekommen, wurde Blindtext verwendet.

%Die kurzen Beschreibungen zu den Abschnitten (jeweils über dem Querstrich) wurden \cite{alexandrakirsch2016} entnommen. Diese \qq{Hinweise zum Erstellen von Bachelor-/Masterarbeiten im Arbeitsbereich Mensch-Computer-Interaktion und Künstliche Intelligenz} sind aber auch darüber hinaus zu empfehlen.
\end{abstract}
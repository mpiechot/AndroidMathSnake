% !TEX root = ../ausarbeitung.tex

\begin{abstract}
\section*{Zusammenfassung}
%Wissenschaftliche Arbeiten fangen normalerweise mit einer kurzen Zusammenfassung an. Deshalb sollte Ihre Arbeit ebenfalls eine solche Zusammenfassung enthalten. Die Zusammenfassung hat einen ähnlichen Inhalt wie die Motivation, nur viel kürzer. Sie soll kurz beschreiben
%\begin{itemize}
%\item worum es in der Arbeit geht (was war das zu lösende Problem?),
%\item welche Methoden zur Problemlösung angewendet wurden,
%\item wie das ganze evaluiert wurde,
%\item evtl. welches Ergebnis/ Schlussfolgerungen sich daraus ergeben.
%\end{itemize}

%\hfil\rule{0.4\textwidth}{0.4pt}

%Dieses Dokument soll als Ausgangs-Template für Bachelorarbeiten dienen. Gleichzeitig soll es zeigen, wie so ein \qq{fertiges Dokument} aussehen könnte. Um die Seiten gefüllt zu bekommen, wurde Blindtext verwendet.

%Die kurzen Beschreibungen zu den Abschnitten (jeweils über dem Querstrich) wurden \cite{alexandrakirsch2016} entnommen. Diese \qq{Hinweise zum Erstellen von Bachelor-/Masterarbeiten im Arbeitsbereich Mensch-Computer-Interaktion und Künstliche Intelligenz} sind aber auch darüber hinaus zu empfehlen.
\end{abstract}
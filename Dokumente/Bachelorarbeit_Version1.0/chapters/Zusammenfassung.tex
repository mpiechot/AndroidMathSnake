% !TEX root = ../ausarbeitung.tex

\begin{abstract}
\section*{Zusammenfassung}
Serious Games sind in der heutigen Zeit keine Neuheit mehr. Meta-Analysen haben gezeigt,dass man mit Serious Games höhere Lernerfolge erzielen kann als mit konventionellen Lehr- und Lernmethoden. Damit gibt es Belege, dass man über ein solches Spiel verwenden kann, um schneller Fortschritte bei Kindern zu erzielen, jedoch wird hier lediglich die Wirksamkeit untersucht und nicht, ob den Kindern das eingesetzte Spiel Spaß bereitet. Dieser Frage gehe ich in dieser Arbeit auf den Grund. Zusätzlich überprüfe ich ob die Perspektive auf das Spiel einen Einfluss hat. \\
Um diese Fragen beantworten zu können wurde ein Spiel entwickelt, welches über das Prinzip der Partnerzahlen die Addition lehrt. Ich habe zwei Versionen des Spiels entwickelt, die sich in ihren Perspektiven unterscheiden. Anhand eines Nutzertests teste ich welche der Perspektiven für den Spieler als besser empfunden werden und ob das Spiel allgemein Spaß bereitet. Dies Ergebnisse zeigen, dass beide Spiele eine hohe Spielerfahrung aufweisen und dass das Spiel aus der Third-Person-Perspektive als besser empfunden wird.



%Wissenschaftliche Arbeiten fangen normalerweise mit einer kurzen Zusammenfassung an. Deshalb sollte Ihre Arbeit ebenfalls eine solche Zusammenfassung enthalten. Die Zusammenfassung hat einen ähnlichen Inhalt wie die Motivation, nur viel kürzer. Sie soll kurz beschreiben
%\begin{itemize}
%\item worum es in der Arbeit geht (was war das zu lösende Problem?),
%\item welche Methoden zur Problemlösung angewendet wurden,
%\item wie das ganze evaluiert wurde,
%\item evtl. welches Ergebnis/ Schlussfolgerungen sich daraus ergeben.
%\end{itemize}

%\hfil\rule{0.4\textwidth}{0.4pt}

%Dieses Dokument soll als Ausgangs-Template für Bachelorarbeiten dienen. Gleichzeitig soll es zeigen, wie so ein \qq{fertiges Dokument} aussehen könnte. Um die Seiten gefüllt zu bekommen, wurde Blindtext verwendet.

%Die kurzen Beschreibungen zu den Abschnitten (jeweils über dem Querstrich) wurden \cite{alexandrakirsch2016} entnommen. Diese \qq{Hinweise zum Erstellen von Bachelor-/Masterarbeiten im Arbeitsbereich Mensch-Computer-Interaktion und Künstliche Intelligenz} sind aber auch darüber hinaus zu empfehlen.
\end{abstract}
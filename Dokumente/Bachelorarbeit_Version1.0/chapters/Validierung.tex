% Main chapter 4:
% Validierung

%Ausblick

\chapter{Diskussion und Validierung}
\label{chap:Validation}
\section{Annahmen und Konfiguration des Systems}
\label{sec:Assumptions_Setup}
\section{Evaluation und Validierung der Prozessergebnisse}
\label{sec:eval__valid_Results}
\section{Diskussion und Zusammenfassung der Prozessergebnisse}
\label{sec:disc_concl_Results}
redundante im schlimmsten Fall widersprechende Requirements, unvollständig spezifiziert, System tut nicht was es soll --> Abdeckung mit automatischer Erzeugung von Zustandsautomaten aus Requirements

Dabei wird insbesondere übersehen, daß darin
enthaltene Mängel später nur schwierig und äußerst kostenintensiv beseitigt
werden können [Boe 81].
Fehleinschätzung und unzureichende Beachtung des Requirements-Engineering
zieht weitreichende Konsequenzen nach sich. Zu nennen sind z.B. die fehlerhafte
Übermittlung von Anforderungen, unkontrolliertes Management durch unrealistische
Kostenschätzungen und Zeitpläne, ein inadäquates Verständnis der Benutzerbedürfnisse
sowie vor allem unvollständige und inkonsistente Spezifikationen und
damit häufiges Fehlverhalten von Software.
Aber selbst wenn in einem Projekt dem Requirements-Engineering der gebührende
Stellenwert eingeräumt wird, tauchen viele Probleme auf, die in prinzipiellen
Mängeln bezüglich durchgängiger Methoden, geeigneter Beschreibungsmittel und
unterstützender Werkzeuge ihre Ursache haben. 

Testpfade im Zustandsautomaten --> man sieht welche tests von welchen requirements abgedeckt werden, ob alle tests von den requirements abgedeckt werden, welche requirements von keinen test überprüft werden...

Dwyer Patterns, da walter bei tests benutzt und funktionsweise gezeigt und formale beschreibung

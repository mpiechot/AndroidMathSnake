% !TEX root = ../ausarbeitung.tex

\chapter{Einleitung}
Kinder sitzen oftmals mehrere Stunden verzweifelt vor ihren Mathe-Hausaufgaben. Das alles, um gelernten Stoff, durch weitere Übungen, zu festigen. Hat das Kind aber bereits nur noch das Spielen im Kopf entwickelt sich die Hausaufgabenzeit zu einer langwierigen Arbeitszeit mit mäßigem Fortschritt. Aber auch in der Schule selbst können in der heutigen Zeit der Digitalisierung bessere Lernerfolge erzielt werden, wenn man sich die Technik hierfür zunutze macht. Mit dieser Bachelorarbeit möchte ich untersuchen, ob das Lernen der Addition in der Mathematik angenehmer für das Kind gestaltet werden und womöglich auch bessere Lernerfolge herbeiführen kann. Dies wird durch den Ansatz der Serious Games versucht, indem wir das Lernen mit dem Spielen kombinieren. Dabei versuchen wir das Konzept der Partnerzahlen in dieses Spiel einfließen zu lassen, wodurch bereits gute Lernerfolge in anderen Spielen erzielt wurden \cite{Jung2016}. Das Spiel umfasst zwei Versionen, die bis auf die Perspektive identisch sind. Ziel dieser Arbeit ist es, herauszufinden, ob dieses Mathespiel Kindern im Grundschulalter Spaß bereitet und welche Perspektive auf das Spielgeschehen sie besser finden.

Partnerzahlen, oder auch verliebte Zahlen genannt, sind in der Mathematik ein beliebtes Mittel um die Addition zu lehren. Dabei wird dies oft im 10-er Zahlenraum für Natürliche Zahlen angewendet, um zwei Zahlen zu finden, die zusammen addiert 10 ergeben ( zum Beispiel 4 + 6 = 10, womit die Partnerzahl von 4 dem entsprechend 6 wäre um auf die Zahl 10 zu kommen). In diesem Spiel wird dieser Bereich, der gesuchten Zahl, kontinuirlich ausgeweitet. Anfangs wird der Zahlenraum von 3 bis 20 verwendet, bis hin aus dem Zahlenraum von 30 bis 100. Die untere Schranke des kleinsten Zahlenraums ist damit begründet, da hier nur wenig Variationen bestehen um diese Zahlen zu erreichen. Durch den Anstieg des Zahlenraums erhoffe ich mir ein anspruchsvolleres und erweiterbares Mathespiel entwickelt zu haben. Außerdem erhoffe ich mir einen Interessanten Aufgabenbereich für das Spiel gewählt zu haben, der den Kindern es leichter macht die Addition zu verstehen und Spaß an dem Spiel zu haben.

Für das Spiel verwenden wir den Ansatz der Serious Games, aber was sind Serious Games eigentlich? Für diese Art von Spiel gibt es zur heutigen Zeit noch weitere Bezeichnungen, wie 'Educational Game' , 'Game-Based-Learning' oder 'Edutainment'. Dabei ist es schwierig zu sagen, ob alle Begriffe die gleichen Spiele kategorisieren. Eine mögliche Kategorisierung orientiert sich am Verhältnis von didaktischen zu spielerischen Elementen \cite{Wechselberger2009}. Dabei sind Serious Games stark didaktisch ausgeprägt, während Edutainment sich stärker an unterhaltende spielerische Elemente orientiert. Es gibt aber auch Definitionen, die allen Begriffen die gleiche Bedeutung zuordnen. Wie 2013 von Hawlitschek\cite{hawlitschek2013} (p.23)
\begin{quote}
\textit{... digitale Lernspiele sind Computerspiele,}
\begin{itemize}
\item \textit{die explizit und systematisch in Hinblick auf ein bestimmtes Lernziel und für den Einsatz in einem pädagogischen Kontext konzipiert wurden.}
\item \textit{die ein positives Spielerlebnis beim Spieler auslösen.}
\item \textit{deren Effektivität bei der Vermittlung der Lerninhalte nachgewiesen werden konnte.}
\end{itemize}
\end{quote}
Mit dieser Arbeit möchte ich ein solches digitales Lernspiel nach dieser Definition implementieren und vor allem den zweiten Punkt, des positiven Spielerlebnisses, untersuchen.\\
\\
In den Folgenden Kapiteln möchte ich aktuelle Erkenntnisse der Forschung präsentieren, sowie Aufbau und Evaluation des entwickelten Spieles.
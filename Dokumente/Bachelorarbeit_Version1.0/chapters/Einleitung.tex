% !TEX root = ../ausarbeitung.tex

\chapter{Einleitung}
Kinder sitzen oftmals mehrere Stunden verzweifelt vor ihren Mathe-Hausaufgaben. Das alles um gelernten Stoff durch weitere Übungen zu festigen. Hat das Kind aber bereits nur noch das Spielen im Kopf entwickelt sich die Hausaufgabenzeit zu einer langwierigen Arbeitszeit mit mäßigem Fortschritt. Aber auch in der Schule selbst können in der heutigen Zeit der Digitalisierung bessere Lernerfolge erzielt werden, wenn man sich die Technik hierfür zunutze macht. Mit dieser Bachelorarbeit möchten wir untersuchen ob wir das Lernen der Addition in der Mathematik angenehmer für das Kind gestallten und womöglich auch bessere Lernerfolge herbeiführen können. Dies wird durch den Ansatz der 'Serious Games' versucht indem wir das Lernen mit dem Spielen kombinieren. Dabei versuchen wir das Konzept der Partnerzahlen in dieses Spiel einfließen zu lassen, wodurch bereits gute Lernerfolge in anderen Spielen erzielt wurde \cite{Jung2016}. Wir entwickelten zwei Versionen dieses Spiels mit unterschiedlichen Perspektiven. Ziel dieser Arbeit ist es, herauszufinden ob dieses Mathespiel Kindern im Grundschulalter Spaß bereitet und welche Perspektive auf das Spielgeschehen sie besser finden.
\section{Partnerzahlen}
Die Partnerzahlen, oder auch verliebte Zahlen genannt, sind in der Mathematik ein beliebtes Mittel um die Addition zu lernen. Dabei wird dies oft im 10-er Zahlenraum angewendet um zwei Zahlen zu finden, die zusammen addiert 10 ergeben ( zum Beispiel 4 + 6 = 10). In dieser Arbeit weiten wir den Bereich aus und suchen zu einer zufälligen Zahl zunächst im Zahlenraum vom 3 bis 20 die mögliche Partnerzahlen. Die untere Schranke ist damit begründet, da hier nur wenig Variationen bestehen um diese Zahlen zu erreichen. Mit diesem Konzept erhoffen wir uns zum einen ein anspruchsvolleres, erweiterbares Mathespiel entwickelt zu haben, aber auch einen Interessanten Aufgabenbereich gewählt zu haben, der den Kindern es leichter macht die Addition zu verstehen und Spaß an dem Spiel zu haben.
\section{Serious Games}
Für das Spiel verwenden wir den Ansatz eines 'Serious Games', aber was sind 'Serious Games' eigentlich? Für diese Art von Spiel gibt es zur heutigen Zeit noch weitere Bezeichnungen, wie 'Educational Game' , 'Game-Based-Learning' oder 'Edutainment'. Dabei ist es schwierig zu sagen ob alle Begriffe die gleichen Spiele kategorisieren. Eine mögliche Kategorisierung orientiert sich am Verhältnis von didaktischen zu spielerischen Elementen \cite{?}. Dabei sind 'Serious Games' stark didaktisch ausgeprägt, während Edutainment sich stärker an unterhaltende spielerische Elemente orientiert. Es gibt aber auch Definitionen, die allen Begriffen die gleiche Bedeutung zuordnen. Wie 2013 von Hawlitschek\cite{3832533915} (p.23)
\begin{quote}
\textit{... digitale Lernspiele sind Computerspiele,}
\begin{itemize}
\item \textit{die explizit und systematisch in Hinblick auf ein bestimmtes Lernziel und für den Einsatz in einem pädagogischen Kontext konzipiert wurden.}
\item \textit{die ein positives Spielerlebnis beim Spieler auslösen.}
\item \textit{deren Effektivität bei der Vermittlung der Lerninhalte nachgewiesen werden konnte.}
\end{itemize}
\end{quote}
In dieser Arbeit möchten wir ein solches digitales Lernspiel nach dieser Definition implementieren und vor allem den zweiten Punkt, des positiven Spielerlebnisses, untersuchen.\\
\\
In den Folgenden Kapiteln möchte Ich aktuelle Erkenntnisse der Forschung präsentieren sowie Aufbau und Evaluation des entwickelten Spieles.
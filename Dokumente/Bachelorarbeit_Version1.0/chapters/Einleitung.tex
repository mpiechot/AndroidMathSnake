% !TEX root = ../ausarbeitung.tex

\chapter{Einleitung}
Kinder sitzen oftmals mehrere Stunden verzweifelt vor ihren Mathe-Hausaufgaben. Diese dienen in der Schule um gelernten Stoff durch weitere Übungen zu festigen. Hat das Kind aber bereits nur noch das Spielen im Kopf entwickelt sich die Hausaufgabenzeit zu einer langwierigen Arbeitszeit mit mäßigem Fortschritt. Mit dieser Bachelorarbeit möchten wir untersuchen ob wir das wiederholen von gelerntem Stoff in der Mathematik angenehmer für das Kind gestallten und womöglich auch bessere Lernerfolge herbeiführen können. Dies wird durch den Ansatz der Serious Games versucht indem wir das Lernen mit dem Spielen kombinieren. Das Thema Serious Games wird immer wichtiger in der heutigen Zeit. Dabei handelt es sich um eine Lernmethode mit der man spielerisch lernen kann. Wir verwenden dieses Konzept um ein Spiel zu entwickeln, welches dem Kind Spaß macht und es dadurch leicht den Stoff aus der Schule wiederholen kann.
\\
Für viele Kinder ist Mathe eines der unbeliebtesten Fächer in der Schule, da die Kinder hier zunächst lernen wie man rechnet, aber nicht warum es genau so ist. Diese Schwierigkeiten betreffen vor allem die Grundoperationen der Mathematik, also die Addition sowie Subtraktion, Multiplikation und Division. Das Lernen dieser Grundoperationen ist oft langwierig und langweilig für die Kinder. Hier haben wir untersucht ob zu diesem Thema ein Spiel für mehr begeisterung sorgen kann. 

\section{Partnerzahlen}


%%% DELETE THIS LATER
Die Einleitung ist (ebenso wie die Zusammenfassung) das Aushängeschild Ihrer Arbeit. Nach der Einleitung entscheidet ein Leser, ob er den Rest der Arbeit überhaupt lesen möchte! Daher sollten Sie vor allem die Einleitung mehrfach überarbeiten, damit sie sich angenehm liest und kurz und präzise die folgenden Punkte beschreibt:
\begin{itemize}
  \item Was wurde in der Arbeit untersucht? Was ist daran neu?
  \item evtl: Im Rahmen welches größeren Projektes am Lehrstuhl wurde die Arbeit durchgeführt?
  \item Warum ist die Arbeit wichtig?
  \item Warum ist die Arbeit schwierig?
  \item Was sind die (wissenschaftlichen) Beiträge der Arbeit?
  \item Welche ähnlichen Arbeiten/Lösungsansätze für ähnliche Probleme gibt es schon? Wie unterscheiden sich diese von dem in der Arbeit gewählten Lösungsweg? (Diese Fragen können auch am Ende der Arbeit oder in einem eigenen Gliederungspunkt beantwortet werden.)
\end{itemize}

Der letzte Abschnitt der Einleitung enthält oft einen Überblick über die darauffolgenden Kapitel, als eine Art \qq{Karte} für den Leser. Dieser Teil ist aber eher eine \qq{Pflichtübung}, die weniger wichtig ist als die obigen Inhalte.

Die gesamte Ausarbeitung sollte eine schlüssige \qq{Geschichte} erzählen und die Einleitung gibt diese Geschichte quasi schon vor:
\begin{itemize}
  \item Was wurde in der Arbeit untersucht?\\
  Wie passt die Arbeit ins Gesamtbild der Wissenschaftslandschaft? Diese Frage sollte am Ende bei der Diskussion/Zusammenfassung noch einmal aufgegriffen werden.
  \item Warum ist die Arbeit wichtig?\\
  Wenn sie nicht wichtig wäre, müsste man sie nicht machen. Die konkrete Fragestellung sollte sich idealerweise aus dem aktuellen Stand der Wissenschaft ergeben ($\rightarrow$ related work)
  \item Warum ist die Arbeit schwierig?\\
  Wenn sie einfach wäre, wäre sie nicht interessant. Welche Herausforderungen sind also konkret in Angriff zu nehmen? \emph{Wie} Sie die Herausforderungen gelöst haben, beschreiben Sie im Hauptteil der Arbeit.
  \item Was sind die (wissenschaftlichen) Beiträge der Arbeit?\\
  Man kann in der Einleitung auch beschreiben, was die Ziele der Arbeit sind und die Beiträge in der Zusammenfassung aufschreiben. Die Beiträge der Arbeit sollten genau den gesetzten Zielen entsprechen.\\
  Die Ziele, die Sie sich in der Einleitung setzen, führen zu entsprechenden Evaluationsmethoden. Z.B.\ wenn ein Ziel/Beitrag ist, dass die von Ihnen entwickelte Nutzeroberfläche barrierefrei ist, dann müssen Sie in der Evaluation zeigen, dass sie dies tatsächlich ist (bzw.\ zu welchem Grad sie es ist). Wissenschaftliche Beiträge sind Dinge, die Sie selbst gemacht haben und die die Wissenschaft ein Stückchen weiter voranbringen. Dies kann eine ausgiebige Literaturrecherche sein, ein Vergleich von Systemen/Nutzeroberflächen, die Implementierung eines Systems, ein Nutzertest, eine Umfrage, etc. Ihre Arbeit sollte 2--4 (3 ist perfekt) Beiträge enthalten.
\end{itemize}
% Main chapter 1:
% Einleitung 

\chapter{Einleitung}
\label{chap:Einleitung}


\section{Motivation}
\label{sec:Motivation}
Durch die Abbildung \ref{img:Mod_V-Model} zum V-Model wird sofort die elementare Wichtigkeit von Requirements ersichtlich. Durch die Verifikations- und Validierungsschritte ist es möglich die Korrektheit und Vollständigkeit des Phasenergebnisses relativ zu seiner direkten Spezifikation (Phaseneingangsdokumenten) nachzuweisen (Verifikation) \cite{SL05} und ob ein (Teil-) Produkt eine festgelegte (spezifizierte) Aufgabe tatsächlich löst (Validation) \cite{SL05}. Die Überprüfung, ob die Anforderungen die Funktionen des Systems beziehungsweise die Funktionsweise des Gesamtsystems korrekt beschreiben stellt jedoch ein Problem dar. Beispielsweise tritt eine unvollständige Spezifikation des Systems, welche nicht alle gewünschten (Teil-) Funktionen des Systems ausreichend beschreibt, womöglich erst bei den Produkttests beziehungsweise der Abnahme zu Tage.  Des weiteren können auch Uneindeutigkeiten durch rendundante, oder sogar sich wiedersprechenden Requirements auftreten.(((Gap zwischen Spezifikationen und Entwicklung mittels Zustandsautomaten schließbar und Gap zwischen Entwicklung und Test schließbar in dem Test wieder auf Zustandsautomate gemappt werden)))\\
\section{Zielsetzung}
\label{sec:Ziel}


\section{Aufbau}
\label{sec:Aufbau}


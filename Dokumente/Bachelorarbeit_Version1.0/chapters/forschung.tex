% !TEX root = ../ausarbeitung.tex

\chapter{Stand der Forschung}
Mathematik spielerisch zu erlernen ist bereits ein weit erforschtes Thema. Bereits 1992 gab es Forschungen in denen belegt wurde, dass Spiele einen positiven Einfluss auf den Lernfortschritt haben können \cite{Randel1992}. Motivation in diesem Bereich zu forschen gibt es genügend, denn mit Themen wie das Verbesserungspotential der Lehre an Schulen \cite{moeslein2018} oder dem Forschen im Bereich der Dyskalkulie \cite{wilms2015} sind noch große Themenbereiche offen in denen 'Serious Games' eine große Rolle spielen können.
\section{Dyskalkulie}
Unter Dyskalkulie versteht man eine Rechenschwäche, die bei etwa 3\% bis 8\% der Kinder und Jugendlichen diagnostiziert wird \cite{Dyskalkulie2018}. Eine Definition der World Health Organization ist unter der Ziffer F81.2 in Englisch definiert, übersetzt bedeutet diese:
\begin{quote}
\textit{ Diese Störung bezeichnet eine Beeinträchtigung von Rechenfertigkeiten, die nicht allein durch eine allgemeine Intelligenzminderung oder eine unangemessene Beschulung erklärbar ist. Das Defizit betrifft vor allem die Beherrschung grundlegender Rechenfertigkeiten wie Addition, Subtraktion, Multiplikation und Division, weniger die höheren Fertigkeiten, die für Algebra, Trigonometrie, Geometrie oder Differential- und Integralrechnung benötigt werden. }
\end{quote}
Eines der Symptome dabei ist häufig das Rechnen mit den Fingern in höheren Klassenstufen. Die Besonderheit der Dyskalkulie ist, dass bei Betroffenen oftmals ein fehlendes kardinales Verständnis vorliegt\cite{fritz2009}, sie besitzen also kein Verständnis für das Teil-Ganzes Konzept. Dieses wird damit definiert, dass Zahlen zerlegbar und aus anderen Zahlen zusammengesetzt sind. Genau dieses Konzept wird durch die Partnerzahlen geschult. Um dieses Verständnis zu lernen, können 'Serious Games' verwendet werden\cite{Sch2016} und bieten damit eine mögliche Behandlungsmethode für Dyskalkulie.
\section{Lernplattformen}
Auch über Lernplattformen werden bereits Forschritte in der Bildung erzielt und damit die Lerneffektivität gesteigert. Lernplattformen sind dabei ein Paket aus Tools, die es dem Lehrenden ermöglichen, Fortschritte des Lernenden zu messen und wenn nötig Hilfestellungen zu geben. Die Plattform stellt aber auch Tools wie Foren oder Chats bereit, um den Lernenden auch Kommunikation untereinander und Bildung von Lerngruppen zu ermöglichen. Um diese Lernfortschritte zu erreichen, stellt eine Lernplattform auch gleichzeitig Instrumente bereit um Fortschritte erzielen zu können. So wurde bereits das Prinzip der Partnerzahlen mit positiven Effekten in 'Serious Games' eingebaut und getestet \cite{JUNG2015}.
\section{Veröffentlichte Lernspiele in Mathematik}
Lernspiele sind aber nicht nur über Lernplattformen verwendbar. Viele Lernspiele gibt es auch als vollwertige Computerspiele oder als Apps in App Stores. Beispielsweise wurde 2002 das Lernspiel \textit{Addy} für den Computer veröffentlicht, welches viele der Schulfächer, unter anderem auch Mathe, behandelt. Aber auch neuere Spiele wie \textit{MathSmashers}, auf das sich in dieser Arbeit bezogen wird, oder \textit{The Counting Kingdom} sind Lernspiele, die es im App Store gibt.
\section{Fazit}
Zusammenfassend lässt sich sehen, dass Serious Games immer mehr ein wichtiger Teil der Bildung werden. Dies wird zusätzlich zu den Eigenständigen Spielen durch Systeme wie Lernplattformen bei Schulen unterstützt. Aus diesem Grund ist es wichtig zu ermitteln ob diese Spiele einen Mehrwert im lernen haben und ebenso Spaß bereiten wie Spiele bei denen nicht das Lernen im Vordergrund steht.
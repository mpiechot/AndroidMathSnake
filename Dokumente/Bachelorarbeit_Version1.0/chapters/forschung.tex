% !TEX root = ../ausarbeitung.tex

\chapter{Stand der Forschung}
Mathematik spielerisch zu erlernen ist bereits ein weit erforschtes Thema. Bereits 1992 gab es Forschungen in denen belegt wurde, dass Spiele einen positiven Einfluss auf den Lernfortschritt haben können \cite{?}. Motivation in diesem Bereich zu forschen gibt es genügend, denn durch Themen wie das Verbesserungspotential der Lehre an Schulen \cite{?} oder dem Forschen im Bereich der Dyskalkulie \cite{?} sind noch große Themenbereiche offen in denen 'Serious Games' eine große Rolle spielen können.
\section{Dyskalkulie}
Unter Dyskalkulie versteht man eine Rechenschwäche, die bei etwa 3\% bis 8\% der Kinder und Jugendlichen diagnostiziert wird \cite{BundesMinisteriumDia}. Eine Definition der World Health Organization ist unter der Ziffer F81.2 definiert:
\begin{quote}
\textit{ Diese Störung bezeichnet eine Beeinträchtigung von Rechenfertigkeiten, die nicht allein durch eine allgemeine Intelligenzminderung oder eine unangemessene Beschulung erklärbar ist. Das Defizit betrifft vor allem die Beherrschung grundlegender Rechenfertigkeiten wie Addition, Subtraktion, Multiplikation und Division, weniger die höheren Fertigkeiten, die für Algebra, Trigonometrie, Geometrie oder Differential- und Integralrechnung benötigt werden }
\end{quote}

Die Symptome dabei sind häufig das Rechnen mit den Fingern in höheren Klassenstufen. Die Besonderheit der Dyskalkulie ist, dass bei Betroffenen oftmals ein fehlendes kardinales Verständnis vorliegt\cite{fritz2009}, sie besitzen also kein Verständnis für Teil-Ganzes Beziehungen. Um dieses Verständnis zu lernen können 'Serious Games' verwendet werden\cite{bachelorarbeit} und bieten damit eine mögliche Behandlungsmethode für Dyskalkulie.
\section{Lernspiele und Lernplattformen}
Auch im Bereich der Lehre wird viel mit Serious Games geforscht.  Hier versucht man den 
Im Laufe der Zeit wurden hier viele Spiele in diesem Bereich entwickelt wie zum Beispiel 2002 dem Lernspiel Addy \cite{?}, welches viele der Schulfächer behandelt. Unter anderem auch Mathe.

Hier zeigen Sie, dass Sie über Ihr Themengebiet gut informiert sind. Sie können entweder den Stand der Forschung dafür heranziehen, um Ihr Thema zu rechtfertigen (\qq{Warum ist es wichtig?}) oder Sie können die Literatur als Grundlage Ihrer Diskussion verwenden (Wie ordnen sich Ihre Beiträge in die Wissenschaftslandschaft allgemein ein?), eine Mischung ist auch möglich.

\hfil\rule{0.4\textwidth}{0.4pt}

Dazu gehören natürlich Referenzen zu anderen Werken. Für deren Verwaltung empfiehlt sich BibTeX, die entsprechende Datei \verb|literatur.bib| (kann natürlich auch anders benannt sein) ist in der Vorlage schon enthalten. Zur Einhaltung der Syntax bietet sich ein Online-Editor wie \cite{BibTexOnlineEditor} an; TeXstudio \cite{texstudio} hält im Menü auch Hilfe beim Erstellen der Bibliographie-Einträge bereit. Für Bücher bietet z.B. auch \url{https://books.google.de/} fertige BibTeX-Einträge an.

Die Einbindung in den Text erfolgt dann mit \verb|\cite{nielsen1994usability}|, wobei der Text in den Klammern durch das in der \verb|.bib|-Datei vergebene Kürzel ersetzt werden müssen. Das Ergebnis ist dann eine Referenz zum (zumindest im Bereich \qq{Usability Engineering}) fast unverzichtbaren gleichnamigen Buch \cite{nielsen1994usability} von Jakob Nielsen.

Achtung: Ihre Ausarbeitung sollte sich (im Gegensatz zu diesem Template) weniger auf Internet-Quellen, als auf Bücher und Paper stützen!
\\

% !TEX root = ../ausarbeitung.tex

\chapter{Stand der Forschung}
Mathematik spielerisch zu erlernen ist bereits ein weit erforschtes Thema. Bereits 1992 gab es Forschungen, in denen belegt wurde, dass Spiele einen positiven Einfluss auf den Lernfortschritt haben können \cite{Randel1992}. Motivation in diesem Bereich zu forschen gibt es genügend, denn mit Themen wie dem Verbesserungspotential der Lehre an Schulen \cite{moeslein2018} oder dem Forschen im Bereich der Dyskalkulie \cite{wilms2015} sind noch große Themenbereiche offen in denen Serious Games eine Rolle spielen können.
\section{Dyskalkulie}
Unter Dyskalkulie versteht man eine Rechenschwäche, die bei etwa 3\% bis 8\% der Kinder und Jugendlichen diagnostiziert wird \cite{Dyskalkulie2018}. Eine Definition der World Health Organization\cite{who} ist unter der Ziffer F81.2 in Englisch definiert, übersetzt bedeutet diese:
\begin{quote}
\textit{ Diese Störung bezeichnet eine Beeinträchtigung von Rechenfertigkeiten, die nicht allein durch eine allgemeine Intelligenzminderung oder eine unangemessene Beschulung erklärbar ist. Das Defizit betrifft vor allem die Beherrschung grundlegender Rechenfertigkeiten wie Addition, Subtraktion, Multiplikation und Division, weniger die höheren Fertigkeiten, die für Algebra, Trigonometrie, Geometrie oder Differential- und Integralrechnung benötigt werden. }
\end{quote}
Ein häufiges Symptom der Dyskalkulie ist zum Beispiel das Rechnen mit den Fingern in höheren Klassenstufen. Die Besonderheit der Dyskalkulie ist, dass bei Betroffenen oftmals ein fehlendes kardinales Verständnis vorliegt\cite{fritz2009}, sie besitzen also kein Verständnis für das Teil-Ganzes Konzept. Dieses wird damit definiert, dass Zahlen zerlegbar und aus anderen Zahlen zusammengesetzt sind. Genau dieses Konzept wird durch die Partnerzahlen geschult. Um dieses Verständnis zu erlernen, können 'Serious Games' verwendet werden\cite{Sch2016} und bieten damit eine mögliche Behandlungsmethode für Dyskalkulie. Dies zeigt auch die Studie von Ariffin\cite{Ariffin2017}, die untersucht ob Personen mit Dyskalkulie durch Lernspiele, auf Tablet oder Handy, Fortschritte in der Mathematik erzielen.
\section{Lernplattformen}
Auch über Lernplattformen werden bereits Forschritte in der Bildung erzielt und damit die Lerneffektivität gesteigert. Eine Lernplattform ist dabei ein Paket aus Tools, das es dem Lehrenden ermöglicht, Fortschritte des Lernenden zu messen und wenn nötig Hilfestellungen zu geben. Die Plattform stellt aber auch Tools wie Foren oder Chats bereit, um den Lernenden auch Kommunikation untereinander und Bildung von Lerngruppen zu ermöglichen. Um diese Lernfortschritte zu erreichen, stellt eine Lernplattform auch gleichzeitig Instrumente bereit um Fortschritte erzielen zu können. So wurde bereits das Prinzip der Partnerzahlen mit positiven Effekten in 'Serious Games' eingebaut und getestet \cite{JUNG2015}.
\section{Veröffentlichte Lernspiele in der Mathematik}
Lernspiele sind aber nicht nur über Lernplattformen verwendbar. Viele Lernspiele gibt es auch als vollwertige Computerspiele oder als Apps in App Stores. Hierzu gibt es ein Review von Lämsä et al.\cite{Lamsa2018}, welche Charasteristiken von Lernspielen speziell in der Mathematik untersucht. Auch zu einzelnen Lernspielen in der Mathematik gibt es Studien. So wurde bereits für das Spiel \textit{Semideus School} \cite{semideus} in einer Studie von Ninaus et al.\cite{Ninaus2016} gezeigt, dass über \textit{Semideus School} größere Fortschritte im Bereich der rationalen Zahlen erzielt werden konnten, als mit konventionellen Lehrmethoden. Aber auch Spiele wie \textit{MathSmashers}\cite{ludoscience}, \textit{Meister Cody Talasia}\cite{meisterCody}, oder \textit{The Counting Kingdom}\cite{kingdom} sind Lernspiele, die als App für Smartphones und Tablet-PCs erhältlich sind und Mathematik spielend lehren.
\section{Fazit}
Zusammenfassend lässt sich sehen, dass Serious Games immer mehr ein wichtiger Teil der Bildung werden\cite{Hainey2016}\cite{Boyle2016}. Dies wird zusätzlich zu den eigenständigen Spielen durch Systeme wie Lernplattformen bei Schulen unterstützt. Das diese Spiele einen didaktischen Mehrwert bieten geht aus den genannten Quellen hervor, allerdings wurde hier nicht untersucht ob diese Spiele dem Nutzer Spaß bieten und auch mögliche Einflüsse auf das Nutzererleben wie Perspektiv-Wechsel wurde nur wenig untersucht. Denn der Spaß ist für den Erfolgs eines Serious Games ebenso wichtig wie die Effektivität.
\chapter{Appendix B}
\label{chap:AppendixB}
\section{Blbliotheken}
\label{sec:libraries}
In diesem Teil des Anhangs sind die vollständigen, in der vorliegenden Arbeiten angewandten Bibliotheken, dargestellt.
\subsection{Atomarisierungs-Bibliothek}
\label{subsec:app_atomization_lib}
Im folgenden ist die vollständige Atomarisierungs-Bibliothek (Tabelle \ref{tab:atom_lib}) mit Auflistung aller Specification Patterns gegeben. Diese gibt an welche Variablen, die als Platzhalter für Requirement States dienen, sich atomarisieren lassen. Kurz gesagt führt sie auf welche Requirement States sich in ihre Atomic States aufspalten lassen.
\begin{table}[H]
	\centering
	\begin{tabularx}{\textwidth}{p{0.22\textwidth}|p{0.22\textwidth}|p{0.22\textwidth}|p{0.22\textwidth}}
		\hline
		Specification\newline Pattern & Ausdruck & atomarisierbar & nicht \newline atomarisierbar \\ \hline
		Absence & P is false [...] & P & - \\ \hline
		Existence & P becomes true [...] & P & - \\ \hline
		Bounded Existence & transitions to P-States occur at most $n$ times [...] & P & - \\ \hline
		Universality & P is true [...] & P & - \\ \hline
		Precedence & S precedes P [...] & P & S \\ \hline
		Response & S responds to P [...] & S & P \\ \hline
		Precedence Chain & S,T precedes P [...] & P & S,T \\ \hline
		Precedence Chain & P precedes S,T [...] & T & P,S \\ \hline
		Response Chain & P responds to S,T [...] & P & S,T \\ \hline
		Response Chain & S,T responds to P [...] & T & P,S \\ \hline
		Constrained Chain Patterns & S,T without Z responds to P [...] & T & P,S,Z \\ 
	\end{tabularx}
	\caption{Atomarisierungs-Bibliothek}
	\label{tab:atom_lib}
\end{table}
%\subsection{Abbildungs-Bibliothek von SPS zu LTL}
%\label{subsec:app_sps2ltl_lib}
\subsection{Abbildungs-Bibliothek von LTL zu FOL}
\label{subsec:app_ltl2fol_lib}
\begin{table}[H]
	\centering
	\begin{tabularx}{\textwidth}{|p{0.47\textwidth}|p{0.475\textwidth}|}
		\hline
		LTL & FOL (FSM) \\ \hline
		G(Q I G (P)) & \includegraphics[width=0.3\textwidth]{images/P_is_true_after_Q.png} \\ \hline
		G NOT Q OR M (NOT P U \newline(S OR G NOT P)) & \includegraphics[width=0.3\textwidth]{images/S_precedes_P_after_Q.png} \\ \hline
		\multicolumn{2}{|c|}{\hspace{0.5cm}\includegraphics[width=0.3\textwidth]{images/LTL2FSM_description.pdf}} \\ \hline
	\end{tabularx}
	\caption{Abbildungs-Bibliothek von LTL zu FOL}
	\label{tab:ltl2fol_lib}
\end{table}
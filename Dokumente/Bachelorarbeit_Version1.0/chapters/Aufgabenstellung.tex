%\includepdf[fitpaper]{images/aufgabe(ueberarbeitet).pdf}

%\chapter[Clustern]{}
%\vspace{-145pt}
%\section*{\centering Clustern ''ähnlicher'' Testfälle mittels eines Fallbasierten Schließverfahrens}

\begingroup
\begin{sloppy}
	{\centering\textbf{\large TBD}
		
		\vspace{6pt}
		%{\centering Maximilian Schilling
		%\vspace{20pt}
		{Jan Martin}
		\vspace{20pt}
		
		\textbf{\normalsize Aufgabenstellung}}
	
	\vspace{12pt}
	TBD
	
	\vspace{12pt}
	Die M.Sc.-Thesis umfasst dabei folgende Arbeitspunkte:
	\begin{itemize}
		\setlength{\itemsep}{-3pt}
		\item[-] TBD
		\item[-] TBD
		\item[-] TBD
		\item[-] TBD
		\item[-] TBD
		\item[-] TBD
		\item[-] TBD
	\end{itemize}
	
	Die M.Sc.-Thesis wird bei der Daimler AG in der Konzernforschung PKW am Standort Sindelfingen erstellt. Die beteiligten Partner sind die Universität Stuttgart (Institut für Statik und Dynamik der Luft- und Raumfahrttechnik) und die Daimler AG (RD/FIT).
	
	\vspace{12pt}
	Bearbeitungszeitraum: 03.11. -- 03.05.2017\\
	Prüfer: Priv.-Doz. Dr.-Ing. Stephan Rudolph (ISD)\\
	Betreuer: Benedikt Walter (RD/FIT)
	
	
	\vspace{12pt}
	Note:
\end{sloppy}


\endgroup

% !TEX root = ../ausarbeitung.tex
%Unter Evaluation der Result Teil mit "nackten Zahlen" als Boxplot + Textuell %Mittelwert und Standardabweichung. Wenn möglich noch ein statistischen Test %zwischen den einzelnen Kategorien.
\chapter{Ergebnisse} %TODO Ref
In diesem Kapitel werden die Ergebnisse des Nutzertests präsentiert. Die Darstellung erfolgt nach den Kategorien, die durch den GEQ definiert wurden. Anhand von Boxplots werden die Ergebnisse der jeweiligen Kategorie von der Third-Person Ansicht denen der TopDown Ansicht gegenüber gestellt.
In Abbildung \ref{fig:topthirdbox} wird ein Überblick aller Kategorien einer Ansicht dargestellt.
\begin{figure}[h!tb]
	\centering
	\includegraphics[width=0.64\textwidth]{topDownPlot}
	\includegraphics[width=0.64\textwidth]{thirdPersonPlot}
	\caption{Boxplots der TopDown- und ThirdPerson-Perspektive\label{fig:topthirdbox}}
\end{figure}
\section{Challenge}
\begin{figure}[h!tb]
	\centering
	\includegraphics[width=0.65\textwidth]{challengePlot}
	\caption{Boxplot der Kategorie Challenge\label{fig:challengebox}}
\end{figure}
Der Durchschnitt in der Challenge Kategorie beträgt für TopDown  $1.93\overline{3} (SD = 0.55)$ und für die ThirdPerson-Variante $1.6 (SD =0.64 )$
\section{Competence}
\begin{figure}[h!tb]
	\centering
	\includegraphics[width=0.65\textwidth]{competencePlot}
	\caption{Boxplot der Kategorie Competence\label{fig:competencebox}}
\end{figure}
Der Durchschnitt in der Competence Kategorie beträgt für TopDown  $2.4 (SD = 1.01)$ und für die ThirdPerson-Variante $3.1\overline{3} (SD =0.61 )$
\newpage
\section{Flow}
\begin{figure}[h!tb]
	\centering
	\includegraphics[width=0.65\textwidth]{flowPlot}
	\caption{Boxplot der Kategorie Flow\label{fig:flowbox}}
\end{figure}
Der Durchschnitt in der Flow Kategorie beträgt für TopDown  $1.9\overline{3} (SD = 0.76)$ und für die ThirdPerson-Variante $2.5\overline{3} (SD =0.56 )$
\section{Immersion}
\begin{figure}[h!tb]
	\centering
	\includegraphics[width=0.65\textwidth]{immersionPlot}
	\caption{Boxplot der Kategorie Immersion\label{fig:immersionbox}}
\end{figure}
Der Durchschnitt in der Immersion Kategorie beträgt für TopDown  $2.4 (SD = 1.14)$ und für die ThirdPerson-Variante $3.1\overline{3} (SD =0.84 )$
\newpage
\section{Positive Affect}
\begin{figure}[htb]
	\centering
	\includegraphics[width=0.65\textwidth]{posEffPlot}
	\caption{Boxplot der Kategorie Positive Affect\label{fig:poseffbox}}
\end{figure}
Der Durchschnitt in der Positive Affect Kategorie beträgt für TopDown  $2.5\overline{3} (SD = 1.35)$ und für die ThirdPerson-Variante $3.0\overline{6} (SD =0.43 )$
\section{Negative Affect}
\begin{figure}[h!tb]
	\centering
	\includegraphics[width=0.65\textwidth]{negEffPlot}
	\caption{Boxplot der Kategorie Negative Affect\label{fig:negeffbox}}
\end{figure}
Der Durchschnitt in der Negative Affect Kategorie beträgt für TopDown  $0.2\overline{6} (SD = 0.6)$ und für die ThirdPerson-Variante $0.2 (SD =0.3 )$
\newpage
\section{Tension}
\begin{figure}[htb]
	\centering
	\includegraphics[width=0.65\textwidth]{tensionPlot}
	\caption{Boxplot der Kategorie Tension\label{fig:tensionbox}}
\end{figure}
Der Durchschnitt in der Tension Kategorie beträgt für TopDown  $0.8\overline{3} (SD = 0.67)$ und für die ThirdPerson-Variante $0.4 (SD =0.43 )$
\section{Schriftliche Fragen}
\subsection{Welches Spiel hat dir mehr Spaß gemacht?}
Diese Frage wurde im Nutzertest von allen Nutzern mit der ThirdPerson Variante beantwortet. Angegebene Gründe waren:
\begin{itemize}
\item die Kamera war näher dran und die Zahlen waren größer
\item 3D wurde als schöner empfunden
\item durch die nähere Perspektive hatte man eher das Gefühl teil des Spiels zu sein.
\end{itemize}
\subsection{Welches Spiel fandest du schwieriger?}
Hier haben vier von fünf Nutzern die TopDown Perspektive angegeben. Begründungen waren wie folgt notiert:
\begin{itemize}
\item da es für drei der Kinder das erste Spiel war und sie dadurch die Steuerung noch nicht im Griff hatten
\item Durch die Perspektive war alles zu klein und dadurch zu schwer
\end{itemize}
Eine Person hat hier die ThirdPerson Perspektive angegeben mit der Begründung, dass Äpfel, die auf den Kopf der Schlange herunterfallen, direkt als gegessen gelten.
\subsection{Hattest du Probleme dich im Spiel zurecht zu finden?}
Diese Frage wurde sehr unterschiedlich beantwortet. Zwei der Nutzer hatten keine Probleme bei beiden Versionen. Jeweils ein Nutzer hatte Probleme mit der ThirdPerson Perspektive und einer mit der TopDown Perspektive. Der letzte Nutzer des Tests hat nichts angegeben.
% !TEX root = ../ausarbeitung.tex
%Erklärung des Nutzertests mit Testgruppe, was herausgefunden werden soll:
%A: Macht so ein Spiel für Partnerzahlen Spaß?
%B: Welche Version ist besser?
\chapter{Evaluation} %TODO Ref
Im Rahmen dieser Arbeit wurde versucht über den GEQ zwei Fragen zu beantworten. Zunächst sollte überprüft werden ob dieses Additionsspiel zur Unterstützung des Lernprozesses bei Addition über Partnerzahlen den Kindern Spaß bereitet. Außerdem sollte ermittelt werden welche der beiden implementierten Versionen besser bei den Kindern ankommt. Zur Beantwortung wurde der GEQ Fragebogen abgewandelt zur KidsGEQ Variante damit Grundschüler alle Fragen verstehen und beantworten können.
\section{Game Experience Questionnaire}
Beim normalen GEQ werden 3 Module eingebaut:
\begin{itemize}
\item Core questionaire
\item Social Presence Module
\item Post-game Module
\end{itemize}
Diese Module werden direkt nach einer Spielrunde durchgegangen, dabei testen die ersten beiden Module wie der Spieler sich beim spielen gefühlt hat, während das Post-Game Module testet wie der Spieler sich nach dem beenden des spielens gefühlt hat.
\subsection{Core questionaire}
In diesem Teil werden dem Spieler Fragen aus den Kategorien Challenge, Competence, Flow, Immersion, Negative Effect, Positive Effect und Tension gestellt. Um eine gute Messung zu erzielen und Puffer zu haben um wenn nötig Fragen streichen zu können, sollte man 5 Fragen pro Kategorie verwenden. In der Auswertung sollte hier außerdem die Gewichtung der Fragen überprüft werden, da es sein kann das bei der Ausführung mit einer Frage Probleme auftreten können. Wenn sie zum Beispiel nicht verstanden wurde, kann es sein das man diese Frage aus der Auswertung entfernen muss.
\subsection{Social Presence Module}

\section{Umsetzung des KidsGEQ}
Der KidsGEQ wurde mit 21 Fragen aufgebaut. Dabei sind jeweils 3 Fragen aus den Kategorien .


Der Evaluationsteil sollte zunächst die Forschungsfragen/anvisierten Beiträge aus der Einleitung aufgreifen und daraus die verwendeten Evaluationsmethoden ableiten. Z.B. \qq{Das Ziel war es eine barrierefreie, effizente Bedienoberfläche zu erstellen, die auf verschiedenen Endgeräten läuft.} Daraus ergeben sich folgende Evaluationsziele und Methoden:
\begin{itemize}
  \item barrierefrei $\rightarrow$ Heuristiken zur Barrierefreiheit nach W3C oder BITV
  \item effizient $\rightarrow$ Messung der Antwortzeit für verschiedene Anfragen
  \item verschiedene Endgeräte $\rightarrow$ Anzeigebeispiele von verschiedenen Endgeräten oder Test durch entsprechendes Tool, das verschiedene Endgeräte simuliert.
\end{itemize}

Es bietet sich an, die Durchführung der Evaluation und ihre Beschreibung parallel zu bearbeiten, denn oft fällt erst beim Schreiben auf, dass die Evaluationsmethode nicht zu den Zielen der Arbeit passt. Die Beschreibung sollte alle Parameter enthalten, die den Versuch ausmachen, d.h. Programmparameter, Testbedingungen, Anzahl Testläufe etc.

%Unter Evaluation der Result Teil mit "nackten Zahlen" als Boxplot + Textuell %Mittelwert und Standardabweichung. Wenn möglich noch ein statistischen Test %zwischen den einzelnen Kategorien.

\hfil\rule{0.4\textwidth}{0.4pt}
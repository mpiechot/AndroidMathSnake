\documentclass[a4paper]{article}

\usepackage[utf8]{inputenc}
\usepackage[a4paper]{geometry}      
\usepackage{parskip}
\usepackage{hyperref}

\newcommand{\replaceMe}[1]{\textit{#1}}

\begin{document}

\begin{center}
\LARGE Projektplan
\end{center}
\vspace{1cm}

\section{Allgemeine Informationen}
\begin{description}
	\item[Art der Arbeit:] Bachelorarbeit
    \item[Titel:] Entwicklung und Evaluation eines Lernspiels zum Addieren für Grundschulkinder
    \item[Bearbeitungszeitraum:] 10/2018 -- 02/2019
	\item[Bearbeiter/in:] Marco Piechotta
    \item[Studiengang:] BSc Informatik
    \item[Email-Adresse(n):] marco.piechotta@ \{student.uni-tuebingen.de , gmail.com\}
\end{description}


\section{Aufgabenstellung}

\subsection{Problemstellung}
Das Thema der Arbeit wird die Entwicklung eines oder mehrerer Spiel-Konzepte sein um das Addieren / Kombinieren von Zahlen zu lernen. Dabei soll am Ende der Arbeit das Spiel anhand von etablierten Fragebögen, wie zum Beispiel \textit{Game Experience Questionnaire} oder \textit{User Experience Questionnaire}, evaluiert werden, ob das gewählte Spiel-Konzept dem Nutzer Spaß bringt und gegebenenfalls untersucht ob der Nutzer durch das Spiel messbare Fortschritte in der Additions- und Subtraktionsfähigkeit erreicht und diese Fortschritte mit herkömmlichen Methoden verglichen.\\

\subsection{Relevanz}
Durch die Bearbeitung dieses Themas können wir erkennen, ob der Nutzer Spaß an einem so genannten "Serious Game" oder dem Game-Based Learning zum schlichten Thema der Addition haben kann. Zudem können wir anhand dessen feststellen, ob die Addierfähigkeit des Nutzers gesteigert werden kann.

\section{Voraussetzungen}

\subsection{Vorkenntnisse}
Zu den Vorkenntnissen von Herr Piechotta zählen der Besuch des Praktikums: Computerspiele Special Effects I bei Prof. Lensch, sowie eigene Entwicklungen von kleineren Spielen mit Unity. Siehe zum Beispiel: \href{https://globalgamejam.org/2017/games/spaceolaf}{SpaceOlaf}. Dieses Spiel entstand Anfang 2017 beim Global Game Jam in Stuttgart an einem Wochenende.

\subsection{Geplante Techniken \& Werkzeuge}
Geplant ist als Entwicklungsumgebung Unity zu verwenden und im Zuge dessen die Programmiersprache C\#. Je nachdem, in welche Richtung das Spiel-Konzept geht, sind Vorkenntnisse für den speziellen Anwendungsfall zu erlernen.

\section{Herangehensweise}

\subsection{Ideen zur Literaturrecherche}
Ein Paper, welches bereits in diese Richtung geht und als guter Anhaltspunkt dient, wäre: \textit{Tangible Tens: Evaluating a Training of Basic Numerical Competencies with an Interactive Tabletop} von Pontual Falc$ \tilde{a} $o et al.
Weitere Stichworte für eine Suche sind wie folgt:\\
\begin{itemize}
\item Forschung Lernspiele
\item Additionsspiele
\item Game-Based Learning
\item Serious Game
\item Educational Game
\item Math Game
\item Partner Numbers
\item Gamification
\item Game Design
\item Children-Centered Design
\end{itemize}

\subsection{Erste Arbeitsschritte}
Die wissenschaftliche Arbeit unterteilt sich ich 3 Phasen:\\
\begin{itemize}
\item \textbf{Phase 1: Ideenfindung}\\In dieser Phase wird Herr Piechotta mehrere Spiel-Konzepte ausarbeiten und dokumentieren. Aus diesen wird dann eines oder mehrere zur Umsetzung ausgewählt.
\item \textbf{Phase 2: Spielumsetzung}\\In dieser Phase wird eines (oder mehrere) Spiel-Konzepte umgesetzt. Die Umsetzung wird vorraussichtlich 1-2 Monate in Anspruch nehmen.
\item \textbf{Phase 3: Evaluierung}\\Zum Abschluss wird es eine Evaluationsphase geben, in der festgestellt werden soll ob das gewählte Konzept dem Nutzer Spaß bringt, bedienbar ist etc. . Dies geschieht über etablierte Fragebögen wie dem \textit{Game Experience Questionnaire} oder dem \textit{User Experience Questionnaire}.
\end{itemize}
Dabei werden die ersten 2 Wochen voraussichtlich durch Phase 1 bis 2 bestimmt, um zunächst passende Spiel-Konzepte für die Addition zu finden, diese zu dokumentieren und aus diesen ein oder mehrere passende Konzepte auszuwählen und umzusetzen.
\end{document}
